\backmatter
\appendix
\chapter{Appendix A — Normative Vocabulary And Modal Definitions}
%---------------------------------------------------------
% AI OSI Stack v5 — Canonical Edition
% Appendix A — Vocabulary & Modal Definitions
\narrative{
Appendix A curates the canon’s vocabulary so that every reader can parse its commitments with precision. The narrative welcomes citizens, policymakers, engineers, and custodians into the Lexicon Commons, a collaborative space where words are negotiated, not assumed. A linguist named Saanvi chairs the session. She invites participants to trace the lineage of key terms—“custodian,” “persona,” “ledger,” “attestation,” “shall,” “must,” and “may”—drawing on AEIP v1, Persona v2, and prior editions of the Stack. Each term is contextualised with stories about its civic origins: how “custodian” emerged from debates about stewardship versus ownership; how “ledger” became an interpretive instrument rather than merely a technical database.

The narrative emphasises the dual-register design. Saanvi shares narrative etymologies, recounting how community advocates resisted opaque jargon by insisting on language that honoured agency. Engineers respond by mapping those narratives into modal definitions that encode obligations within AEIP manifests. Cross-references to Appendices E and O show how language shapes rights and provenance. When disagreements surface—for example, whether “audit” implies adversarial scrutiny or collaborative learning—the Commons convenes a mini-deliberation. Participants annotate the hermeneutic ledger with interpretive notes, demonstrating how vocabulary evolves transparently.

The appendix also introduces modal verbs aligned with ISO-2119 conventions. The narrative recounts a workshop where standardisation experts harmonised “SHALL,” “MUST,” “SHOULD,” and “MAY” with civic expectations. They discuss the risks of modal inflation—overusing strong requirements can erode credibility—and describe safeguards such as periodic lexicon reviews and public feedback loops. The Lexicon Commons closes by publishing a living glossary that will be maintained through Appendix L’s ledger, ensuring future revisions remain accountable to the communities they affect.
}
\normative{
Canonical vocabulary SHALL be maintained in the Lexicon Commons and versioned through Appendix L interpretive records. Definitions of core roles, artefacts, and procedures (including “custodian,” “persona,” “ledger,” “attestation,” “audit,” and “manifest”) MUST include narrative context, normative obligations, and AEIP linkage identifiers referencing Appendices E, F, and O. Revisions to vocabulary SHALL undergo public notice and comment through Appendix N transparency channels before adoption.

Modal verbs SHALL align with ISO-2119 semantics: “SHALL” and “MUST” indicate binding requirements; “SHOULD” denotes recommended practices subject to documented justification if not followed; “MAY” indicates discretionary actions that must remain consistent with canonical principles. Any deviation from these definitions MUST be explicitly annotated in the relevant document section and cross-referenced in Appendix C change logs.

Lexicon audits SHALL occur annually, convening linguistic experts, civic representatives, and implementers. Audit findings MUST document term usage patterns, identify ambiguity risks, and recommend updates. All vocabulary artifacts SHALL be published in machine-readable formats for AEIP integration and in accessible narratives for civic audiences, ensuring language remains a shared governance instrument.
}
%---------------------------------------------------------

Normative Vocabulary and
Modal Definitions
Scope: Establishes authoritative meanings of modal verbs and related normative terms used in
this specification. These definitions SHALL be treated as binding.
A.1 Modal Verbs
SHALL – Denotes a mandatory requirement. Non-fulfillment constitutes non-conformance.
SHALL NOT – Denotes a mandatory prohibition.
SHOULD – Denotes a recommended requirement; deviations MUST be justified and docu-
mented.
SHOULD NOT – Denotes a recommended prohibition; deviations MUST be justified.
MAY – Denotes an optional or permissible action with no obligation.
CAN – Denotes capability or possibility, not obligation.
A.2 Cross-Referenced Terms
Custodian – The accountable individual or entity charged with implementing and evidencing
compliance for a given layer.
Governance Disclosure Statement (GDS) – The public document summarizing confor-
mance and evidence outputs.
Integrity Ledger Entry (ILE) – Atomic, immutable record of a governance event.
AEIP Frame – Canonical message unit defined in Chapter 12.
Provenance Registry – Append-only database of data lineage entries (Layer 2).
Override Authority – Human role empowered to interrupt model operations (Layer 4).
Civic Interface – Mechanism enabling participatory oversight (Layer 8).
57
58APPENDIX A. APPENDIX A – NORMATIVE VOCABULARY AND MODAL DEFINITIONS
A.3 Interpretive Principles
1. Normative verbs SHALL be interpreted exactly as defined above.
2. Clauses marked “informative” are explanatory and carry no conformance weight.
3. When a requirement references another standard, the latest publicly available version
SHALL apply unless specified otherwise.
4. In case of conflict between textual interpretation and implementation example, the text
SHALL prevail.
5. The English edition of this document SHALL serve as the canonical reference for
translation.
