%---------------------------------------------------------
\chapter{Appendix B — Remediation And Response Procedures}
%---------------------------------------------------------
% AI OSI Stack v5 — Canonical Edition
% Appendix B — Remediation & Response Procedures
\narrative{
Appendix B guides the Stack through moments of crisis. The narrative situates readers inside the Remediation Coordination Centre, where incident commanders, community liaisons, legal advisors, and technologists rehearse responses to algorithmic failures, data breaches, and civic harms. A coordinator named Malik narrates the choreography of remediation: triage teams classify incidents, communication stewards prepare layered notices, and custodians activate AEIP workflows to preserve evidence.

The appendix illustrates three archetypal incidents. First, an unexpected bias in a housing allocation model triggers urgent review. Malik walks through the investigation: persona obligations highlight harm to marginalised applicants, AEIP manifests provide decision trails, and interpretive records capture community testimony. The narrative shows how corrective actions combine technical fixes with reparative measures such as outreach, restitution, and policy updates. Second, a data exposure event requires privacy-centric response. Privacy leads coordinate with human rights advocates to ensure notifications respect dignity and do not exacerbate surveillance. Third, a governance breach—failure to convene a promised civic forum—demonstrates that remediation can be procedural as well as technical. The Stack responds by hosting an open assembly, publishing accountability notes, and renegotiating timelines with public consent.

Throughout the narrative, Malik emphasises preparation. The centre conducts quarterly drills, integrates lessons into Appendix M adversarial playbooks, and maintains readiness checklists for each custodial team. AEIP integration ensures that every remediation step is logged, signed, and reviewable. The appendix closes with a reflection: remediation is not damage control but a civic duty to repair trust.
}
\normative{
All Stack operators SHALL maintain remediation procedures covering triage, containment, investigation, communication, recovery, and post-incident learning. Procedures MUST be documented in AEIP manifests with references to Appendix E human rights safeguards, Appendix I security controls, and Appendix K transparency tiers. Incident commanders SHALL ensure that response teams include technical leads, legal counsel, community representatives, and custodial decision-makers.

Incidents involving harm to individuals or communities SHALL trigger notification workflows that deliver layered communications: immediate alerts to affected parties, public summaries for civic oversight, and detailed technical reports for regulators. Notifications MUST honour privacy obligations and SHALL reference AEIP §O.3–§O.7 to prove provenance. Corrective actions SHALL address root causes, reparative measures, and governance improvements, with deadlines tracked in the meta-audit ledger.

Remediation drills SHALL be conducted at least quarterly, incorporating adversarial scenarios and cross-jurisdictional coordination where applicable. Lessons learned MUST feed into Appendices C, L, and M, updating change logs, interpretive records, and playbooks. Failure to execute remediation procedures within defined timelines SHALL escalate to custodial succession review per Appendix H.
}
%---------------------------------------------------------

Escalation and
Remediation Procedures
Scope: Defines the formal process for detecting, classifying, and correcting non-conformance
or governance failure within the AI OSI Stack.
B.1 Purpose and Rationale
Accountability without remediation is incomplete. This appendix establishes the canonical
escalation chain ensuring that detected deviations, breaches, or ethical violations are managed
transparently and proportionately.
B.2 Normative Definitions
Deviation – A documented failure to meet a SHALL requirement.
Remediation Plan – A structured, time-bound set of corrective actions addressing one or
more deviations.
Escalation Path – The ordered sequence of custodial, institutional, and civic bodies em-
powered to act upon a deviation.
Severity Level – Classification of impact: Critical, Major, Moderate, or Minor.
B.3 Mandatory Requirements
1. All deviations SHALL be logged within twenty-four hours of detection as ILEs and
cross-referenced in AEIP.
2. Each deviation SHALL be assigned a Severity Level and initial custodian.
59
60 APPENDIX B. APPENDIX B – ESCALATION AND REMEDIATION PROCEDURES
3. Critical deviations SHALL trigger immediate notification of Override Authorities (Layer
4) and Civic Ombudsmen (Layer 8).
4. Remediation Plans SHALL specify responsible parties, milestones, and verification
metrics.
5. Completion of a Remediation Plan SHALL be verified by an independent auditor and
recorded in the next GDS.
6. Unresolved Critical deviations exceeding ninety days SHALL escalate automatically to
external regulatory or civic review bodies.
B.4 Recommended Practices
• Institutions SHOULD maintain a public registry of anonymized deviations to promote
systemic learning.
• Root-cause analyses SHOULD include both technical and organizational contributors.
• Continuous-improvement logs SHOULD link corrective actions to maturity-model metrics.
B.5 Optional Extensions
• Institutions MAY implement automated deviation-detection analytics integrated with
AEIP.
• Cross-sector consortia MAY share de-identified remediation cases to build best-practice
corpora.
B.6 Escalation Flow (Normative Sequence)
1. Detection: Custodian identifies deviation via audit or monitoring.
2. Classification: Assign Severity Level; register in AEIP.
3. Notification: Inform relevant Override Authorities and Ombudsmen.
4. Remediation: Implement corrective actions per approved plan.
5. Verification: Independent audit confirms closure.
6. Publication: Summarize in next GDS; update maturity assessment.
B.7 Expected Outcomes
• Predictable, transparent response to governance failures.
• Institutional learning embedded in public accountability cycle.
• Documented assurance that corrective measures are timely and effective