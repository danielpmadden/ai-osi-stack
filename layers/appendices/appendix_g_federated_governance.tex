%---------------------------------------------------------
\chapter{Appendix G — Federated Governance And Policy Partnership}
%---------------------------------------------------------
% AI OSI Stack v5 — Canonical Edition
% Appendix G — Federated Governance & Policy Partnership
\narrative{
Appendix G narrates how the Stack operates across jurisdictions through federated governance. The story follows a policy partnership network linking municipalities, national regulators, and civil-society organisations. The network convenes in the Federation Council Chamber, where delegates negotiate shared standards while respecting local autonomy. Policy strategist Leila introduces the agenda: aligning transparency tiers, coordinating incident response, and harmonising legal interpretations of AI obligations.

Delegates exchange situational reports. A coastal city shares lessons from flood response scenarios, emphasising how federated agreements enabled rapid mobilisation of custodial support. A national regulator explains how harmonised AEIP manifests simplified cross-border oversight. Civil-society representatives advocate for inclusion of indigenous data governance principles, leading to an expansion of Appendix L interpretive protocols. The narrative highlights trust-building rituals, such as mutual audits and shared custodianship drills.

The appendix also explores conflict resolution. When two jurisdictions disagree about surveillance boundaries, the Council convenes a mediation panel with human rights experts. They reference Appendices E and K to ensure that any compromise upholds the non-revocable principle. The resolution includes a joint statement, shared monitoring plan, and public reporting commitments. Federated governance is portrayed as an ongoing process rather than a static treaty.

The story concludes with a federated innovation lab where partners co-create policy sandboxes. They pilot adaptive consent frameworks, distributed ledger interoperability, and civic deliberation platforms. Lessons feed back into national regulations and local charters, demonstrating the reciprocity of the partnership.
}
\normative{
Federated partners SHALL formalise governance agreements that specify shared principles, interoperability requirements, incident coordination protocols, and oversight mechanisms. Agreements MUST reference Appendices B, E, I, K, L, N, and O, ensuring alignment on remediation, rights, security, transparency, interpretive records, public attestation, and provenance. Custodial representatives from each jurisdiction SHALL sign AEIP manifests documenting the agreement.

Federated councils SHALL convene at least quarterly to review implementation status, share incident learnings, and approve updates. Meetings MUST produce minutes, decision logs, and action trackers recorded in the hermeneutic ledger. Dispute resolution procedures SHALL include mediation panels with human rights experts and civic representatives, with outcomes published through Appendix N channels.

Policy experiments conducted under federated agreements SHALL incorporate evaluation frameworks, risk assessments, and exit criteria. Results MUST be shared across the partnership, including successes, failures, and interpretive insights. Partners SHALL honour community-driven governance principles, ensuring indigenous, marginalised, and cross-border communities have representation and veto rights where their data or rights are implicated.
}
%---------------------------------------------------------
%---------------------------------------------------------