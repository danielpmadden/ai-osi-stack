%---------------------------------------------------------
\chapter{Appendix C — Change Log And Lineage}
%---------------------------------------------------------
% AI OSI Stack v5 — Canonical Edition
% Appendix C — Change Log & Lineage
\narrative{

Version History
v 4.1 – Professional Reformat (Nov 2025) LaTeX Architect Edition rebuild with com-
prehensive formatting, typographic, and normative refinements. Clarified AEIP v1
transport, expanded maturity model, and codified civic-participation mechanisms. Re-
moved duplicate Section 15.13 (Expected Outcomes) to ensure correct numbering through
Chapter 15. Added aiosi.org domain reference for official canonical hosting and public
integrity linkage.
v 4.0 – Expanded with Canonical Blueprint Integration (Nov 2025) Canonical in-
tegrated specification including AEIP v1 transport, governance maps, integrity ledger,
and offline blueprint. Established CC BY-NC-ND 4.0 licence and custodianship clauses.
Supersedes all previous releases.
v 3.0 – Epistemology Alignment (Oct 2025) Integrated Epistemology by Design and
introduced the initial AI Epistemic Infrastructure Protocol concept. Marked as non-
canonical and retained solely for historical reference.
v 2.0 – Persona Architecture Expansion (Sep 2025) Introduced layered persona con-
trol and instruction hierarchy aligning cognitive and ethical accountability.
v 1.0 – Foundational Stack Overview (Sep 2025) Original release establishing the
seven-layer conceptual architecture and baseline governance principles.

Appendix C tells the story of evolution. It guides readers through the lineage of the Stack, documenting how each version emerged from debates, experiments, and collective learning. The narrative unfolds as a guided tour of the Canonical Lineage Gallery, where timelines, ledger excerpts, and oral histories are displayed side by side. A curator named Elena narrates how v1 focused on establishing civic mandate, how v3 introduced federated governance pilots, and how v5 consolidates appendices, ledgers, and AEIP integration into a cohesive canon.

Visitors witness pivotal moments. The gallery showcases annotated AEIP manifests from crisis responses, persona revisions that reshaped service design, and international endorsements that expanded the Stack’s jurisdictional reach. Elena emphasises that lineage is not merely chronology but interpretation. Each exhibit includes commentary from citizens who experienced the policies, engineers who implemented controls, and custodians who navigated legal frameworks. The hermeneutic ledger provides audio transcripts of deliberations, preserving nuance that static documents cannot.

The appendix also recounts how the non-revocable principle was reaffirmed. After a contentious debate about surveillance technologies, the Stack codified safeguards that centre consent and proportionality. That decision is presented as a lineage milestone, linking Appendices E, K, and O. The narrative concludes by inviting readers to contribute to future lineage entries through public attestation channels, reinforcing that the canon’s history is co-authored by the community.
}
\normative{
All canonical changes SHALL be recorded in the change log with version identifiers, effective dates, responsible custodians, and references to affected chapters or appendices. Entries MUST include narrative summaries, normative rationale, and AEIP manifest identifiers that capture evidence of deliberation and approval. Change records SHALL be cross-linked to Appendix L interpretive notes and Appendix O provenance signatures.

Lineage updates SHALL be published in accessible formats, including interactive timelines and machine-readable datasets. Any alteration to canonical principles, governance structures, or obligations MUST undergo public consultation via Appendix N channels and receive custodial quorum approval per Appendix H. Rejected proposals SHALL also be logged, with rationale preserved for future reference.

Annual lineage reviews SHALL evaluate whether change logs remain comprehensive, accurate, and comprehensible to civic audiences. Reviews MUST document gaps, corrective actions, and recommendations for archive enhancement. Preservation plans SHALL ensure redundant storage of lineage records across custodial jurisdictions, safeguarding them against tampering or loss.