% AI OSI Stack v5 — Canonical Edition
% Daniel P. Madden — Independent AI Researcher
% Created: 2025-11-04
% File: 10_Layer7_Governance_Publication.tex
% Purpose: Canonical chapter content for Layer 7
\narrative{
Layer 7 transforms internal diligence into civic evidence. The narrative chronicles how governance publication becomes an
operational control rather than a public-relations gloss. Deployment teams, legal custodians, archivists, and civic reviewers
converge in a publication studio where accountability must be reconstructable. Every disclosure is sourced from AEIP receipts,
Appendix crosswalks, and signed attestations, ensuring that the public record mirrors operational reality. Publication is staged
as a relay: Layer 6 delivers evidence bundles, Layer 7 curates them into canonical packages, and the public receives artifacts
that enable independent verification without exposing sensitive materials.

\subsection*{Publication Artifacts}
The chapter opens with the choreography of Governance Disclosure Statements (GDS). Editors assemble purpose narratives, risk
summaries, control mappings, and civic commitments into dual-register documents. Technical sections draw directly from Annex IV
alignment tables, Appendix I security mappings, and GDPR accountability logs. Narrative sections translate the same facts into
accessible storytelling so that citizens, policymakers, and implementers understand the stakes. The publication studio treats
each GDS as an accountability dossier: it contains decision cards that explain major approvals, clarity packages that visualise
how data moves, and appendices that point to AEIP evidence without duplicating sensitive content.

The storyline follows a GDS release day. Custodians rehearse the disclosure with oversight councils, ensuring that obligations
under Appendix N §1 (public verification and attestation guide) are satisfied. Legal advisors double-check that no privacy
constraints are breached while still providing enough detail for audit. Engineers contribute architecture diagrams, while civic
liaisons craft plain-language summaries. By the time the GDS is published, every claim is backed by a receipt hash and cross-
referenced to operational logs.

\subsection*{Indices and Hashes}
Publication does not dump raw logs; it offers verifiable indices. Archivists curate cryptographic hash registries that correspond
to AEIP receipts, model versions, and policy statements. The narrative shows how these indices let auditors request specific
evidence without guesswork. Citizens can confirm that what they are shown matches what operators recorded. The stack’s dual-
transparency principle ensures that requesting access to sensitive data triggers reciprocal disclosures, preventing asymmetric
power.

During the narrative’s audit drill, a civil society group asks for proof that a high-risk change passed privacy validation. The
publication team points to the index entry, which includes a receipt hash, a timestamp, and a reference to Appendix I §3
crosswalk tables. Auditors submit the hash through a public verification service aligned with Appendix N §1, receiving an
attestation that the evidence remains intact. The process is transparent yet protective: personal data stays sequestered, but the
public can confirm integrity without blind trust.

\subsection*{Renewal and Metrics}
Governance publication is cyclical. Renewal schedules and performance metrics are plotted into public calendars so that no system
quietly drifts into opacity. The narrative follows the renewal sprint where custodians evaluate impact indicators, privacy
complaints, fairness metrics, and service availability. Each metric is linked to AEIP evidence paths, ensuring that numbers are
not cherry-picked. Renewal packages explain whether the system remains compliant with Annex IV obligations, whether any GDPR
Article 30 records need updating, and how mitigation plans from Appendix B have progressed.

Metrics are treated as storytelling tools. Charts show response times for civic grievances, adherence to change windows, and the
percentage of CAPA actions closed on schedule. Narrative voiceovers explain deviations and outline remedial commitments. Renewal
outputs feed into Layer 8 civic interfaces, where communities can subscribe to alerts or request deeper briefings. The public
record thus becomes an early-warning system for governance fatigue.

\subsection*{Change Notices}
Change management has its own publication cadence. Substantive API or behavioural changes trigger formal notices that are
attested, timestamped, and archived alongside prior versions. The story describes a hotfix scenario where a bias mitigation
algorithm is adjusted. Because the change modifies system behaviour, custodians prepare a Change Notice referencing the risk
thresholds defined in Layer 6. The notice summarises the rationale, evidence, and impacts. It attaches signatures from technical
leads and governance custodians, aligning with the Dual-Transparency Rule: invoking audit powers requires reciprocal artifacts
(Appendix E §3).

Historical continuity matters. The publication team maintains a reference library where citizens can compare versions, trace
superseded policies, and understand the lifecycle of each decision. Nothing vanishes; revisions are layered atop prior statements
with diff annotations. This practice upholds the principle that accountability must be reconstructable even years later.

\subsection*{Reciprocity}
Layer 7 culminates in a reciprocity exchange. When institutions exercise oversight powers—requesting logs, demanding corrective
actions, or auditing safety claims—they must also document their own governance posture. The narrative portrays a joint review
between a national regulator and a municipal cooperative. Both parties upload governance artifacts into the AEIP exchange, each
signing with their respective custodial keys (AEIP §O.7). The cooperative shares its deployment records; the regulator discloses
its auditing criteria, legal mandates, and safeguards for handling sensitive data. Reciprocity prevents oversight from becoming a
one-way extraction, aligning with Appendix E §3.

Public witnesses observe the exchange through a civic livestream where summaries are narrated in accessible language. Viewers
learn not only what the system does but how institutions hold each other accountable. Publication is therefore both a window and
a mirror: it reveals operational truth and reflects oversight commitments back to the public. Layer 7 stands as the civic
recordkeeper that keeps democratic control alive between deployments.
}
\normative{
Institutions operating at Layer 7 SHALL publish Governance Disclosure Statements that summarise purpose, risk models, control
mappings, and civic obligations. Each GDS SHALL include indices of non-sensitive AEIP receipts or corresponding hashes, enabling
stakeholders to request evidence through Appendix N §1 verification flows. Narrative explanations SHALL coexist with technical
details so that the dual-register standard is preserved for varied audiences.

Publication pipelines SHALL maintain cryptographic indices referencing Annex IV crosswalk tables, Appendix I §3 conformance
status, and GDPR accountability records. Indices MUST be updated with every substantive change and SHALL expose query interfaces
that allow auditors to confirm receipt integrity without revealing personal data. Any removal or redaction of entries MUST be
recorded with justification and AEIP signatures.

Renewal schedules and performance metrics SHALL be published alongside their data sources, timestamps, and AEIP receipt hashes.
Metrics MUST cover risk thresholds, grievance response times, CAPA closure rates, and privacy conformance outcomes. Where metrics
fall outside declared tolerances, institutions SHALL publish remediation plans referencing Appendix B actions and SHALL provide
progress updates until closure.

Substantive API or behavioural changes MUST be attested, timestamped, and recorded in Change Notices. Prior versions SHALL remain
publicly referencable, with diff annotations and links to supporting evidence. Emergency hotfixes MAY proceed within declared
change windows only if accompanied by a signed justification and an after-action AEIP bundle released through Layer 7 channels.

When invoking audit powers or requesting privileged evidence, oversight actors SHALL reciprocate by publishing their governance
artifacts, including legal authorities, handling safeguards, and accountability contacts (Appendix E §3). Dual-Transparency SHALL
apply to all federated exchanges: any request for deeper access MUST be matched with an attested statement of oversight duties
and AEIP verification tokens. Failure to reciprocate SHALL pause evidence transfer until obligations are met.
}
