% Implements: schemas/aeip/tecl-schema.json
% AI OSI Stack v5 — Canonical Deepbuild
% Chapter 08 — Layer 5: Reasoning Exchange
% Sources: Version 5 Draft, AEIP v1, Persona Architecture v2, Epistemology by Design v1, Update Plans 6, 10

\narrative{
Layer 5 manages the dialogue between systems, humans, and institutions. Reasoning exchange encompasses deliberation protocols,
explanation interfaces, and hermeneutic ledgers that capture how decisions are justified. The narrative emphasises that governance
requires more than accurate outputs; it demands transparent conversations where reasoning can be scrutinised, contested, and
revised.
}

\subsection*{Triple Register}
\textbf{Narrative Intent:} Layer 5 is designed for the human problem of inscrutable AI deliberations, forcing conversation trails to be legible, contestable, and aligned with civic norms.
\textbf{Normative Clauses:}
\begin{itemize}
\item Dialogue stewards \shall{} preserve exchange transcripts within \texttt{schemas/interpretive-trace-package.jsonld} to capture context and participants.
\item Facilitators \shall{} reconcile vocabulary conflicts using \texttt{schemas/svc/semantic-registry.jsonld} before approving automated reasoning policies.
\item Governance publishers \should{} summarise debate outcomes through \texttt{schemas/governance-decision-summary.jsonld} so civic readers see the implications.
\end{itemize}
\textbf{Plain-Speak Summary:} This layer keeps AI-human conversations accountable. It stores dialogue, checks language, and reports what decisions came out of the exchange. People can replay the reasoning trail without special tools. That visibility protects against hidden persuasion or bias.


\section*{Dialogue Protocols}
\narrative{
Dialogue protocols define how reasoning flows across stakeholders. Drawing from Persona Architecture v2 and Epistemology by Design
v1, protocols include turn-taking rules, evidence citation formats, and escalation procedures. Update Plan 6 emphasised ensuring
that dialogues remain inclusive and accessible, supporting multilingual and multimodal communication.
}
\normative{
Organizations \shall{} formalise reasoning dialogue protocols, specifying participant roles, evidence requirements, and escalation
paths. Protocols MUST be published and \shall{} reference relevant appendices, including Appendix G participation safeguards.
Implementation details \shall{} be documented in (AEIP §O.3). Systems \shall{not} engage in high-stakes reasoning without an approved
protocol.
}

\section*{Hermeneutic Logging}
\narrative{
Hermeneutic logs capture the interpretive journey of decision-making. They record questions asked, evidence considered, values
invoked, and dissent raised. The narrative explains how logs enable retrospective understanding of why a decision was made and how
it aligned with the civic mandate. Logs also provide material for future training, audits, and amendments.
}
\normative{
Reasoning exchanges \shall{} generate hermeneutic logs stored in (AEIP §O.4). Logs MUST include timestamps, participants, evidence
references, and decision outcomes. Sensitive content \shall{} be redacted in accordance with Appendix E §3, with access controls noted
in AEIP metadata. Custodians \shall{} ensure logs are discoverable for oversight and \shall{} provide summaries for public review when
full disclosure would compromise privacy.
}

\section*{Explanation Interfaces}
\narrative{
Explanation interfaces deliver reasoning to audiences in meaningful formats. The narrative explores layered explanations: technical
details for engineers, legal rationales for policymakers, and accessible narratives for citizens. Persona Architecture v2 supports
context-aware explanations tailored to community norms.
}
\normative{
Explanation interfaces \shall{} present consistent information across audiences while adapting depth and language. Interfaces MUST
reference charter clauses, mandate obligations, and AEIP evidence IDs. Updates to reasoning or outcomes \shall{} trigger interface
refreshes within defined service windows. Accessibility standards, including support for assistive technologies and multilingual
content, \shall{} be met.
}

\section*{Deliberative Review}
\narrative{
Deliberative review invites communities to evaluate reasoning quality. Citizens, experts, and oversight bodies convene to assess
whether decisions reflected values, evidence, and rights commitments. The narrative highlights review assemblies documented in
Update Plan 10, where participants debated trade-offs and recommended adjustments to charters and mandates.
}
\normative{
Deliberative reviews \shall{} occur at scheduled intervals and after major incidents. Reviews MUST include diverse stakeholders and
\shall{} be documented in (AEIP §O.6) with agendas, minutes, and outcomes. Recommendations \shall{} receive formal responses from
custodians within 60 days. Failure to conduct reviews \shall{} trigger audits under Appendix H oversight provisions.
}

\section*{Conflict Resolution and Mediation}
\narrative{
Reasoning exchange includes structured mediation when disputes arise. Mediators help reconcile conflicting interpretations of data,
values, or obligations. The narrative discusses mediation frameworks that respect community autonomy and preserve rights.
}
\normative{
Mediation protocols \shall{} be established in alignment with Appendix B remediation processes. Mediators MUST be impartial, trained,
and recorded in (AEIP §O.5). Mediation outcomes \shall{} specify whether charters, mandates, or operational policies require updates.
Parties \shall{} receive written summaries, and unresolved disputes \may{} escalate to mandate review.
}

\section*{Continuous Learning}
\narrative{
Layer 5 closes the loop by feeding insights back into governance. Lessons from reasoning exchanges inform updates to charters,
data stewardship, and model design. The narrative emphasises iterative learning, drawing parallels to the living constitution theme
in later chapters.
}
\normative{
Insights derived from reasoning exchanges \shall{} be synthesised into learning reports stored in (AEIP §O.7). Reports MUST highlight
implications for each preceding layer and \shall{} recommend actions or amendments. Custodians \shall{} track implementation of learning
recommendations and \shall{} report progress through Governance Publication channels. Neglecting continuous learning \shall{} be treated
as a governance deficiency subject to oversight intervention.
}

\n\subsection*{Verification and Enforcement}
Conformance is evidenced through artefacts \texttt{schemas/interpretive-trace-package.jsonld}, \texttt{schemas/svc/semantic-registry.jsonld}, and \texttt{schemas/governance-decision-summary.jsonld} and corresponding AEIP audit records.
% \begin{autogenerated}
\begin{autogenerated}
\begin{longtable}{p{0.22\textwidth}p{0.73\textwidth}}
\toprule
\textbf{Purpose} & Preserve contestable reasoning trails between humans and systems to surface persuasion risks and civic impacts. \ 
\midrule
\textbf{Obligations} & \begin{itemize}
\item Dialogue stewards \shall{archive transcripts and semantic diff reports} for significant deliberations.
\item Facilitators \shall{document vocabulary reconciliations} before publishing policy effects.
\item Governance publishers \should{link debate outcomes to decision summaries} accessible to the public.
\end{itemize} \ 
\midrule
\textbf{Verification Artefact} & Reasoning exchange bundle with transcript hashes, semantic registry pointers, and decision outputs. \ 
\midrule
\textbf{AEIP Schema} & \texttt{schemas/interpretive-trace-package.jsonld} \ 
\midrule
\textbf{Evidence Fields} & \begin{itemize}
\item \texttt{transcript.hash}
\item \texttt{semantic_alignment.entries[]}
\item \texttt{decision_outputs[].reference}
\end{itemize} \ 
\bottomrule
\end{longtable}
\textbf{Verification Example}\label{tab:layer5-verification}\par
\begin{lstlisting}[language=json,caption={Verification artefact for layer5},label={lst:layer5-example}]
{\n  "transcript": {\n    "hash": "sha512-33b1...",\n    "participants": ["community_forum", "stack_mediator"]\n  },\n  "semantic_alignment": {\n    "entries": ["fair_housing_glossary:v3"]\n  },\n  "decision_outputs": [\n    {\n      "reference": "GDS-2025-09"\n    }\n  ]\n}
\end{lstlisting}
\end{autogenerated}
% \end{autogenerated}

\clearpage
