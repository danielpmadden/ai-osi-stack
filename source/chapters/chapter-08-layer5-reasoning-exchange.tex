% AI OSI Stack v5 — Canonical Deepbuild
% Chapter 08 — Layer 5: Reasoning Exchange
% Sources: Version 5 Draft, AEIP v1, Persona Architecture v2, Epistemology by Design v1, Update Plans 6, 10

\narrative{
Layer 5 manages the dialogue between systems, humans, and institutions. Reasoning exchange encompasses deliberation protocols,
explanation interfaces, and hermeneutic ledgers that capture how decisions are justified. The narrative emphasises that governance
requires more than accurate outputs; it demands transparent conversations where reasoning can be scrutinised, contested, and
revised.
}

\section*{Dialogue Protocols}
\narrative{
Dialogue protocols define how reasoning flows across stakeholders. Drawing from Persona Architecture v2 and Epistemology by Design
v1, protocols include turn-taking rules, evidence citation formats, and escalation procedures. Update Plan 6 emphasised ensuring
that dialogues remain inclusive and accessible, supporting multilingual and multimodal communication.
}
\normative{
Organizations SHALL formalise reasoning dialogue protocols, specifying participant roles, evidence requirements, and escalation
paths. Protocols MUST be published and SHALL reference relevant appendices, including Appendix G participation safeguards.
Implementation details SHALL be documented in (AEIP §O.3). Systems SHALL NOT engage in high-stakes reasoning without an approved
protocol.
}

\section*{Hermeneutic Logging}
\narrative{
Hermeneutic logs capture the interpretive journey of decision-making. They record questions asked, evidence considered, values
invoked, and dissent raised. The narrative explains how logs enable retrospective understanding of why a decision was made and how
it aligned with the civic mandate. Logs also provide material for future training, audits, and amendments.
}
\normative{
Reasoning exchanges SHALL generate hermeneutic logs stored in (AEIP §O.4). Logs MUST include timestamps, participants, evidence
references, and decision outcomes. Sensitive content SHALL be redacted in accordance with Appendix E §3, with access controls noted
in AEIP metadata. Custodians SHALL ensure logs are discoverable for oversight and SHALL provide summaries for public review when
full disclosure would compromise privacy.
}

\section*{Explanation Interfaces}
\narrative{
Explanation interfaces deliver reasoning to audiences in meaningful formats. The narrative explores layered explanations: technical
details for engineers, legal rationales for policymakers, and accessible narratives for citizens. Persona Architecture v2 supports
context-aware explanations tailored to community norms.
}
\normative{
Explanation interfaces SHALL present consistent information across audiences while adapting depth and language. Interfaces MUST
reference charter clauses, mandate obligations, and AEIP evidence IDs. Updates to reasoning or outcomes SHALL trigger interface
refreshes within defined service windows. Accessibility standards, including support for assistive technologies and multilingual
content, SHALL be met.
}

\section*{Deliberative Review}
\narrative{
Deliberative review invites communities to evaluate reasoning quality. Citizens, experts, and oversight bodies convene to assess
whether decisions reflected values, evidence, and rights commitments. The narrative highlights review assemblies documented in
Update Plan 10, where participants debated trade-offs and recommended adjustments to charters and mandates.
}
\normative{
Deliberative reviews SHALL occur at scheduled intervals and after major incidents. Reviews MUST include diverse stakeholders and
SHALL be documented in (AEIP §O.6) with agendas, minutes, and outcomes. Recommendations SHALL receive formal responses from
custodians within 60 days. Failure to conduct reviews SHALL trigger audits under Appendix H oversight provisions.
}

\section*{Conflict Resolution and Mediation}
\narrative{
Reasoning exchange includes structured mediation when disputes arise. Mediators help reconcile conflicting interpretations of data,
values, or obligations. The narrative discusses mediation frameworks that respect community autonomy and preserve rights.
}
\normative{
Mediation protocols SHALL be established in alignment with Appendix B remediation processes. Mediators MUST be impartial, trained,
and recorded in (AEIP §O.5). Mediation outcomes SHALL specify whether charters, mandates, or operational policies require updates.
Parties SHALL receive written summaries, and unresolved disputes MAY escalate to mandate review.
}

\section*{Continuous Learning}
\narrative{
Layer 5 closes the loop by feeding insights back into governance. Lessons from reasoning exchanges inform updates to charters,
data stewardship, and model design. The narrative emphasises iterative learning, drawing parallels to the living constitution theme
in later chapters.
}
\normative{
Insights derived from reasoning exchanges SHALL be synthesised into learning reports stored in (AEIP §O.7). Reports MUST highlight
implications for each preceding layer and SHALL recommend actions or amendments. Custodians SHALL track implementation of learning
recommendations and SHALL report progress through Governance Publication channels. Neglecting continuous learning SHALL be treated
as a governance deficiency subject to oversight intervention.
}

\clearpage
