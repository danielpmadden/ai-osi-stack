% Chapter 00 — Introduction and Purpose
\narrative{
The AI OSI Stack rebuild begins with a problem statement: democratic societies require AI infrastructure that can be inspected, contested, and renewed by the public. Rather than treating governance as a policy appendix to technical delivery, the stack positions it as the architecture that binds mandates, data rights, model behaviour, instruction control, reasoning exchange, deployment governance, and civic participation into a layered civic system. Each layer operationalises a specific obligation — from the Civic Mandate that confers legitimacy to the Governance Publication and Civic Participation layers that maintain accountability — ensuring that transparency is coupled with enforceable agency. The introduction guides readers through how these layers interlock with the AI Epistemic Infrastructure Protocol (AEIP) to produce verifiable evidence trails and with Appendix E safeguards to sustain rights-respecting deployments.
}
\normative{
Adopting entities SHALL implement Layers 0 through 8 in full, recording AEIP evidence artifacts for every decision point, interface, and policy control (AEIP §O.3). They SHALL maintain governance disclosure registers that include release hashes, stewardship contacts, civic appeal channels, and remediation procedures aligned with Appendix E §3. Any delegation or automation of obligations SHALL remain subject to human-overseen accountability and MUST expose audit hooks for community verification. Implementers SHALL document purpose statements and risk hypotheses in AEIP submissions prior to deployment, and they SHALL publish periodic attestations confirming continuous compliance or triggering remediation under Appendix B when variance is detected.
}
\clearpage
