% AI OSI Stack v5 — Canonical Edition
% Daniel P. Madden — Independent AI Researcher
% Created: 2025-11-04
% File: 13_Persona_Architecture_Integration.tex
% Purpose: Canonical chapter content for persona architecture
\narrative{
Chapter 13 traces the Persona Architecture from design studio to civic deployment. Personas are not marketing avatars; they are interpretive vessels that encode civic commitments, legal obligations, and user agency into the AI OSI Stack. The narrative follows two intertwined journeys. First, a civic designer named Amina revisits Persona v2 templates to model the lived experience of a street medic who relies on AI triage advice. Second, a policy liaison named Arturo curates governance clauses that must bind a municipal procurement officer. Their parallel workflows demonstrate how the Persona Architecture integrates qualitative narratives, normative duties, and technical safeguards into AEIP manifests that travel across every layer.

Amina begins with interpretive fieldwork. She convenes a listening session with street medics, privacy advocates, and disability-rights organizers to capture the stakes of frontline care. The story emphasizes the dual-register approach: conversational transcripts become narrative arcs while obligations map into modal statements. When medics describe fear of surveillance, Amina encodes those concerns into persona constraints that require minimal data retention and contextual transparency disclosures referencing Appendix K. She collaborates with engineers to translate the persona into adaptive interface specifications at Layer 7, ensuring that guidance is delivered in plain language without sacrificing traceability.

Arturo, meanwhile, reviews procurement policies and international standards to craft a persona for institutional stewards. He aligns persona duties with Annex IV risk classes, EU AI Act obligations, and AEIP §O.3 lifecycle checkpoints. The narrative showcases how his normative statements inform contract language, onboarding scripts, and custodial key management described in Appendix H. Arturo cross-references Appendices E and G to guarantee that human rights safeguards and federated governance obligations remain explicit. As he assembles the persona bundle, the story highlights toolchains that generate dual outputs: narrative dossiers for training and machine-readable requirements embedded into Layer 4 control logic.

Integration culminates in a joint rehearsal. Amina and Arturo sit alongside developers, quality-assurance leads, and community observers inside the Persona Integration Lab. They load both persona manifests into the AEIP simulator, which traces how obligations flow through data collection, reasoning, deployment, and publication. The simulator surfaces tension: the street medic persona demands low-latency responses during crises, while the procurement persona mandates explicit auditing hooks. The team resolves the conflict by designing escalation tiers that balance responsiveness with attestation, referencing Appendix M’s adversarial playbook to guard against exploitative requests. The narrative affirms that personas are living documents that adapt through hermeneutic ledgers (Appendix L) rather than static compliance checklists.

The chapter closes with a civic symposium where persona storytellers present their findings alongside normative commitments. Citizens sign up to become co-curators, extending the Persona Architecture beyond the project team. The dual register is literal: participants hear the narrative rationale, then ratify the modal obligations. The Stack’s persona practice becomes a public institution, reinforcing the canon’s principle that governance is co-produced with the people it serves.
}
\normative{
Persona development SHALL follow a documented interpretive process that includes stakeholder engagement, field observation, and ethical review, with outputs recorded in the hermeneutic ledger per Appendix L. Each persona manifest MUST include narrative context, normative obligations, operational constraints, and AEIP linkage identifiers that bind the persona to Layer 0–Layer 8 artefacts.

Personas guiding frontline or high-risk interactions SHALL encode explicit privacy expectations, consent boundaries, and communication accessibility requirements. System designers MUST verify that interface components, control logic, and audit trails reflect those constraints before deployment. Conflicts between personas SHALL trigger structured deliberation sessions whose outcomes are recorded as interpretive addenda referencing Appendix O.3 provenance clauses.

Institutional personas for custodians, policymakers, or procurement officers MUST reference applicable legal frameworks (including Annex IV, GDPR, and EU AI Act obligations), federation agreements (Appendix G), and custodianship protocols (Appendix H). AEIP validators SHALL enforce persona linkage for every major change request, ensuring that evidence bundles cite the relevant persona obligations. Persona updates SHALL be versioned, co-signed by civic representatives, and communicated through public attestation channels described in Appendix N.
}
