% Implements: schemas/aeip/instruction-log-schema.json
% AI OSI Stack v5 — Canonical Deepbuild
% Chapter 07 — Layer 4: Instruction and Control
% Sources: Version 5 Draft, Persona Architecture v2, AEIP v1, Update Plans 5, 8, Appendix G

\narrative{
Layer 4 governs how humans and systems communicate through instructions, prompts, control policies, and persona interfaces. It
ensures that operational directives align with civic mandates, ethical charters, and model capabilities. The narrative explores how
instruction governance prevents misuse, preserves agency, and enables contestable control over intelligent systems.
}

\section*{Instruction Taxonomy}
\narrative{
Instruction governance begins with a taxonomy that categorises prompts, commands, supervisory policies, and automated workflows.
Persona Architecture v2 enriches this taxonomy by defining persona roles, authority scopes, and escalation paths. The narrative
explains how Update Plan 8 aligned instruction categories with AEIP evidence requirements to ensure traceability.
}
\normative{
Organizations SHALL define instruction taxonomies covering human-issued prompts, automated routines, and persona-mediated actions.
Each category MUST include allowed actions, prohibited actions, and escalation procedures. Taxonomies SHALL be recorded in (AEIP
§O.3) and SHALL be reviewed alongside ethical charters. Implementers SHALL NOT deploy instruction channels lacking taxonomy
coverage.
}

\section*{Control Policies and Safeguards}
\narrative{
Control policies translate taxonomies into operational safeguards. Policies define guardrails, rate limits, approval workflows, and
fallback procedures. The narrative describes how community oversight boards help configure policies to prevent coercive or harmful
interactions. Layer 4 ensures that control rests with authorised actors and remains contestable.
}
\normative{
Control policies SHALL specify approval requirements, monitoring triggers, and shutoff capabilities. Policies MUST be mapped to
charter clauses and SHALL be tested before activation. AEIP (AEIP §O.4) SHALL log policy configurations, change histories, and test
results. Custodians SHALL provide civic auditors with the ability to inspect and simulate control policies without risking live
operations.
}

\section*{Persona Governance}
\narrative{
Personas mediate interactions between communities and systems. Persona Architecture v2 defines persona capabilities, scripts, and
limits. The narrative illustrates how personas represent civic bodies, deliver plain-language explanations, and enforce community
constraints during dialogue with AI systems.
}
\normative{
Persona definitions SHALL include authority scope, behavioural guidelines, language profiles, and accountability hooks. Personas
MUST refuse instructions that violate mandates or charters and SHALL document refusals in (AEIP §O.4). Custodians SHALL train
personas on community-approved corpora and SHALL monitor for drift or misuse. Community representatives SHALL periodically review
persona transcripts to ensure fidelity.
}

\section*{Prompt and Response Logging}
\narrative{
Layer 4 insists on detailed logging of prompts, responses, and control actions. Logs enable oversight and remediation without
exposing sensitive data. The narrative references Appendix G participation rights, ensuring communities can inspect dialogues that
shape decisions affecting them.
}
\normative{
Instruction logs SHALL capture prompt metadata, response summaries, decision outcomes, and escalation events. Logs MUST respect
privacy constraints in Appendix E §3, employing redaction or aggregation where necessary. Access to logs SHALL be role-controlled
and documented in (AEIP §O.5). When logs reveal misuse or harm, incident workflows SHALL activate immediately.
}

\section*{Human-in-the-Loop Assurance}
\narrative{
Human oversight remains central. The narrative depicts control rooms where operators review automated decisions, intervene when
necessary, and document rationales. Update Plan 5 introduced minimum oversight coverage requirements, ensuring humans remain
accountable at critical junctures.
}
\normative{
Systems SHALL maintain human-in-the-loop checkpoints for high-risk actions. Oversight roles MUST have authority to pause or override
operations. AEIP (AEIP §O.3) SHALL record oversight decisions, including rationale and supporting evidence. Training for oversight
roles SHALL be documented, and rotations SHOULD prevent fatigue or capture. Delegation to automated monitors SHALL require explicit
charter authorization.
}

\section*{Appeals and Contestation}
\narrative{
Layer 4 ties into civic participation by providing immediate pathways to contest instructions or outcomes. Citizens can flag
harmful prompts or request review of automated responses. The narrative describes community help desks and digital portals that make
contestation accessible.
}
\normative{
Instruction channels SHALL provide appeal mechanisms embedded in user interfaces. Appeals MUST be acknowledged within defined
service levels and SHALL be tracked in (AEIP §O.7). Custodians SHALL communicate outcomes to appellants and SHALL integrate lessons
into policy updates. Retaliation against appellants SHALL be prohibited and enforced through Appendix B sanctions.
}

\clearpage
