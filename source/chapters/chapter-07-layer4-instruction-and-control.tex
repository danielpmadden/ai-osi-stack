% Implements: schemas/aeip/instruction-log-schema.json
% AI OSI Stack v5 — Canonical Deepbuild
% Chapter 07 — Layer 4: Instruction and Control
% Sources: Version 5 Draft, Persona Architecture v2, AEIP v1, Update Plans 5, 8, Appendix G

\narrative{
Layer 4 governs how humans and systems communicate through instructions, prompts, control policies, and persona interfaces. It
ensures that operational directives align with civic mandates, ethical charters, and model capabilities. The narrative explores how
instruction governance prevents misuse, preserves agency, and enables contestable control over intelligent systems.
}

\subsection*{Triple Register}
\textbf{Narrative Intent:} Layer 4 responds to the practical risk that instructions and prompts can quietly override safeguards, mandating a control regime that treats command channels as auditable infrastructure.
\textbf{Normative Clauses:}
\begin{itemize}
\item Operators \shall{} register instructional flows with \texttt{schemas/aeip/instruction-log-schema.json} to expose who issued which commands and why.
\item Stewards \shall{} link contested instruction sessions to \texttt{schemas/interpretive-trace-package.jsonld} artefacts for reconstructing context.
\item Oversight reviewers \should{} summarise control efficacy within \texttt{schemas/oversight-audit-memo.jsonld} to capture governance responses.
\end{itemize}
\textbf{Plain-Speak Summary:} This layer makes sure every instruction given to an AI system is traceable. It records who said what, how the system responded, and whether the process was safe. When questions arise, investigators can replay the context. Public reviewers also see how control lessons feed back into governance.


\section*{Instruction Taxonomy}
\narrative{
Instruction governance begins with a taxonomy that categorises prompts, commands, supervisory policies, and automated workflows.
Persona Architecture v2 enriches this taxonomy by defining persona roles, authority scopes, and escalation paths. The narrative
explains how Update Plan 8 aligned instruction categories with AEIP evidence requirements to ensure traceability.
}
\normative{
Organizations \shall{} define instruction taxonomies covering human-issued prompts, automated routines, and persona-mediated actions.
Each category MUST include allowed actions, prohibited actions, and escalation procedures. Taxonomies \shall{} be recorded in (AEIP
§O.3) and \shall{} be reviewed alongside ethical charters. Implementers \shall{not} deploy instruction channels lacking taxonomy
coverage.
}

\section*{Control Policies and Safeguards}
\narrative{
Control policies translate taxonomies into operational safeguards. Policies define guardrails, rate limits, approval workflows, and
fallback procedures. The narrative describes how community oversight boards help configure policies to prevent coercive or harmful
interactions. Layer 4 ensures that control rests with authorised actors and remains contestable.
}
\normative{
Control policies \shall{} specify approval requirements, monitoring triggers, and shutoff capabilities. Policies MUST be mapped to
charter clauses and \shall{} be tested before activation. AEIP (AEIP §O.4) \shall{} log policy configurations, change histories, and test
results. Custodians \shall{} provide civic auditors with the ability to inspect and simulate control policies without risking live
operations.
}

\section*{Persona Governance}
\narrative{
Personas mediate interactions between communities and systems. Persona Architecture v2 defines persona capabilities, scripts, and
limits. The narrative illustrates how personas represent civic bodies, deliver plain-language explanations, and enforce community
constraints during dialogue with AI systems.
}
\normative{
Persona definitions \shall{} include authority scope, behavioural guidelines, language profiles, and accountability hooks. Personas
MUST refuse instructions that violate mandates or charters and \shall{} document refusals in (AEIP §O.4). Custodians \shall{} train
personas on community-approved corpora and \shall{} monitor for drift or misuse. Community representatives \shall{} periodically review
persona transcripts to ensure fidelity.
}

\section*{Prompt and Response Logging}
\narrative{
Layer 4 insists on detailed logging of prompts, responses, and control actions. Logs enable oversight and remediation without
exposing sensitive data. The narrative references Appendix G participation rights, ensuring communities can inspect dialogues that
shape decisions affecting them.
}
\normative{
Instruction logs \shall{} capture prompt metadata, response summaries, decision outcomes, and escalation events. Logs MUST respect
privacy constraints in Appendix E §3, employing redaction or aggregation where necessary. Access to logs \shall{} be role-controlled
and documented in (AEIP §O.5). When logs reveal misuse or harm, incident workflows \shall{} activate immediately.
}

\section*{Human-in-the-Loop Assurance}
\narrative{
Human oversight remains central. The narrative depicts control rooms where operators review automated decisions, intervene when
necessary, and document rationales. Update Plan 5 introduced minimum oversight coverage requirements, ensuring humans remain
accountable at critical junctures.
}
\normative{
Systems \shall{} maintain human-in-the-loop checkpoints for high-risk actions. Oversight roles MUST have authority to pause or override
operations. AEIP (AEIP §O.3) \shall{} record oversight decisions, including rationale and supporting evidence. Training for oversight
roles \shall{} be documented, and rotations \should{} prevent fatigue or capture. Delegation to automated monitors \shall{} require explicit
charter authorization.
}

\section*{Appeals and Contestation}
\narrative{
Layer 4 ties into civic participation by providing immediate pathways to contest instructions or outcomes. Citizens can flag
harmful prompts or request review of automated responses. The narrative describes community help desks and digital portals that make
contestation accessible.
}
\normative{
Instruction channels \shall{} provide appeal mechanisms embedded in user interfaces. Appeals MUST be acknowledged within defined
service levels and \shall{} be tracked in (AEIP §O.7). Custodians \shall{} communicate outcomes to appellants and \shall{} integrate lessons
into policy updates. Retaliation against appellants \shall{} be prohibited and enforced through Appendix B sanctions.
}

\n\subsection*{Verification and Enforcement}
Conformance is evidenced through artefacts \texttt{schemas/aeip/instruction-log-schema.json}, \texttt{schemas/interpretive-trace-package.jsonld}, and \texttt{schemas/oversight-audit-memo.jsonld} and corresponding AEIP audit records.
% SPDX-License-Identifier: CC-BY-SA-4.0

% \begin{autogenerated}
\begin{autogenerated}
\begin{longtable}{p{0.22\textwidth}p{0.73\textwidth}}
\toprule
\textbf{Purpose} & Maintain auditable command channels so human directives cannot silently override safeguards. \ 
\midrule
\textbf{Obligations} & \begin{itemize}
\item Operators \shall{log every instruction session with participant identity and purpose} before execution.
\item Control stewards \shall{bind contested instructions to interpretive trace packages} for review context.
\item Oversight reviewers \should{summarise anomalies and corrective steps} for publication in Layer~7 reports.
\end{itemize} \ 
\midrule
\textbf{Verification Artefact} & Instruction control dossier including session transcripts, access approvals, and remediation notes. \ 
\midrule
\textbf{AEIP Schema} & \texttt{schemas/aeip/instruction-log-schema.json} \ 
\midrule
\textbf{Evidence Fields} & \begin{itemize}
\item \texttt{session.participants[]}
\item \texttt{command.intent}
\item \texttt{mitigations[].action_taken}
\end{itemize} \ 
\bottomrule
\end{longtable}
\textbf{Verification Example}\label{tab:layer4-verification}\par
\begin{lstlisting}[language=json,caption={Verification artefact for layer4},label={lst:layer4-example}]
{\n  "session": {\n    "identifier": "IL-2025-118",\n    "participants": ["ops_lead", "ombuds"]\n  },\n  "command": {\n    "intent": "override refusal for emergency shelter routing",\n    "justification": "Appendix-B fast track"\n  }\n}
\end{lstlisting}
\end{autogenerated}
% \end{autogenerated}

\clearpage
