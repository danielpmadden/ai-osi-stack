% © 2025 Daniel P. Madden. Custodial Edition – AI OSI Stack v5.0-open-core.
% Unauthorized reproductions or derivatives are not recognized custodial works.
% Refer to CANONICAL_PROVENANCE.yaml for official verification.
% SPDX-License-Identifier: CC-BY-SA-4.0

% Implements: schemas/aeip/civic-charter-schema.json
% AI OSI Stack v5 — Canonical Deepbuild
% Chapter 03 — Layer 0: Civic Mandate
% Sources: Version 5 Draft, Update Plans 1, 4, 7, AEIP v1, Persona Architecture v2

\narrative{
Layer 0 defines the civic mandate that legitimises every subsequent layer. It codifies how communities authorise, oversee, and
revoke the use of intelligent systems. The narrative emphasises that without a clear mandate, the stack collapses: technical
controls lack legitimacy, and transparency risks becoming spectacle. Historical consultations captured in Update Plans 1, 4, and 7
stress that communities must not merely be informed—they must co-govern. Layer 0, therefore, combines constitutional commitments,
participatory procedures, and evidence obligations into a single foundational covenant.
}

\subsection*{Triple Register}
\textbf{Narrative Intent:} Layer 0 responds to the democratic deficit that occurs when AI deployments skip explicit civic authorization, ensuring communities can see and challenge the mandate that permits governance technology.
\textbf{Normative Clauses:}
\begin{itemize}
\item Custodians \shall{} encode mandate terms within \texttt{schemas/aeip/civic-charter-schema.json} before any Layer 0 service is activated.
\item Oversight teams \shall{} log accountability triggers using \texttt{schemas/aeip/incident-report-schema.json} whenever mandate duties are breached.
\item Civic monitors \should{} publish summary ledgers referencing \texttt{schemas/integrity-ledger-entry.jsonld} so residents can verify fulfilment of obligations.
\end{itemize}
\textbf{Plain-Speak Summary:} This layer states that no AI system should operate without a public license to do so. It shows how to record that permission and what happens if the mandate is broken. People can read the logged promises and compare them with reality. If something goes wrong, the records explain who must fix it.


\section*{Mandate Formation}
\narrative{
Mandates originate in public deliberation. Citizens articulate goals, boundaries, and accountability expectations. Persona
Architecture v2 records how representative personas are authorised to speak on behalf of communities. Municipal charters or
institutional bylaws then formalise the mandate, referencing Appendix C for constitutional alignment. AEIP receipts log each stage:
proposal, consultation, ratification, and publication.
}
\normative{
Entities deploying the stack \shall{} secure a documented civic mandate before any technical implementation. The mandate MUST include
purpose statements, scope limitations, rights protections, and appeal pathways. Ratification procedures \shall{} meet quorum and
participation thresholds defined in Appendix G. AEIP entries in (AEIP §O.3) \shall{} record consultation minutes, voting tallies, and
ratification evidence. Implementers \shall{not} commence data collection or model development absent a valid mandate.
}

\section*{Mandate Maintenance}
\narrative{
Mandates require upkeep. Communities evolve; so must the authorisations. Layer 0 introduces periodic renewal cycles, typically
annual but adjustable per community preference. Renewal integrates learning from oversight reports, incident reviews, and civic
feedback. The narrative recounts how Update Plan 7 introduced adaptive renewal to accommodate emergent risks discovered during
pilot deployments.
}
\normative{
Custodians \shall{} schedule mandate reviews at least annually and MUST initiate an extraordinary review when material risks emerge.
Renewal processes \shall{} solicit input from affected groups, including marginalized communities identified in AEIP demographics.
Any changes \shall{} be documented in (AEIP §O.6) amendment logs and \shall{} be cross-referenced with Appendix B remediation outcomes.
If a mandate lapses, operations MUST pause until renewal is completed and published.
}

\section*{Mandate Enforcement}
\narrative{
Mandates are enforceable commitments. Oversight councils, ombuds offices, and civic auditors verify that operations adhere to the
mandated purpose. AEIP enables automatic alerts when actions deviate from authorised scope. The narrative details how communities
use dashboards to monitor compliance indicators, ensuring the mandate remains a living guardrail rather than a ceremonial charter.
}
\normative{
Oversight bodies \shall{} have real-time access to compliance dashboards and AEIP exception logs. Deviations from mandate scope MUST
trigger Appendix B remediation workflows within defined service-level windows. Custodians \shall{} provide enforcement authorities
with subpoena-ready evidence packages, including communication trails and decision logs. Implementers \shall{} document corrective
actions in (AEIP §O.5) and \shall{} brief the public on resolution outcomes through Governance Publication channels.
}

\section*{Mandate Revocation and Suspension}
\narrative{
Communities retain the right to suspend or revoke mandates. Revocation can occur when harms persist, transparency falters, or
trust erodes. Suspension mechanisms allow temporary halts while investigations proceed. Historical cases documented in
\texttt{versions/historical/} show how revocation powers restored public confidence by demonstrating that civic control is
substantive.
}
\normative{
Mandates \shall{} include explicit revocation and suspension clauses. When triggered, custodians MUST halt affected operations,
preserve evidence, and notify stakeholders within the timeframe defined in Appendix B. Revocation proceedings \shall{} be recorded in
(AEIP §O.7) communication logs and \shall{} invite independent observers. Restoration of operations \shall{} require a renewed mandate and
public attestation of corrective measures.
}

\section*{Mandate Transparency}
\narrative{
Transparency about mandates is essential for legitimacy. Layer 0 mandates public portals showing the scope, obligations, and
renewal status of each mandate. Personas representing communities provide narrative explanations so that citizens can understand
the commitments without legal training. The narrative highlights how interactive ledgers allow residents to inspect obligations and
track their fulfilment.
}
\normative{
Custodians \shall{} maintain publicly accessible mandate registries with machine-readable and human-readable formats. Registries MUST
include purpose statements, decision logs, renewal dates, and contact points for appeals. Updates \shall{} be published within 48
hours of any change. Privacy considerations \shall{} be handled according to Appendix E §3, ensuring sensitive details are protected
without obscuring accountability. Civic interfaces \shall{} comply with accessibility standards and \should{} support multilingual
presentations.
}

\n\subsection*{Verification and Enforcement}
Conformance is evidenced through artefacts \texttt{schemas/aeip/civic-charter-schema.json}, \texttt{schemas/aeip/incident-report-schema.json}, and \texttt{schemas/integrity-ledger-entry.jsonld} and corresponding AEIP audit records.
% SPDX-License-Identifier: CC-BY-SA-4.0

% \begin{autogenerated}
% Rationale: Align Layer 0 control mapping with AEIP civic charter evidence for Phase 1+ synthesis.
% [SYNTHESIZED v5 PH1+]
\begin{autogenerated}
\begin{longtable}{p{0.22\textwidth}p{0.73\textwidth}}
\toprule
\textbf{Purpose} & Anchor civic authorisation for all stack operations through community ratification and renewal checkpoints. \ 
\midrule
\textbf{Obligations} & \begin{itemize}
\item Layer stewards \shall{record ratification outcomes and quorum data} within the civic charter manifest.
\item Custodians \shall{publish renewal schedules and escalation triggers} aligned with Appendix~B timelines.
\item Oversight councils \should{link breach notices to public remediation briefings} in Appendix~K portals.
\end{itemize} \ 
\midrule
\textbf{Verification Artefact} & AEIP Civic Charter dossier, including consultation minutes and ratification seals. \ 
\midrule
\textbf{AEIP Schema} & \texttt{schemas/aeip/civic-charter-schema.json} \ 
\midrule
\textbf{Evidence Fields} & \begin{itemize}
\item \texttt{civic_mandate.purpose_statement}
\item \texttt{ratification.quorum_evidence}
\item \texttt{revocation.triggers[]}
\end{itemize} \ 
\bottomrule
\end{longtable}
\textbf{Verification Example}\label{tab:layer0-verification}\par
% Rationale: Provide a representative civic charter evidence payload for verification walkthroughs.
% [SYNTHESIZED v5 PH1+]
\begin{lstlisting}[language=json,caption={Verification artefact for layer0},label={lst:layer0-example}]
{\n  "civic_mandate": {\n    "purpose_statement": "Safeguard equitable housing allocation",\n    "affected_population": "Ward 7 applicants"\n  },\n  "ratification": {\n    "vote_tally": {\n      "affirm": 4821,\n      "oppose": 317\n    },\n    "quorum_evidence": "https://records.example.gov/ward7-vote"\n  },\n  "revocation": {\n    "triggers": ["material bias finding", "non-compliant outreach"]\n  }\n}
\end{lstlisting}
\end{autogenerated}
% \end{autogenerated}

\clearpage