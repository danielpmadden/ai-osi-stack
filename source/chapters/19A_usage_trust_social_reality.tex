% SPDX-License-Identifier: CC-BY-NC-ND-4.0
\chapter{Layer 19A — Usage Trust \& Social Reality}
\section{Narrative Intent}
Layer 19A explores how deployed systems interact with social expectations and trust dynamics. It
acknowledges that technical compliance alone cannot sustain legitimacy; civic trust depends on aligning
usage patterns with the lived realities of communities, especially when models mediate public services.

\section{Normative Clauses}
\begin{enumerate}
  \item \textbf{Contextual Fit Assessments.} Custodians SHALL conduct social reality assessments prior to major
        deployments, documenting cultural norms, historical sensitivities, and accessibility needs.
  \item \textbf{Trust Feedback Channels.} Ongoing usage SHALL be accompanied by feedback channels that measure
        perceived trustworthiness, with results reviewed alongside technical metrics.
  \item \textbf{Narrative Integrity.} Public communications about system capabilities SHALL be accurate,
        avoiding hype that could erode trust or misrepresent civic protections.
  \item \textbf{Remedy Pathways.} Individuals SHALL have accessible pathways to contest outcomes, seek
        remediation, and obtain human review when automated decisions diverge from social expectations.
\end{enumerate}

\section{Plain-Speak Summary}
This layer keeps the system grounded in real life. Teams study the communities they serve, collect
feedback about trust, speak honestly about what the system can do, and make sure people can challenge
its decisions when reality does not match the promise.

\cleardoublepage

% Authored and maintained solely by the Custodial Editorial Committee.
% This is a non-operational, publication-grade governance artifact.
% No AEIP runtime specs or machine-readable schemas are included.
