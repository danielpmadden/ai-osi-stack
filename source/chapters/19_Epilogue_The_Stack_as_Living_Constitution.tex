% AI OSI Stack v5 — Canonical Edition
% Daniel P. Madden — Independent AI Researcher
% Created: 2025-11-04
% File: 19_Epilogue_The_Stack_as_Living_Constitution.tex
% Purpose: Canonical chapter content for epilogue
\narrative{
The epilogue reflects on the stack as a living constitution that binds technologists, policymakers, and communities into a shared guardianship of intelligence infrastructure. It retraces how successive volumes translated civic expectations into layered safeguards, each amendment stewarded through participatory hearings, ledger attestations, and reparative commitments. Readers are invited to imagine future custodians extending the covenant to new domains—climate resilience hubs, local knowledge commons, indigenous data trusts—while carrying forward the humility that governance must remain accountable to those most affected. (Appendix H §3)

Narrative vignettes highlight convergence moments when citizen juries, academic fellows, and public servants co-authored interpretive addenda to resolve emerging dilemmas. These stories emphasise that legitimacy is earned by inviting dissent, nurturing solidarity networks, and recognising that constitutional stewardship is perpetual work. The stack closes by dedicating v5 to the communities who insisted that automation serve public purpose, reminding future editors that continuity depends on open scholarship, transparent tools, and intergenerational care. (Appendix O §1)
}
\normative{
Version 5 of the AI OSI Stack SHALL remain accessible for non-commercial study and civic implementation under the CC BY-NC-ND 4.0 licence, with custodians publishing canonical artefacts, manifests, and proofs in public repositories. Future editions MUST preserve authorship integrity by maintaining provenance ledgers, contribution histories, and community acknowledgements; any derivative SHALL reference v5 as foundational lineage and document interpretive departures. (Appendix O §1)

Governance stewards SHALL maintain open channels for critique, innovation, and redress, including public comment dockets, participatory audits, and restorative justice pathways. Amendments to the stack MUST align with the civic mandate, ethical charter, and resilience commitments articulated across prior chapters, demonstrating through AEIP manifests how new controls uphold community-centred governance. In all deliberations, decision-makers SHALL prioritise human dignity and ecological balance over performance metrics, reaffirming that the living constitution endures only through continual practice and shared custodianship. (Appendix H §3)
}
