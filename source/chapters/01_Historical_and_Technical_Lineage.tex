% AI OSI Stack v5 — Canonical Edition
% Daniel P. Madden — Independent AI Researcher
% Created: 2025-11-04
% File: 01_Historical_and_Technical_Lineage.tex
% Purpose: Chapter 01 rewritten tracing lineage with normative clauses
\narrative{
The historical arc of the AI OSI Stack begins with the Open Systems Interconnection (OSI) model, a communications framework that separated complex networks into legible layers. Governance practitioners adapted that insight to civic technology in the early 2020s, experimenting with layered mandates that could coordinate policy, engineering, and oversight. Versions 1 and 2 of the stack functioned as prototypes, mapping democratic intent to technical control points. Version 3 responded to early deployments, introducing civic appeals and data stewardship scaffolds. Version 4 matured the approach with annexes, public consultations, and AEIP lifecycle tooling that connected provenance records to operational evidence. Version 5 now formalises the stack as an epistemic governance architecture: each layer expresses a commitment to transparency-without-surveillance, translating ontological assumptions, epistemic proofs, and axiological safeguards into auditable design decisions.

The lineage is both historical and technical. Appendix C documents how each edition incorporated specific regulatory frameworks, from GDPR accountability to Annex IV crosswalks, while the Canonical Provenance Statement preserves hashes and signatures for every release. (AEIP §O.6) situates lineage as an operational requirement, ensuring that when code changes, data schemas evolve, or instructions are reinterpreted, the historical record remains intact. The narrative invites readers to see lineage as a living heritage: understanding the past is prerequisite to governing the future. The stack retains openness to future amendments, yet insists that continuity be maintained so communities can trace accountability across time and jurisdiction.
}
\normative{
\begin{itemize}
  \item \textbf{Lineage Declaration.} Custodians SHALL document the origin, influences, and revisions of each layer within the AEIP lineage register, citing prior versions, annex crosswalks, and public consultations to maintain interpretive continuity.
  \item \textbf{Contextual Consistency Requirement.} Implementers SHALL assess new deployments against historical assumptions, ensuring that inherited safeguards remain valid within the current legal, cultural, and technical context; deviations MUST be recorded with rationale and mitigation steps in Appendix C supplements.
  \item \textbf{Version Integrity Clause.} Any modification, fork, or derivative guidance SHALL be accompanied by an AEIP §O.6 integrity bundle containing hashes, signatures, and supersession notices; absent such evidence, the derivative SHALL NOT be represented as compliant with the canonical AI OSI Stack.
\end{itemize}
}
\clearpage
