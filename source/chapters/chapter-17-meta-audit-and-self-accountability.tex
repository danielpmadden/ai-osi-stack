% SPDX-License-Identifier: CC-BY-SA-4.0

% Implements: schemas/integrity-ledger-entry.jsonld
% AI OSI Stack v5 — Canonical Edition
% Daniel P. Madden — Independent AI Researcher
% Created: 2025-11-04
% File: 17_Meta_Audit_and_Self_Accountability.tex
% Purpose: Canonical chapter content for meta-audit and self-accountability
\narrative{
Chapter 17 chronicles the Meta-Audit Guild, a body tasked with auditing not only systems but the governance process itself. Meta-audit is framed as civic self-accountability: a structured practice that interrogates whether the Stack keeps its promises, honours its principles, and evolves responsibly. The narrative follows guild members preparing for the annual canonical audit, which reviews evidence across AEIP ledgers, persona archives, and custodial decisions.

The story begins with scoping. The guild assembles citizen observers, international partners, and subject-matter experts to define audit questions. They consult Appendix C change logs, Appendix O provenance signatures, and Appendix N public attestations to identify areas of concern. The audit scope spans compliance verification, interpretive integrity, and impact evaluation. Guild members craft hypotheses about potential drift—has transparency slipped into surveillance? Have custodians upheld succession protocols? Are interpretive records accessible to communities? Each hypothesis is tied to AEIP manifests to ensure evidence can be traced.

Fieldwork follows. Audit teams conduct interviews with engineers, custodians, and civic delegates. They replay system behaviours using AEIP sample manifests and verify that signatures remain valid. When the guild inspects a dispute resolution case, they traverse Appendix L’s hermeneutic ledger to confirm that citizen appeals were recorded, adjudicated, and communicated through Appendix K transparency tiers. The narrative reveals tensions: an external partner questions whether remediation timelines met Appendix B requirements, prompting auditors to dig into incident bundles. They discover delays caused by resource constraints and document them for public reporting.

The meta-audit culminates in a public accountability session. Guild members present findings in dual register: narrative stories that explain what occurred and normative assessments that classify obligations as met, partially met, or breached. Where breaches are identified, the guild issues binding corrective actions with deadlines and custodial owners. The session invites civic questions, encouraging communities to challenge interpretations and request additional evidence. AEIP manifests are updated with audit references, ensuring that future reviews can evaluate whether corrective actions were satisfied.

The chapter closes by emphasising self-renewal. The Meta-Audit Guild publishes a methodological appendix describing lessons learned, tool improvements, and areas for expansion—such as integrating algorithmic impact assessments or extending participation to international observers. Self-accountability is portrayed as a cycle that strengthens legitimacy by confronting shortcomings openly.
}

\subsection*{Triple Register}
\textbf{Narrative Intent:} Meta-audit addresses the question of who watches the guardians, describing how the stack inspects its own controls and invites external challenge.
\textbf{Normative Clauses:}
\begin{itemize}
\item Audit anchors \shall{} produce oversight verdicts via \texttt{schemas/oversight-audit-memo.jsonld} to display method and scope.
\item Custodians \shall{} register self-audit findings within \texttt{schemas/integrity-ledger-entry.jsonld} to expose remedial actions.
\item Governance leads \should{} compare self-assessment outcomes with \texttt{schemas/aeip/aeip-frame-schema.json} to ensure each layer remains covered.
\end{itemize}
\textbf{Plain-Speak Summary:} This chapter explains how the stack audits itself. It records findings in shared ledgers and templates. The process keeps self-review honest by inviting outside scrutiny. Readers can trace how identified issues are resolved.

\normative{
The Stack \shall{} maintain an independent Meta-Audit Guild with authority to evaluate governance processes, custodial performance, and system impacts. Meta-audits MUST occur annually and \shall{} encompass scoping, fieldwork, public reporting, and follow-up verification. Audit scopes \shall{} cross-reference AEIP manifests, Appendices B, C, E, H, K, L, N, and O to ensure comprehensive coverage of obligations.

Audit evidence \shall{} include interviews, document reviews, system replays, and ledger inspections. Findings MUST be documented in AEIP audit manifests with signed attestations from guild leads and citizen observers. Identified breaches or partial fulfilments \shall{} trigger corrective action plans specifying owners, remediation steps, and deadlines. Progress \shall{} be tracked through the meta-audit ledger and reviewed at least quarterly until closure.

Public accountability sessions \shall{} present audit results in accessible formats, aligning with transparency tiers in Appendix K. Responses from custodians MUST be recorded and appended to the audit record. Failure to implement corrective actions within agreed timelines \shall{} escalate to custodial succession reviews per Appendix H and, where relevant, federated governance forums outlined in Appendix G.
}
\n\subsection*{Verification and Enforcement}
Conformance is evidenced through artefacts \texttt{schemas/oversight-audit-memo.jsonld}, \texttt{schemas/integrity-ledger-entry.jsonld}, and \texttt{schemas/aeip/aeip-frame-schema.json} and corresponding AEIP audit records.
