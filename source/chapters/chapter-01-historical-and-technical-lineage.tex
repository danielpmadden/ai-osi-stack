% Implements: tools/version-check.py
% AI OSI Stack v5 — Canonical Deepbuild
% Chapter 01 — Historical and Technical Lineage
% Sources: Version 4 Master, Version 5 Draft, Update Plans 1–10, Canonical Provenance Statement

\narrative{
The lineage chapter recounts how the AI OSI Stack matured from early civic prototypes into the canonical Version 5 architecture.
Version 4 established the basic layering pattern, but placeholders persisted in the introductory volumes. Version 5 draws on the
full historical archive to replace those gaps with verifiable commitments. The narrative traces contributions from municipal
pilots, international policy dialogues, and independent research labs that tested the stack’s feasibility. The canonical
provenance statement anchors each milestone to signed releases, while Update Plans 1–10 reveal the deliberative adjustments that
brought the stack into alignment with community demands.
}

\subsection*{Triple Register}
\textbf{Narrative Intent:} The lineage chapter addresses the recurring problem of institutional amnesia, detailing how public records, technical decisions, and civic mandates evolved so implementers do not reinvent past mistakes.
\textbf{Normative Clauses:}
\begin{itemize}
\item Archivists \shall{} compile provenance packets using \texttt{schemas/interpretive-trace-package.jsonld} to demonstrate continuity from prior stack versions.
\item Custodians \shall{} register lineage claims within \texttt{schemas/integrity-ledger-entry.jsonld} so that auditors can trace when obligations first appeared.
\item Policy leads \should{} cite Layer 0 mandates alongside \texttt{schemas/aeip/aeip-frame-schema.json} entries when briefing legislators on the stack’s history.
\end{itemize}
\textbf{Plain-Speak Summary:} This chapter retells how the stack was built and why each revision mattered. It urges teams to keep a public memory so lessons from earlier deployments stay visible. Readers learn which artefacts capture these commitments. Anyone checking the history can inspect those logs to confirm what changed and why.


\section*{Early Experiments}
\narrative{
The stack began as a response to civic concern: automated decision systems were deployed without clear accountability. Early
iterations documented in Version 4 Master emphasised transparency but lacked enforceable hooks. Civic technologists in 2023
experimented with AEIP prototypes to bind decisions to evidence receipts. These experiments demonstrated that governance could be
instrumented without sacrificing privacy when Appendix E principles were respected. The narrative recounts how community hackathons
and oversight pilots stress-tested disclosure protocols and seeded the participatory ethos now embedded in the stack.
}
\normative{
Custodians \shall{} maintain archival continuity between early experiments and current practices. Implementers referencing historical
material MUST cite the corresponding AEIP receipt or update plan identifier. When lessons from early pilots influence new controls,
organizations \shall{} document the lineage in (AEIP §O.6) amendment records and \shall{} acknowledge contributors consistent with Appendix
G participation standards. Historical artifacts \may{} be redacted for privacy but \shall{} remain discoverable through metadata indices.
}

\section*{Technical Architecture Evolution}
\narrative{
Technical lineage follows the adoption of layered modularity. Version 4 introduced the civic-to-technical ladder, but Version 5
reengineers each interface with AEIP hooks, persona-aware controls, and reasoning safeguards. Engineers refined data schemas,
introduced custodial APIs, and codified dual-register publication templates. The lineage charts how each component matured: civic
mandates obtained structured consultation logs; data stewardship integrated differential privacy thresholds; model development
embedded traceability pipelines; instruction governance harnessed persona matrices to avoid role confusion; reasoning exchange
adopted hermeneutic logging to preserve deliberative context.
}
\normative{
Implementers \shall{} reference this technical lineage when designing compatible systems. Architecture documents MUST map each module
to the relevant layer obligations, including Appendix crosswalks. Changes to shared schemas \shall{} undergo AEIP change-control with
notifications recorded in (AEIP §O.5). Integrations \shall{not} bypass custodial APIs, and all persona-aware interfaces MUST adhere to
Persona Architecture v2 safeguards. Engineering teams \should{} conduct lineage reviews prior to major releases to confirm alignment
with canonical patterns.
}

\section*{Policy and Legal Harmonisation}
\narrative{
The stack’s lineage is also legal. Update Plans 3 and 4 translated civic mandates into regulatory language suitable for municipal
charters and international frameworks. The Version 5 Draft integrates references to data protection statutes, algorithmic
accountability acts, and indigenous data sovereignty protocols. The narrative highlights collaborations with custodians of
community knowledge who ensured that the stack would not replicate colonial governance models. Harmonisation required bridging
lexicons: legal clauses had to coexist with engineering specifications without diluting either.
}
\normative{
Jurisdictions adopting the stack \shall{} map local laws to the canonical obligations and \shall{} publish the crosswalk in their AEIP
registries. Policy harmonisation efforts MUST preserve the non-revocable principle and \shall{} avoid weakening Appendix E §3 privacy
guarantees. Legal amendments inspired by this stack \shall{} cite the canonical edition and MUST include public consultation records.
Custodians \should{} provide comparative tables to help lawmakers align statutes with layered obligations.
}

\section*{Community Stewardship}
\narrative{
Lineage is incomplete without the people who sustained it. Community stewardship involved librarians maintaining public archives,
activists convening listening sessions, and educators creating curricula. Persona Architecture v2 emerged from these collaborations,
ensuring that digital agents representing communities operate under explicit civic mandates. The narrative recounts how community
feedback influenced every layer, from data minimisation practices to reasoning exchange safeguards.
}
\normative{
Stewardship programs \shall{} remain open to community participation. Custodians MUST provide accessible pathways for feedback and
\shall{} log submissions in (AEIP §O.7). When communities raise concerns, implementers \shall{} respond with documented remediation plans
referencing Appendix B protocols. Educational institutions \may{} adapt stewardship materials for training, but they \shall{} credit the
canonical sources and \shall{} invite civic reviewers to audit their coursework.
}

\section*{Continuity and Future Proofing}
\narrative{
The lineage closes by looking forward. Version 5 positions the stack as a living constitution, anticipating future iterations that
will integrate new forms of intelligence, data modalities, and governance arenas. The historical corpus is intentionally preserved
to enable comparative analysis; future custodians can study how the stack evolved to address emerging risks. Continuity demands
institutional humility: no layer is final, yet each amendment must honour the commitments recorded here.
}
\normative{
Future custodians \shall{} maintain version control practices that preserve historical context. Any proposal for change MUST include a
lineage assessment, \shall{} cite the affected obligations, and \shall{} publish public consultation summaries prior to ratification.
Continuity reviews \shall{} occur at least biennially and MUST involve representatives from civic, technical, legal, and academic
communities. Evolution \may{} occur, but it \shall{} never erase the public’s right to trace how governance obligations have shifted over
time.
}

\n\subsection*{Verification and Enforcement}
Conformance is evidenced through artefacts \texttt{schemas/interpretive-trace-package.jsonld}, \texttt{schemas/integrity-ledger-entry.jsonld}, and \texttt{schemas/aeip/aeip-frame-schema.json} and corresponding AEIP audit records.
\clearpage
