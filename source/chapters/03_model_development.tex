% SPDX-License-Identifier: CC-BY-NC-ND-4.0
\chapter{Layer 03 — Model Development}
\section{Narrative Intent}
\subsection*{Overview}
Layer 03 defines the controls governing model design, training, evaluation, and release preparation. It
clarifies lifecycle gating and creates explicit links to Layer 11 verification activities, ensuring that
models cannot progress without satisfying both ethical and civic checkpoints. The narrative situates
model development within a transparent pipeline that remains open to public scrutiny.

\section{Normative Clauses}
\subsection*{Development Controls}
\begin{enumerate}
  \item \textbf{Lifecycle Gates.} Custodians SHALL document and enforce decision gates for data readiness,
        architecture approval, training launch, and evaluation sign-off, each tied to accountable stewards.
  \item \textbf{Evaluation Protocols.} Evaluation plans SHALL include bias detection, robustness testing, and
        civic impact scenarios, with results archived for Appendix D audit traceability.
  \item \textbf{Traceable Artifacts.} Source code, training scripts, and configuration files SHALL be versioned
        and cross-referenced to the civic accountability charter for reproducibility.
  \item \textbf{Change Control.} Material deviations from approved designs SHALL trigger Appendix C change log
        entries and require explicit ethical debt evaluation before resuming development.
\end{enumerate}

\section{Plain-Speak Summary}
Model work cannot be ad hoc. This layer insists on clear checkpoints, documented tests, and stored
artifacts so anyone can retrace how a model came to be. If the team changes course, it must explain the
impact and get approval before moving ahead.

\emph{License: Creative Commons Attribution–NonCommercial–NoDerivatives 4.0 International.}

\cleardoublepage
