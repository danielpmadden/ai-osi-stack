% Chapter 03 — Layer 0: Civic Mandate
\narrative{
Layer 0 establishes that legitimacy precedes computation. The Civic Mandate chronicles how communities confer authority before any data is collected, model is trained, or service is launched. Consent is treated as a renewable covenant rather than a one-time signature, and custodianship is framed as a public trust that can be revoked when obligations lapse. Narrative case studies describe how citizen assemblies, policy partners, and independent auditors convened during the November 2025 rebuild to revalidate mandates and define renewal thresholds. The chapter emphasises that every subsequent layer depends on this civic grounding; without it, technical sophistication becomes an exercise in unaccountable power.
}
\normative{
Every deployment SHALL maintain a ratified Civic Charter detailing scope, beneficiaries, obligations, and renewal cadence. Custodians SHALL designate accountable stewards, succession plans, and contingency triggers consistent with Appendix H §2, documenting each element within AEIP Civic Mandate records (AEIP §O.4). Approvals, renewals, and incident responses SHALL be recorded as AEIP receipts, with signatures from public representatives and custodial leads. Public consultation artifacts — agendas, transcripts, deliberative outputs — MUST be published within the governance register, and failure to do so SHALL suspend operational authority until remediation is complete. Non-compliance or breach of consent SHALL trigger immediate review, with suspension or decommissioning mandated unless a civic appeals body restores confidence through documented deliberation.
}
\clearpage
