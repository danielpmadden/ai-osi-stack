% AI OSI Stack v5 — Canonical Edition
% Daniel P. Madden — Independent AI Researcher
% Created: 2025-11-04
% File: 03_Layer0_Civic_Mandate.tex
% Purpose: Chapter 03 rewritten for Layer 0 civic mandate
\narrative{
Layer 0 is the civic covenant that authorises every subsequent layer. Consent precedes computation: no model is trained, no dataset is ingested, and no interface is deployed until a community has ratified the mandate that defines purpose, limits, and accountability. The narrative describes councils, cooperatives, or public agencies convening assemblies to deliberate on the intended service, the rights at stake, and the mechanisms of appeal. Version 5 emphasises custodianship as public trust—custodians are not silent administrators but named stewards who hold legal, ethical, and technical responsibilities. Their legitimacy stems from transparent selection, recorded attestations, and recurring renewal rituals documented in AEIP civic bundles.

The chapter illustrates how the civic mandate anchors the entire governance architecture. Mandate charters specify lawful bases, equity benchmarks, and anticipated harms. They define how Layer 1’s ethical charter is drafted, how Layer 2’s data stewardship policies are constrained, and how Layers 3–5 prove integrity through evidence. By embedding references to (AEIP §O.4), Appendix H §2, and Appendix N §1, the narrative clarifies that civic oversight, appeals, and escalation forums are not optional extras—they are structural components that convert consent into enforceable power. When the community renews or withdraws consent, the rest of the stack must respond.
}
\normative{
\begin{itemize}
  \item A Civic Charter SHALL be ratified through documented public deliberation before any AI OSI Stack layer is implemented; the charter SHALL define scope, rights impacts, appeal structures, and oversight bodies, with records stored in AEIP civic mandate bundles (AEIP §O.4).
  \item Custodians MUST be named, vetted, and publicly disclosed within the charter; their roles, tenure, and accountability pathways SHALL be recorded in Appendix H §2-aligned registers and cross-referenced with Appendix N §1 escalation forums.
  \item Mandate renewals, amendments, or withdrawals SHALL be logged through AEIP renewal receipts within thirty days of the decision, including signatures from civic representatives and custodial stewards; absence of a current renewal SHALL be treated as non-compliance and SHALL trigger suspension procedures.
  \item Any service operating without a valid, recorded civic mandate SHALL be classified as non-compliant; custodians MUST cease deployment activities until a new mandate is ratified and registered in the canonical ledger.
\end{itemize}
}
\clearpage
