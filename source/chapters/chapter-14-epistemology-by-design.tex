% SPDX-License-Identifier: CC-BY-SA-4.0

% Implements: schemas/interpretive-trace-package.jsonld
% AI OSI Stack v5 — Canonical Edition
% Daniel P. Madden — Independent AI Researcher
% Created: 2025-11-04
% File: 14_Epistemology_by_Design.tex
% Purpose: Canonical chapter content for epistemology by design
\narrative{
Chapter 14 articulates epistemology as an engineering discipline. The Stack’s promise—transparency without surveillance—depends on designing systems that know why they know. The narrative follows an interdisciplinary guild composed of philosophers, data scientists, community advocates, and archivists who convene in the Epistemology Studio. Their charter is to operationalise Epistemology v1 across the full lifecycle. They treat epistemic virtues—justification, contestability, humility, and repairability—as infrastructure that must be built, tested, and audited.

The guild begins by mapping knowledge claims. Every model capability, policy rule, and interface assertion is rewritten as an epistemic statement with provenance references. The story walks through a workshop where a language model’s medical triage suggestion is decomposed into data lineage, inference heuristics, and normative assumptions. Participants tag each component with AEIP references to Appendices E and F to highlight human rights safeguards and operational annex obligations. They expose gaps where assumptions lack evidence or where evidence lacks community consent. Those gaps become backlog items for Layer 3 data stewardship and Layer 5 reasoning exchange teams.

Next, the chapter portrays epistemic testing. Instead of relying solely on accuracy metrics, the guild designs epistemic experiments: counterfactual simulations, narrative audits, and hermeneutic peer reviews. A citizen historian challenges the system with archival cases where prior policies produced inequitable outcomes. The AI must explain its reasoning in plain language while providing verifiable citations. When the explanation falls short, the guild triggers an epistemic corrective action that propagates through Appendix L’s ledger. The story emphasizes the role of humility—acknowledging uncertainty, documenting knowledge limits, and signalling when human judgement must intervene.

Epistemic continuity is the third act. The guild maintains interpretive dossiers that travel with systems through upgrades, migrations, and custodial succession. When a new model version is proposed, its epistemic impacts are compared against prior commitments stored in AEIP manifests and Appendices C and O. Provenance signatures ensure that historical debates remain accessible, allowing future custodians to understand why particular assumptions were accepted or rejected. The narrative shows how epistemic continuity supports civic appeals: when citizens challenge a decision, adjudicators can navigate the hermeneutic ledger to reconstruct the reasoning chain and evaluate whether obligations were honoured.

The chapter concludes with a civic colloquium where epistemic findings are published alongside normative commitments. The guild issues a public epistemic charter that invites citizens to contribute counterexamples, ask questions, and propose new interpretive frameworks. The Stack treats epistemology as a participatory commons, reaffirming that knowledge legitimacy emerges from sustained dialogue between experts and the communities they serve.
}

\subsection*{Triple Register}
\textbf{Narrative Intent:} Epistemology by design tackles the difficulty of proving that knowledge claims inside the stack have verifiable sources, aligning semantics, evidence, and audit pathways.
\textbf{Normative Clauses:}
\begin{itemize}
\item Researchers \shall{} bind contested terms to \texttt{schemas/svc/semantic-registry.jsonld} entries before they influence obligations.
\item Custodians \shall{} register provenance updates within \texttt{schemas/integrity-ledger-entry.jsonld} to preserve audit trails.
\item Analysts \should{} attach inference narratives to \texttt{schemas/interpretive-trace-package.jsonld} artefacts when documenting knowledge transitions.
\end{itemize}
\textbf{Plain-Speak Summary:} This chapter makes knowledge claims trackable. It links definitions, evidence, and reasoning in one place. Teams see which ledgers to update when facts change. That way, anyone can challenge or confirm how the stack knows something.

\normative{
AI systems governed by the Stack \shall{} maintain explicit epistemic statements for all consequential outputs, including data provenance, inference logic, normative assumptions, and interpretive boundaries. These statements MUST be recorded within AEIP manifests and cross-referenced to Appendix L interpretive records and Appendix O provenance signatures.

Epistemic testing protocols \shall{} include counterfactual analyses, adversarial interpretive reviews, and participatory audits with affected communities. Failures to justify or explain outputs MUST trigger corrective actions recorded in the hermeneutic ledger, with remediation timelines aligned to Appendix B response procedures. Systems \shall{} disclose uncertainty metrics and escalation paths whenever epistemic confidence falls below thresholds agreed in persona obligations.

Custodial transitions and model updates \shall{} include epistemic continuity reviews that compare new knowledge claims against historical commitments captured in Appendix C. Any deviation MUST be justified with new evidence, community consultation records, and AEIP signatures from custodial quorums per Appendix H. Public-facing epistemic summaries \shall{} be published through transparency tiers defined in Appendix K, enabling civic scrutiny without exposing personal data or sensitive security details.
}
\n\subsection*{Verification and Enforcement}
Conformance is evidenced through artefacts \texttt{schemas/svc/semantic-registry.jsonld}, \texttt{schemas/integrity-ledger-entry.jsonld}, and \texttt{schemas/interpretive-trace-package.jsonld} and corresponding AEIP audit records.
