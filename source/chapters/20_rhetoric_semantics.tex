% SPDX-License-Identifier: CC-BY-NC-ND-4.0
\chapter{Layer 20 — Rhetoric \& Semantics}
\section{Narrative Intent}
Layer 20 governs how language is crafted, deployed, and interpreted throughout the Stack. It recognizes
that rhetoric shapes perception and can either build or erode civic trust. The narrative sets expectations
for precision, contextual awareness, and accountability when communicating about AI behavior and
policy.

\section{Normative Clauses}
\begin{enumerate}
  \item \textbf{Controlled Vocabulary.} Custodians SHALL reference Appendix A when drafting public statements,
        ensuring that terms carry consistent meaning across audiences.
  \item \textbf{Contextual Review.} High-impact communications SHALL undergo semantic review to identify
        ambiguity, harmful framing, or exclusionary language.
  \item \textbf{Truthfulness Obligation.} Rhetorical devices SHALL not exaggerate capabilities or downplay risks;
        statements must align with verified evidence and disclosed limitations.
  \item \textbf{Feedback Integration.} Civic feedback on confusing or misleading language SHALL trigger updates to
        both communication artifacts and the controlled vocabulary.
\end{enumerate}

\section{Plain-Speak Summary}
Words matter. This layer makes sure the team uses agreed definitions, checks messages for clarity,
sticks to the facts, and updates materials when people say the language is unclear or misleading.

\cleardoublepage
