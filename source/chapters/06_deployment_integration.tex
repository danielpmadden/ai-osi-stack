% SPDX-License-Identifier: CC-BY-NC-ND-4.0
\chapter{Layer 06 — Deployment Integration}
\section{Narrative Intent}
\subsection*{Overview}
Layer 06 preserves the multi-stage integration pipeline while aligning outputs with governance
publication duties. It defines how tested models enter production, how interfaces are configured, and how
handoffs occur across operational teams. The narrative highlights collaborative ownership across
engineering, policy, and civic accountability roles.

\section{Normative Clauses}
\subsection*{Integration Controls}
\begin{enumerate}
  \item \textbf{Stage Gate Documentation.} Deployment SHALL progress through documented stages—sandbox,
        staging, limited release, and full release—with explicit exit criteria tied to civic impact assessments.
  \item \textbf{Operational Readiness.} Integration teams SHALL verify that monitoring, override controls, and
        escalation pathways are configured before public exposure.
  \item \textbf{Publication Alignment.} Deployment decisions SHALL be synchronized with governance publication
        cadences so the public receives timely updates on functionality and safeguards.
  \item \textbf{Rollback Planning.} Each deployment stage SHALL include rollback scripts and decision triggers,
        ensuring rapid restoration when controls fail or civic commitments are breached.
\end{enumerate}

\section{Plain-Speak Summary}
Deployments move through clearly defined stages. Each step checks that monitoring works, the public is
informed, and a rollback is ready if things go wrong. Integration is not done until policy, technical, and
civic teams agree the system is safe to use.

\emph{License: Creative Commons Attribution–NonCommercial–NoDerivatives 4.0 International.}

\cleardoublepage

% Authored and maintained solely by the Custodial Editorial Committee.
% This is a non-operational, publication-grade governance artifact.
% No AEIP runtime specs or machine-readable schemas are included.
