% SPDX-License-Identifier: CC-BY-NC-ND-4.0
\chapter{Layer 08 — Civic Participation}
\section{Narrative Intent}
\subsection*{Overview}
Layer 08 elevates participatory metrics and multi-stakeholder review loops. It defines how civic voices
shape governance, how feedback is processed, and how outcomes are communicated back to participants.
The narrative emphasizes co-creation and accountability to affected communities.

\section{Normative Clauses}
\subsection*{Participation Controls}
\begin{enumerate}
  \item \textbf{Stakeholder Mapping.} Custodians SHALL maintain an inclusive map of stakeholder groups,
        updating it when new constituencies emerge or civic priorities shift.
  \item \textbf{Engagement Protocols.} Each participatory mechanism SHALL document recruitment methods,
        facilitation practices, and anti-harm safeguards tailored to community needs.
  \item \textbf{Feedback Disposition.} Input received from civic participants SHALL be categorized, responded to,
        and published in a disposition log that records acceptance, modification, or rejection.
  \item \textbf{Metric Tracking.} Participation quality metrics—including representativeness, satisfaction,
        and follow-up actions—SHALL be monitored and disclosed in governance publications.
\end{enumerate}

\section{Plain-Speak Summary}
The Stack must welcome and act on public input. This layer requires teams to know who is affected,
invite them in, keep them safe while they participate, and show how their ideas change the work. People
should see the results of their contributions.

\emph{License: Creative Commons Attribution–NonCommercial–NoDerivatives 4.0 International.}

\cleardoublepage

% Authored and maintained solely by the Custodial Editorial Committee.
% This is a non-operational, publication-grade governance artifact.
% No AEIP runtime specs or machine-readable schemas are included.
