% © 2025 Daniel P. Madden — Custodial Author
% AI OSI Stack v5.0-open-core (Civic Standard Edition)

% © 2025 Daniel P. Madden. Custodial Edition – AI OSI Stack v5.0-open-core.
% Unauthorized reproductions or derivatives are not recognized custodial works.
% Refer to CANONICAL_PROVENANCE.yaml for official verification.
% SPDX-License-Identifier: CC-BY-SA-4.0

% Implements: schemas/aeip/aeip-frame-schema.json
% AI OSI Stack v5 — Canonical Edition
% Daniel P. Madden — Independent AI Researcher
% Created: 2025-11-09
% File: 09_Layer6_Deployment_and_Integration.tex
% Purpose: Canonical chapter content for Layer 6
\narrative{
Layer 6 marks the threshold where designs become civic infrastructure. Deployment is narrated here as a duty-of-care storyline:
release engineers confer with governance custodians, risk officers confer with community stewards, and every transition from
sandbox to production is traced through the AI Epistemic Infrastructure Protocol (AEIP). The narrative follows an archetypal
release train in which actors anticipate harm, stage controls, monitor outcomes, and learn from incidents without surrendering
transparency to surveillance. By the time a system graduates into production, Annex IV conformity assessments, GDPR principles,
and ENISA/ISO security mappings have already been woven into the deployment fabric, turning compliance into an operating rhythm
instead of a paperwork afterthought.

\subsection*{Risk Model}
Duty-of-care begins with the shared risk model that orients the deployment team. Governance historians compile AEIP receipts from
Layers 0 through 5 to restate the system’s civic mandate, lawful bases, and accumulated technical debts. Risk analysts translate
those records into scenario maps that span safety, security, privacy, and societal impact. The narrative casts this as a multi-day
preparation sprint where engineers, legal advisors, and community delegates challenge assumptions. Annex IV criteria anchor the
assessment, GDPR’s data protection principles delineate boundaries, and ENISA control catalogs provide countermeasure patterns.
The risk model is not a static document; it is a living conversation encoded into the AEIP ledger so that decisions can be re-read
by auditors, civic partners, and future maintainers.

Cross-jurisdictional deployments bring federated partners into the room. Municipal utilities, public health custodians, and
cross-border regulators contribute their localized threat intelligence. The narrative highlights how transparency never becomes
surveillance because every data flow is justified, minimised, and reversible. When a higher-risk scenario is identified—such as a
model’s recommendation influencing emergency services—the risk model expands to describe civic escalation paths, redress
channels, and the explicit thresholds that will trigger rollback. These thresholds become the compass for all subsequent activity.

\subsection*{Safety Testing}
Safety testing rehearses those thresholds before real citizens are affected. The story follows a layered validation pipeline: staged
environments replay historical incidents, synthetic adversaries attempt to exploit controls, and privacy probes verify GDPR
conformance with AEIP 1.3 validators that enforce privacy.* fields. Annex IV’s technical documentation requirements are rehearsed
within these exercises so that evidence is captured automatically. Engineers conduct canary deployments, dark launches, and
feature flags to isolate failure domains. Observers from oversight boards witness the tests, ensuring that safety is not just self-
attested but civically corroborated.

Safety narratives emphasise interpretability. Each test is accompanied by explainers that articulate why a safeguard works, how it
fails, and how it interacts with social impacts. The stack’s commitment to transparency-without-surveillance is dramatized by the
privacy engineering lead who refuses to log sensitive personal data even when debugging would be easier. Instead, the team relies
on anonymised metrics, consent-aware telemetry, and synthetic replicas. Testing outputs feed back into the risk model, tightening
definitions of acceptable behaviour and updating Annex IV crosswalk tables that bind security, privacy, and ethical obligations.

\subsection*{Incident Handling}
Despite preparation, incidents happen. The narrative introduces the Incident Hermeneutic Room—a virtual war room instrumented
with AEIP lifecycle hooks. When a deviation occurs, Layer 6 operators convene with legal observers, federated custodians, and
citizen liaisons. Incident data is ingested under strict privacy rules, hashed into AEIP receipts, and logged into the civic ledger.
The story describes a simulated service outage cascading from a cloud dependency. Operators consult Appendix B §2–§4 to follow
remediation scripts: triage, containment, eradication, and recovery. They document every hypothesis, decision, and communication,
knowing that post-incident transparency is owed to the public.

Corrective and Preventive Actions (CAPA) are dramatized as investigative arcs. Evidence chains link telemetry, interviews,
regression tests, and governance approvals. Each action is backed by verifiable data so that oversight committees can replay the
incident without speculation. The narrative stresses that AEIP receipts are not bureaucratic overhead—they are the civic memory
that converts mishaps into institutional learning. Incident reviews close with a public-oriented summary, referencing Appendix I’s
security controls and the GDPR accountability principle to explain both root causes and safeguards going forward.

\subsection*{Rollback and Recovery}
Rollback is presented as a disciplined craft, not an emergency panic. Deployment stewards maintain pre-authorised recovery plans
that enumerate triggers, decision owners, and communication trees. When a change crosses a predefined risk threshold, the
release train pauses and a rollback rehearsal is invoked. Stories portray dual-approval moments where technical stewards and
civic custodians jointly sign AEIP rollback bundles before production traffic is diverted. Restoration is executed through
immutable deployment artefacts, verifiable infrastructure-as-code, and tamper-evident logs that align with Appendix I §1–§4.

Recovery includes consent refreshes and public notice. If a system regression affects personal data processing, GDPR mandates
that citizens be informed. The narrative recounts how the communications team publishes layered notices: immediate alerts for
impacted individuals, a governance bulletin for oversight partners, and an archival entry in the public record. After the rollback,
engineers conduct comparative analyses between pre-incident and post-restoration telemetry to confirm safety baselines. The risk
model is then updated, and the lessons propagate to Layer 7 publication and Layer 8 civic interfaces.

\subsection*{Evidence Export}
Deployment concludes with evidence export, turning operational diligence into verifiable civic accountability. AEIP export bundles
collect test results, incident logs, recovery attestations, and crosswalk mappings to Annex IV and Appendix I. These bundles are
signed, timestamped, and prepared for external audit in line with AEIP §O.3–§O.7. The narrative follows the documentation team
as they curate dual-register summaries: technical appendices for expert reviewers and accessible narratives for civic audiences.
They coordinate with Layer 7 custodians to publish indices and hashes that prove authenticity without revealing sensitive data.

Evidence export also powers international interoperability. Federated partners request proofs to reconcile their own compliance
obligations. Through shared schemas and privacy-preserving exchange protocols, Layer 6 turns deployment into a shared civic
ritual rather than a closed-door operation. The chapter closes with the image of a deployment steward handing the evidence
bundle to a citizen oversight delegate, reinforcing the canonical principle that transparency must never become surveillance.
}

\subsection*{Triple Register}
\textbf{Narrative Intent:} Layer 6 confronts the operational fear that models will be deployed without context-aware safeguards, linking rollout decisions to observable accountability checkpoints.
\textbf{Normative Clauses:}
\begin{itemize}
\item Release managers \shall{} register launch justifications within \texttt{schemas/decision-rationale-record.jsonld} before systems go live.
\item Operational leads \shall{} associate each integration with \texttt{schemas/aeip/incident-report-schema.json} triggers to pre-plan remediation pathways.
\item Auditors \should{} reference \texttt{schemas/oam-schema.yaml} when documenting deployment oversight findings for the public record.
\end{itemize}
\textbf{Plain-Speak Summary:} This layer explains how deployments are approved and monitored. It captures the reasoning behind go-live decisions and the safety nets around them. Teams learn which records to check before shipping updates. Residents can see what will happen if things fail.

\normative{
Layer 6 deployments \shall{} define explicit risk thresholds, safety test matrices, rollback triggers, and recovery procedures in
accordance with Appendix B §2–§4. Each threshold \shall{} reference the corresponding Annex IV requirement, GDPR principle, and
Appendix I security control that justifies the boundary. Deployment teams \shall{} document these obligations within AEIP lifecycle
records prior to production promotion.

Pre-production safety testing \shall{} execute scripted scenarios covering functional safety, adversarial security, privacy
conformance, and societal impact. Results \shall{} be captured as AEIP receipts with privacy.* validators demonstrating GDPR
compliance. No deployment \shall{} proceed without dual sign-off from technical stewards and governance custodians that the defined
risk thresholds remain within tolerance.

Post-deployment monitoring \shall{} continuously log incidents, anomalies, and citizen-reported issues into the AEIP ledger.
Corrective and Preventive Action workflows MUST be evidence-backed, linking telemetry, interviews, and remediation tasks to
verifiable receipts. Change windows MUST be declared in advance; any hotfix outside a scheduled window REQUIRES a signed
justification, AEIP just-in-time notification, and an after-action bundle referencing Appendix B §3 and AEIP §O.5.

Rollback and recovery procedures \shall{} maintain versioned artefacts, cryptographically signed release manifests, and traceable
communication plans. Executing a rollback MUST trigger civic notifications proportional to impact, including GDPR-mandated data
processing disclosures when personal data is implicated. Recovery validation \shall{} include regression testing against the
original safety baselines and \shall{} document updates to crosswalk tables in Appendix I §1–§4.

Evidence export \shall{} generate sealed bundles containing test results, incident analyses, change approvals, and crosswalk maps.
These bundles MUST be made available for Layer 7 publication pipelines with indices or hashes suitable for public disclosure.
Federated partners \shall{} receive synchronized exports when joint services are affected. All exports MUST comply with AEIP §O.3–§O.7,
ensuring lifecycle integrity, verifiable signatures, and readiness for independent audit without exposing unnecessary personal
data.
}
\n\subsection*{Verification and Enforcement}
Conformance is evidenced through artefacts \texttt{schemas/decision-rationale-record.jsonld}, \texttt{schemas/aeip/incident-report-schema.json}, and \texttt{schemas/oam-schema.yaml} and corresponding AEIP audit records.
% SPDX-License-Identifier: CC-BY-SA-4.0

% \begin{autogenerated}
\begin{autogenerated}
\begin{longtable}{p{0.22\textwidth}p{0.73\textwidth}}
\toprule
\textbf{Purpose} & Control deployment approvals, rollback readiness, and integration guardrails across operational environments. \ 
\midrule
\textbf{Obligations} & \begin{itemize}
\item Release managers \shall{attach decision rationale records} to every go-live authorisation.
\item Integration leads \shall{pre-register incident pathways} mapped to remediation teams.
\item Auditors \should{record oversight verdicts and risk acceptances} tied to the Governance Integration Index.
\end{itemize} \ 
\midrule
\textbf{Verification Artefact} & Deployment dossier with rollout decision memo, integration checklist, and pre-authorised fallbacks. \ 
\midrule
\textbf{AEIP Schema} & \texttt{schemas/decision-rationale-record.jsonld} \ 
\midrule
\textbf{Evidence Fields} & \begin{itemize}
\item \texttt{decision.summary}
\item \texttt{risk_acceptance.conditions[]}
\item \texttt{integration_checks[].owner}
\end{itemize} \ 
\bottomrule
\end{longtable}
\textbf{Verification Example}\label{tab:layer6-verification}\par
\begin{lstlisting}[language=json,caption={Verification artefact for layer6},label={lst:layer6-example}]
{\n  "decision": {\n    "summary": "Launch housing-advisor v2 in pilot wards",\n    "approver": "custodian_council"\n  },\n  "risk_acceptance": {\n    "conditions": ["complete Appendix-B drill", "publish Layer-7 notice"]\n  }\n}
\end{lstlisting}
\end{autogenerated}
% \end{autogenerated}

\clearpage