% © 2025 Daniel P. Madden — Custodial Author
% AI OSI Stack v5.0-open-core (Civic Standard Edition)

% © 2025 Daniel P. Madden. Custodial Edition – AI OSI Stack v5.0-open-core.
% Unauthorized reproductions or derivatives are not recognized custodial works.
% Refer to CANONICAL_PROVENANCE.yaml for official verification.
% SPDX-License-Identifier: CC-BY-SA-4.0

% Implements: schemas/interpretive-trace-package.jsonld
% AI OSI Stack v5 — Canonical Edition
% Daniel P. Madden — Independent AI Researcher
% Created: 2025-11-09
% File: 18_Interpretive_and_Temporal_Continuity.tex
% Purpose: Canonical chapter content for interpretive and temporal continuity
\narrative{
Chapter 18 explores how the Stack preserves meaning across time. Interpretive continuity ensures that future custodians, citizens, and researchers can understand why decisions were made and how obligations evolved. Temporal continuity guarantees that commitments persist through leadership changes, technological upgrades, and civic transitions. The narrative centres on the Continuity Stewardship Circle, a group of archivists, legal historians, youth representatives, and technologists entrusted with safeguarding institutional memory.

The chapter opens inside the Continuity Archive, a hybrid space where physical artifacts and digital ledgers coexist. The Stewardship Circle reviews hermeneutic dossiers from Appendix L, aligning them with AEIP manifests and Appendix C change logs. They uncover an early debate about data minimisation strategies. By replaying interpretive sessions, they trace how the Stack negotiated tensions between analytic utility and privacy protection. The narrative shows how continuity work transforms past disagreements into guidance for future design, preventing the rediscovery of old mistakes.

Temporal continuity is tested when a custodial succession occurs. A retiring custodian transfers responsibilities to a new cohort, activating Appendix H protocols. The story details the succession ceremony: keys are rotated, interpretive briefings are delivered, and civic witnesses sign attestation forms referenced in Appendix N. The Circle ensures that the incoming custodians comprehend not only procedures but the values encoded in the canon. They provide annotated timelines that map milestones, crises, and community interventions, demonstrating how commitments were reaffirmed during stress.

The chapter then follows a time capsule project. Youth delegates collaborate with elders to document visions for the Stack’s next decade. They use narrative recordings, normative commitments, and speculative scenarios to articulate aspirations and warnings. These materials are sealed with provenance signatures (Appendix O) and added to the continuity archive. The time capsule embodies intergenerational stewardship: the Stack acknowledges that governance is a relay that must remain intelligible to those who inherit it.

Finally, the Circle hosts a Continuity Forum where participants rehearse disaster recovery of meaning. They simulate a situation where key records are lost or misinterpreted. Using redundant archives, community oral histories, and cross-jurisdictional partnerships, they reconstruct the canon’s intent. The exercise affirms that continuity requires plurality—multiple perspectives, redundant storage, and shared ownership. The chapter ends with the Circle publishing an annual continuity letter summarizing interpretive themes, emerging risks, and commitments for the coming year.
}

\subsection*{Triple Register}
\textbf{Narrative Intent:} This chapter solves the risk of obligations drifting over time by enforcing interpretive continuity, showing how each era’s context is preserved for future stewards.
\textbf{Normative Clauses:}
\begin{itemize}
\item Custodians \shall{} maintain continuity dossiers within \texttt{schemas/interpretive-trace-package.jsonld} so future readers inherit full context.
\item Semantic stewards \shall{} track evolving terminology through \texttt{schemas/svc/semantic-registry.jsonld} to flag meaning shifts.
\item Governance historians \should{} document temporal reconciliations in \texttt{schemas/integrity-ledger-entry.jsonld} to evidence why clauses persist or change.
\end{itemize}
\textbf{Plain-Speak Summary:} This chapter keeps the stack understandable across decades. It stores context packages, vocabulary updates, and ledger entries together. Future teams can see why decisions were made. That transparency prevents quiet rewrites of civic commitments.

\normative{
The Stack \shall{} maintain a Continuity Stewardship Circle responsible for interpretive and temporal continuity. The Circle MUST curate hermeneutic dossiers, change logs, persona archives, and AEIP manifests to preserve institutional memory. Continuity reviews \shall{} occur at least biannually, producing public summaries aligned with transparency tiers in Appendix K.

Custodial transitions \shall{} follow Appendix H succession protocols, including key rotation, interpretive briefings, civic witnessing, and attestation through Appendix N public channels. Incoming custodians MUST acknowledge receipt of continuity archives and commit to upholding the canon’s non-revocable principles. Any gaps or missing records \shall{} trigger remediation plans documented in AEIP manifests.

Continuity archives \shall{} maintain redundant storage across jurisdictions, incorporating physical, digital, and community-hosted repositories. Interpretive materials MUST be indexed with provenance signatures per Appendix O and linked to relevant appendices and chapters. Annual continuity letters \shall{} summarise interpretive insights, risk outlooks, and planned updates, inviting civic comment and international collaboration to sustain collective memory.
}
\n\subsection*{Verification and Enforcement}
Conformance is evidenced through artefacts \texttt{schemas/interpretive-trace-package.jsonld}, \texttt{schemas/svc/semantic-registry.jsonld}, and \texttt{schemas/integrity-ledger-entry.jsonld} and corresponding AEIP audit records.