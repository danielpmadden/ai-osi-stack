% SPDX-License-Identifier: CC-BY-NC-ND-4.0
\chapter{Layer 24 — Governance Paradox}
\section{Narrative Intent}
Layer 24 addresses the tension between adaptive governance and the need for stable commitments. The
narrative recognizes that continual iteration can destabilize trust if changes outpace civic understanding,
yet rigidity can inhibit necessary reforms. The layer offers principles for balancing these pressures.

\section{Normative Clauses}
\begin{enumerate}
  \item \textbf{Change Justification.} Material governance changes SHALL include a documented paradox analysis
        explaining why adaptation is necessary and how stability is preserved.
  \item \textbf{Dual Track Communication.} Custodians SHALL communicate both the enduring commitments and the
        experimental elements of governance, highlighting guardrails protecting civic interests.
  \item \textbf{Sunset Controls.} Temporary measures SHALL include sunset criteria, monitoring requirements, and
        decision points for renewal or retirement.
  \item \textbf{Public Reflection.} Annual reflections SHALL evaluate how well the balance was maintained,
        inviting civic feedback on whether adjustments felt predictable and fair.
\end{enumerate}

\section{Plain-Speak Summary}
Governance needs to change without becoming chaotic. This layer requires teams to explain why updates
are needed, remind the public which promises remain steady, set expiration dates on experiments, and
ask communities whether the pace feels right.

\cleardoublepage
