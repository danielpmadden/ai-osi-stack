% AI OSI Stack v5 — Canonical Edition
% Daniel P. Madden — Independent AI Researcher
% Created: 2025-11-04
% File: 17_Meta_Audit_and_Self_Accountability.tex
% Purpose: Canonical chapter content for meta-audit
\narrative{
Meta-audit is narrated as a reflexive discipline in which the stack interrogates its own interpretive frames and redistributes authority when blind spots surface. Custodians convene hermeneutic colloquia that pair auditors, community historians, and technical stewards to re-read prior findings, asking how positionality, resource asymmetry, or emergent harms may have skewed earlier judgments. Annual cycles culminate in public hearings where ledger excerpts, dissenting annotations, and corrective actions are rehearsed as civic theatre so citizens can contest or affirm the governance narrative. (Appendix L §1–§4)

The narrative follows an exemplar audit in which translation biases were uncovered within multilingual grievance queues. Investigators replayed decision traces, cross-checked contextual metadata, and invited community translators to co-author remediation, demonstrating that accountability is strongest when those affected drive the interpretive corrections. These rituals reinforce a culture of institutional humility and ensure that self-accountability is not merely procedural but relational, anchoring trust in the shared stewardship of meaning. (Appendix N §1)
}
\normative{
Meta-audit programmes SHALL convene at least annually, with supplemental reviews triggered when harm indicators exceed risk-tier thresholds or when civic petitions demand redress. Each meta-audit MUST log its scope, methodologies, and interpretive hypotheses in the Hermeneutic Ledger, tagging entries with AEIP fields for audit.scope, audit.findings, and audit.remediation. Reports SHALL include evidentiary crosswalks to underlying manifests and MUST publish both majority conclusions and minority reservations. (AEIP §O.7)

Custodians SHALL guarantee independence by rotating lead reviewers, publishing conflict-of-interest declarations, and seating community observers with veto authority over closure declarations. Follow-up actions MUST specify measurable outcomes, timelines, and responsible stewards; completion SHALL be confirmed through ledger attestations and community witness signatures. Failure to complete corrective paths SHALL escalate to custodial oversight councils, where sanctions or structural reforms are imposed and recorded for public scrutiny. (Appendix L §1–§4),(Appendix N §1)
}
