% SPDX-License-Identifier: CC-BY-NC-ND-4.0
\chapter{Layer 3 — Model Lifecycle}
\section{Narrative Intent}
Layer 3 complements Layer 03 by ensuring that post-development activities—deployment, monitoring,
retirement, and archival—are governed with the same rigor. It emphasizes continuity between
laboratory validation and real-world operation, drawing on the Version 4 lifecycle expectations while
integrating resilience and escalation pathways introduced across Layers 06 and 12.

\section{Normative Clauses}
\begin{enumerate}
  \item \textbf{Deployment Authorization.} No model SHALL enter production without documented confirmation
        that its ethical, civic, and technical controls remain valid for the intended environment.
  \item \textbf{Runtime Monitoring.} Custodians SHALL maintain monitoring dashboards that track performance,
        safety indicators, and civic impact metrics, escalating anomalies via Appendix B procedures.
  \item \textbf{Lifecycle Reviews.} Scheduled reviews SHALL evaluate whether models still fulfill their civic
        purpose, with decisions captured in Appendix C change logs and Appendix D audit artifacts.
  \item \textbf{Retirement and Archival.} When models are decommissioned, stewards SHALL preserve sufficient
        artifacts to reproduce outcomes while revoking access to live data paths.
\end{enumerate}

\section{Plain-Speak Summary}
Once a model ships, the work is not over. This layer requires teams to keep watching how it behaves,
log what they find, and retire it responsibly when it no longer meets expectations. The public should be
able to trace every major decision from launch to sunset.

\cleardoublepage

% Authored and maintained solely by the Custodial Editorial Committee.
% This is a non-operational, publication-grade governance artifact.
% No AEIP runtime specs or machine-readable schemas are included.
