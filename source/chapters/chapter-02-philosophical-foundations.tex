% Implements: schemas/aeip/aeip-frame-schema.json
% AI OSI Stack v5 — Canonical Deepbuild
% Chapter 02 — Philosophical Foundations
% Sources: Epistemology by Design v1, AEIP v1, Persona Architecture v2, Update Plans 2, 5, 8

\narrative{
Philosophical foundations situate the stack within a tradition of civic republicanism, epistemic humility, and human-rights
constitutionalism. The stack is not merely an engineering blueprint; it is a normative project that treats intelligence systems
as civic institutions. Drawing from Epistemology by Design v1, the chapter argues that governance legitimacy arises when knowledge
claims are contestable, transparent, and accountable to the people affected. Persona Architecture v2 contributes insights about
self-determination and representational fidelity, ensuring that automated agents do not eclipse the communities they are meant to
serve.
}

\section*{Civic Republican Ethos}
\narrative{
The stack embraces civic republican principles: freedom as non-domination, shared responsibility for the common good, and
deliberative oversight. The narrative recounts dialogues captured in Update Plans 2 and 8 where community members insisted that
AI systems must remain subordinate to civic control. The stack therefore frames every technological capability as a delegated duty,
not an autonomous force. Transparency is used to empower collective agency, not as a tool of coercion. This ethos informs the
interpretive principle that transparency must never become surveillance.
}
\normative{
Implementers SHALL treat the stack as a civic mandate. Deployment decisions MUST include deliberative checkpoints with community
representation documented in (AEIP §O.3) and (AEIP §O.6). Any design that risks creating domination—through asymmetrical access,
opaque reasoning, or coercive nudging—SHALL be reworked or halted. Oversight councils SHALL evaluate whether proposed uses honour
freedom as non-domination and SHALL publish their findings for public review.
}

\section*{Epistemic Integrity}
\narrative{
Epistemology by Design v1 teaches that trustworthy systems expose how they know what they claim to know. The stack embeds this
philosophy through AEIP receipts, hermeneutic logging, and reasoning exchange safeguards. The narrative explores how knowledge
claims are validated: data provenance is documented, model decisions are interrogable, and instruction sets are accountable to
interpretive principles. Epistemic integrity requires humility; systems must be able to admit uncertainty, facilitate correction,
and welcome critique.
}
\normative{
All knowledge claims derived from stack-aligned systems SHALL include verifiable evidence trails. AEIP records MUST capture data
sources, transformation logic, evaluation metrics, and human oversight notes. Implementers SHALL provide civic auditors with
mechanisms to challenge or invalidate claims, triggering Appendix B remediation when errors surface. Persona-mediated interactions
SHALL disclose when reasoning relies on probabilistic inference or contested knowledge so that communities can respond accordingly.
}

\section*{Human Dignity and Non-Instrumentalisation}
\narrative{
The stack’s philosophical stance rejects the instrumentalisation of people. Citizens are co-authors, not data sources. The
narrative references dialogues from Update Plans 5 and 8 where community advocates emphasised that governance must protect dignity
by design. Persona Architecture v2 addresses how digital representatives must honour community instructions, while Appendix E
protects privacy, agency, and cultural context. Human dignity extends to custodial accountability: custodians are bound to respond
with care, not bureaucratic opacity.
}
\normative{
Implementers SHALL design every layer to respect human dignity. Data collection MUST observe Appendix E §3 safeguards and SHALL be
justified in (AEIP §O.4) rationale logs. Automated decisions impacting rights SHALL provide meaningful appeal pathways in line with
Appendix G. Persona-driven experiences SHALL offer opt-outs and MUST reflect community-authored behavioural constraints. Any
instrumentalisation of persons for efficiency gains SHALL be considered a violation triggering immediate remediation.
}

\section*{Pluralism and Equity}
\narrative{
Pluralistic societies demand governance that accommodates diverse epistemologies and cultural norms. The stack incorporates equity
by treating marginalized voices as authoritative partners. Narrative segments recount engagements with indigenous data stewards,
disability advocates, and linguistic minorities who shaped the obligations around accessibility, interpretability, and community
control. Pluralism is not a garnish; it is integral to the stack’s legitimacy.
}
\normative{
Custodians SHALL facilitate participation from historically excluded communities. Consultation processes MUST include accessible
formats, translation support, and remuneration where appropriate. AEIP registries SHALL track demographic representation in
consultations and SHALL flag gaps for remediation. Implementers SHALL evaluate the equity impacts of each deployment and MUST
publish mitigation plans aligned with Appendix F fairness protocols.
}

\section*{Ethics of Care and Accountability}
\narrative{
The stack intertwines ethics of care with accountability. Governance is not solely about rules; it is about relational
responsibility. Care manifests in how custodians respond to harm, how auditors engage with affected communities, and how systems are
maintained over time. Accountability ensures that care is not paternalistic but grounded in mutual respect and enforceable
obligations. The narrative highlights community testimonies demanding responsive remediation and ongoing dialogue.
}
\normative{
Care obligations SHALL be codified as operational requirements. Incident response plans MUST prioritise affected persons, offering
clear communication, support, and restitution consistent with Appendix B. Custodians SHALL maintain channels for continuous
feedback, and oversight bodies SHALL monitor whether remediation addresses structural causes. Accountability audits SHALL evaluate
not only compliance but the quality of care delivered during remediation.
}

\clearpage
