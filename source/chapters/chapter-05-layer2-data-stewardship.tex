% Implements: schemas/aeip/gds-schema.json
% AI OSI Stack v5 — Canonical Deepbuild
% Chapter 05 — Layer 2: Data Stewardship
% Sources: Version 5 Draft, Appendix E, Update Plans 3, 5, 10, AEIP v1

\narrative{
Layer 2 anchors trustworthy AI on rigorous data stewardship. It governs how data is collected, classified, protected, and shared.
The narrative underscores that data is not raw material but a reflection of people and communities. Stewardship is therefore a
moral and legal duty. Drawing from Appendix E and Update Plans 3, 5, and 10, the chapter explains how rights-respecting data
practices support the stack’s higher layers and ensure that transparency never becomes surveillance.
}

\subsection*{Triple Register}
\textbf{Narrative Intent:} Layer 2 recognises that communities fear losing control of their data, so it articulates how stewardship, consent, and remedy pathways are enforced throughout the lifecycle.
\textbf{Normative Clauses:}
\begin{itemize}
\item Data custodians SHALL catalogue collection and usage rules in \texttt{schemas/drr-schema.yaml} prior to ingesting records.
\item Incident commanders SHALL document breaches of stewardship duties through \texttt{schemas/aeip/incident-report-schema.json} within mandated timeframes.
\item Programme owners SHOULD update trust briefings with citations to \texttt{schemas/aeip-template.yaml} so residents know where stewardship evidence lives.
\end{itemize}
\textbf{Plain-Speak Summary:} This layer explains how the stack protects the data it relies on. It points to the forms that describe lawful use, incidents, and responses. Readers can trace who is responsible for each dataset. Those records make it clear how to challenge misuse.


\section*{Data Inventory and Classification}
\narrative{
Effective stewardship begins with comprehensive inventories. Teams map data sources, types, sensitivity levels, and provenance.
Community consultations inform which datasets require heightened safeguards. The narrative walks through the creation of custodial
catalogues and linkages to AEIP receipts, ensuring every dataset can be traced back to its authorisation under Layer 0 mandates and
Layer 1 charters.
}
\normative{
Custodians SHALL maintain living data inventories that record origin, lawful basis, sensitivity classification, retention policies,
and mandated safeguards. Inventories MUST be version-controlled, with changes logged in (AEIP §O.5). Sensitive data categories, as
defined in Appendix E §3, SHALL trigger enhanced protections, including minimisation, encryption, and differential privacy where
appropriate. No dataset SHALL enter operational pipelines without documented authorization.
}

\section*{Collection and Consent}
\narrative{
Collection practices must respect community agency. The narrative describes how consent frameworks, community agreements, and legal
bases are negotiated. Update Plan 3 introduced participatory consent models allowing community representatives to co-govern shared
data. AEIP logs capture consent artefacts, ensuring they can be audited and revoked when necessary.
}
\normative{
Data collection SHALL adhere to explicit legal or community-authorised bases. Consent processes MUST be understandable, revocable,
and recorded in (AEIP §O.3). Collective data agreements SHALL reflect community governance structures and SHALL include dispute
resolution mechanisms. Implementers SHALL immediately honour withdrawal requests and MUST update downstream systems to reflect the
change, documenting actions in Appendix B remediation logs if necessary.
}

\section*{Protection and Access Control}
\narrative{
Protection mechanisms guard against misuse. The narrative covers encryption, access segmentation, secure enclaves, and accountability
audits. Communities insisted during Update Plan 5 consultations that access controls must be transparent, allowing citizens to know
who interacts with their data and why. AEIP integration ensures that every access event leaves an auditable trace.
}
\normative{
Data at rest and in transit SHALL be protected using state-of-the-art cryptographic controls commensurate with sensitivity.
Access SHALL follow least-privilege principles, with roles defined in Appendix I security matrices. Every access event MUST be
logged to (AEIP §O.4) with user identity, purpose, and legal basis. Periodic access reviews SHALL be conducted quarterly, and
unauthorised access SHALL trigger immediate incident response under Appendix B.
}

\section*{Quality, Integrity, and Bias Mitigation}
\narrative{
Stewardship includes ensuring data quality and mitigating bias. Narrative sections discuss validation pipelines, provenance checks,
and community review boards that evaluate representational balance. Update Plan 10 emphasised the need for reflexive audits where
communities inspect datasets for harmful proxies or omissions.
}
\normative{
Implementers SHALL establish quality assurance processes covering data accuracy, completeness, and timeliness. Bias assessments
MUST be conducted prior to model training and SHALL involve community reviewers where feasible. Findings SHALL be recorded in (AEIP
§O.5) and SHALL feed into mitigation plans referencing Appendix F fairness guidelines. Datasets failing quality or equity checks
SHALL NOT be deployed until remediated.
}

\section*{Retention and Deletion}
\narrative{
Responsible stewardship includes clear retention and deletion practices. The narrative outlines how custodians set retention
schedules aligned with legal obligations and community expectations. Deletion ceremonies, documented in the historical corpus,
demonstrate accountability when data is no longer necessary.
}
\normative{
Retention schedules SHALL be published and SHALL align with legal requirements and community agreements. Data MUST be deleted or
irreversibly anonymised once retention periods expire unless a renewed mandate authorises continued use. Deletion events SHALL be
logged in (AEIP §O.6) with verification evidence. Custodians SHALL provide public summaries of deletion activities, ensuring that
records of destruction do not expose personal data while confirming compliance.
}

\section*{Sharing and Interoperability}
\narrative{
When data must be shared across agencies or jurisdictions, stewardship principles travel with it. The narrative explains how data
sharing agreements embed charter obligations, privacy safeguards, and audit hooks. Interoperability is designed to support civic
collaboration, not to enable surveillance networks.
}
\normative{
Data sharing SHALL occur only under agreements that restate relevant mandates, charters, and Appendix E safeguards. Receiving
parties MUST commit to equivalent or stronger protections and SHALL log their compliance in AEIP-compatible registries. Cross-
jurisdictional transfers SHALL undergo risk assessments, and results SHALL be shared with affected communities. Interoperability
interfaces SHOULD implement privacy-preserving technologies to minimise exposure while enabling accountability.
}

\n\subsection*{Verification and Enforcement}
Conformance is evidenced through artefacts \texttt{schemas/drr-schema.yaml}, \texttt{schemas/aeip/incident-report-schema.json}, and \texttt{schemas/aeip-template.yaml} and corresponding AEIP audit records.
\clearpage
