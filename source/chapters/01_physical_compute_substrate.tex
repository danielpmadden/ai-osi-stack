% SPDX-License-Identifier: CC-BY-NC-ND-4.0
\chapter{Layer 01 — Physical Compute Substrate}
\section{Narrative Intent}
\subsection*{Overview}
Layer 01 translates the civic mandate into tangible obligations for the physical infrastructure that
hosts AI workloads. It expands the Version 4 hardware resilience requirements with supply-chain
provenance metadata and lifecycle maintenance checkpoints. The narrative goal is to guarantee that
all compute assets remain trustworthy, resilient, and ready for audit at any time.

\section{Normative Clauses}
\subsection*{Infrastructure Controls}
\begin{enumerate}
  \item \textbf{Provenance Tracking.} Operators SHALL maintain verifiable provenance metadata for every
        compute asset, including component origin, custody chain, and firmware attestation status.
  \item \textbf{Resilience Benchmarks.} Facilities SHALL implement redundancy and failover criteria sized to
        civic service levels, ensuring continuity for mandated public interfaces.
  \item \textbf{Lifecycle Maintenance.} Custodians SHALL document inspection intervals, remediation actions,
        and retirement procedures, linking each to accountable personnel and civic risk assessments.
  \item \textbf{Tamper Evidence.} All hardware hosting AEIP workloads SHALL use tamper-evident seals or
        equivalent controls, with incident response triggers when integrity is questioned.
\end{enumerate}

\section{Plain-Speak Summary}
This layer makes sure the machines running the system are sourced responsibly, hardened against
failure, and checked on a disciplined schedule. If something goes wrong or a component is swapped,
there must be records that show what happened, who approved it, and how the public’s risk was kept
within agreed bounds.

\emph{License: Creative Commons Attribution–NonCommercial–NoDerivatives 4.0 International.}

\cleardoublepage
