% AI OSI Stack v5 — Canonical Edition
% Daniel P. Madden — Independent AI Researcher
% Created: 2025-11-04
% File: 11_Layer8_Civic_Participation.tex
% Purpose: Canonical chapter content for Layer 8
\narrative{
Layer 8 is the civic interface where the public steps into co-governance. The narrative follows residents, advocates, journalists,
and federated partners as they engage with systems that are intentionally designed for participation rather than passive
observation. Interfaces are accessible, explanations are contextual, and grievance channels are embedded from the first design
sketch. Transparency remains bound by consent and proportionality, honouring the Right-to-Opacity articulated in Appendix E §1–§6:
the public may inspect the system without forfeiting personal privacy. Civic participation is thus depicted as a continuous loop of
feedback, deliberation, and renewal.

\subsection*{Civic Access}
The chapter opens with a day at the Civic Oversight Interface. Citizens log into a public portal that offers layered explanations of
what the system does, why it exists, and how it has behaved. Accessibility features—screen-reader support, multilingual
translations, and mobile-friendly layouts—are treated as core functionality, not optional enhancements. Narrative vignettes show
a small business owner reviewing how automated permits were adjudicated, a researcher downloading anonymised metrics through an
open API, and a community liaison requesting a briefing in plain language.

Every interaction is anchored by AEIP evidence references so that users can traverse from summaries to detailed artifacts. Privacy
assurances are visible: users see how their data is treated, what consent scopes apply, and where opt-outs can be exercised. The
interface explains the limits of transparency, clarifying that personal data is redacted unless a lawful and justified exception is
met. Layer 8 therefore manifests the canonical principle—transparency must never become surveillance—by making oversight possible
without exposing individuals.

\subsection*{Appeals and Redress}
Participation includes contestation. The narrative introduces the Civic Feedback Desk, where grievances, appeals, and suggestions
are submitted through accessible forms, hotlines, and in-person clinics. Each submission triggers an AEIP Civic Feedback receipt
with timestamps, custodial routing, and expected resolution windows. Case handlers triage issues based on severity, referencing
Appendix B remediation protocols and Appendix E human-rights safeguards. Stories depict a resident appealing an eligibility
decision, a civil liberties group flagging potential discrimination, and a developer proposing usability improvements.

Appeals are not mere customer service tickets; they are governance events. Each case spawns a deliberation thread visible to the
submitter and, where appropriate, to the wider public. Decisions are explained, evidence is referenced, and when mistakes are
confirmed, rectifications are executed with accountability. The narrative emphasises that redress is restorative: apologies,
compensation, policy updates, and systemic fixes are tracked through AEIP workflows so that nothing slips into obscurity.

\subsection*{Attestations}
Trust in civic participation relies on independent attestations. The chapter narrates how at least one external oversight actor—a
registered NGO, ombudsperson, or academic consortium—provides cryptographic attestations confirming that oversight interactions
have been honoured. These attestations are aligned with Appendix E §6 and Update Plan 10, binding oversight actors to their own
codes of conduct. The storyline follows an annual public audit where the oversight actor signs AEIP export bundles verifying that
appeals were processed, privacy limits respected, and federated obligations maintained.

Attestations are also social rituals. Public ceremonies allow oversight actors to present findings, answer questions, and reaffirm
commitments. Citizens witness signatures being applied in real time, and they can verify the attestations through public ledgers.
This shared witnessing strengthens legitimacy while respecting Right-to-Opacity: sensitive case details remain shielded, yet the
fact of oversight is undeniable.

\subsection*{Transparency Tiers}
Layer 8 navigates the tension between openness and privacy through transparency tiers. The narrative describes three tiers:
public summaries, restricted partner dashboards, and controlled research enclaves. Public summaries provide narratives, metrics,
and policy updates. Restricted dashboards are available to accredited civic organisations that agree to custodial obligations and
privacy handling standards. Research enclaves host more granular datasets under strict access agreements, ensuring proportionality
and compliance with privacy.* validators.

The storyline tracks how a journalist escalates from public tier access to a restricted dashboard by demonstrating legitimate
purpose and agreeing to reciprocity terms (Appendix N §2). Each tier upgrade generates AEIP receipts and, when personal data is
implicated, requires consent checks or lawful basis confirmation. Transparency is thus tiered, not binary, enabling meaningful
scrutiny without unbounded exposure.

\subsection*{Federation Interface}
Civic participation extends beyond a single institution. Federated partners—municipal agencies, civic cooperatives, and regional
alliances—coordinate through the Federation Interface. The narrative depicts a collaborative session where partners synchronise
policy updates, share attestations, and reconcile service metrics. Appendix G §1–§4 establishes custodial criteria, while Appendix
H §2 prevents fragmentation by requiring interoperable governance schemas. Partners exchange AEIP packages containing deployment
data, civic feedback summaries, and oversight attestations.

The Federation Interface also mediates conflict. When jurisdictions disagree on risk thresholds or privacy interpretations, a
structured dialogue protocol is invoked. Mediators review AEIP evidence, consult Appendix E safeguards, and negotiate consistent
commitments. The narrative closes with a federated pledge: partners sign a mutual assurance charter reaffirming transparency-with-
consent, collective responsibility, and the shared duty to keep civic interfaces accessible to all residents.
}
\normative{
Implementations of Layer 8 SHALL provide accessible civic portals, APIs, and documentation that allow citizens to inspect
operations, review evidence summaries, and understand rights. Interfaces SHALL meet accessibility standards, support
multilingual communication, and present AEIP references for every published artifact. Personal data MUST remain redacted unless
a justified exception is documented with privacy.* validators and citizen consent where applicable.

Appeal and feedback channels SHALL be available through digital, telephonic, and in-person modalities. Every submission SHALL
produce an AEIP Civic Feedback receipt capturing timestamps, routing custodians, and expected resolution windows. Institutions
SHALL publish service-level targets for grievance handling and MUST escalate unresolved cases according to Appendix B
remediation tiers. Outcomes SHALL be communicated with evidence references and, when corrective actions occur, SHALL update Layer
6 deployment and Layer 7 publication records.

At least one external oversight actor SHALL provide cryptographic attestations of civic engagement processes each renewal cycle.
Attestations MUST reference AEIP bundles covering appeals, privacy controls, and federated obligations, and SHALL be signed in
accordance with AEIP §O.5. Oversight actors SHALL publish their own accountability statements and MUST confirm adherence to human-
rights safeguards outlined in Appendix E §1–§6.

Transparency tiers SHALL be defined, published, and enforced. Tier upgrades MUST require documented justification, reciprocity
agreements, and AEIP receipts. Restricted access SHALL include custodial obligations, audit hooks, and proportional data minimisation. Any access involving personal data MUST pass privacy.* validator checks and SHALL respect Right-to-Opacity limitations.

Federated partners participating in Layer 8 interfaces SHALL meet custodial criteria from Appendix G §1–§3 and SHALL honour
non-fragmentation clauses from Appendix H §2. Shared governance sessions MUST exchange AEIP packages that summarise deployment
status, civic feedback trends, and oversight attestations. Disputes SHALL trigger mediation workflows referencing Appendix E
safeguards and MUST conclude with documented resolutions accessible through the civic portal.
}
