% SPDX-License-Identifier: CC-BY-NC-ND-4.0
\chapter{Layer 04 — Instruction Control}
\section{Narrative Intent}
\subsection*{Overview}
Layer 04 updates the instructional interface guidance from Version 4, introducing instruction
traceability and civic override hooks. Its purpose is to ensure that prompts, directives, and
configuration adjustments remain auditable and can be overridden when civic risk thresholds are met.

\section{Normative Clauses}
\subsection*{Control Obligations}
\begin{enumerate}
  \item \textbf{Instruction Logging.} All operator- and system-level instructions SHALL be logged with
        contextual metadata, including requester identity, civic justification, and resulting state changes.
  \item \textbf{Override Mechanisms.} Custodians SHALL implement civic override hooks that allow designated
        stewards to halt or redirect instructions when harm indicators are detected.
  \item \textbf{Traceability Assurance.} Instruction histories SHALL be linked to downstream outputs so auditors
        can reconstruct how responses were produced and whether deviations occurred.
  \item \textbf{Access Governance.} Role-based access controls SHALL govern who may issue instructions, with
        periodic reviews to confirm necessity and proportionality.
\end{enumerate}

\section{Plain-Speak Summary}
This layer makes sure every instruction the system receives is captured, justified, and reversible. If
an instruction looks risky, there are tools to stop it. People allowed to send commands must prove they
need that power, and auditors can follow the trail from instruction to outcome.

\emph{License: Creative Commons Attribution–NonCommercial–NoDerivatives 4.0 International.}

\cleardoublepage
