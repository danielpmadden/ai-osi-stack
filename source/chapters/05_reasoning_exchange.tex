% SPDX-License-Identifier: CC-BY-NC-ND-4.0
\chapter{Layer 05 — Reasoning Exchange}
\section{Narrative Intent}
\subsection*{Overview}
Layer 05 governs how reasoning sessions are conducted, recorded, and shared. The Version 5 update adds
attestation requirements and non-repudiation pathways so that exchanges between humans and systems
remain defensible in civic contexts. The narrative intent is to preserve epistemic integrity while enabling
collaborative problem solving.

\section{Normative Clauses}
\subsection*{Exchange Requirements}
\begin{enumerate}
  \item \textbf{Session Attestation.} Each reasoning session SHALL produce an attestation summarizing inputs,
        outputs, key decisions, and participants, signed by responsible custodians.
  \item \textbf{Evidence Linking.} Claims made during exchanges SHALL reference supporting evidence registered
        within the epistemic infrastructure protocol of Layer 09.
  \item \textbf{Non-repudiation.} Communication channels SHALL implement cryptographic or procedural controls
        that prevent parties from denying their participation or commitments.
  \item \textbf{Public Disclosure.} Sessions with civic impact SHALL produce a redacted public summary to foster
        transparency without exposing sensitive data.
\end{enumerate}

\section{Plain-Speak Summary}
Reasoning exchanges are treated like formal civic records. Everyone involved signs off on what was said,
links their claims to evidence, and cannot quietly walk back commitments. When the public is affected,
a clear summary must be shared.

\emph{License: Creative Commons Attribution–NonCommercial–NoDerivatives 4.0 International.}

\cleardoublepage
