% Implements: protocol/aeip-handshake.py
% AI OSI Stack v5 — Canonical Edition
% Daniel P. Madden — Independent AI Researcher
% Created: 2025-11-04
% File: 15_Governance_Transport_and_Maturity_Model.tex
% Purpose: Canonical chapter content for governance transport and maturity
\narrative{
Chapter 15 examines how governance moves through institutions and matures across jurisdictions. Governance transport is depicted as a civic relay: obligations originate in Layer 0 mandates, traverse operational layers, and arrive at frontline services without losing fidelity. The narrative follows a cohort of custodians tasked with deploying the Stack across three contexts—a city transit authority, a regional hospital network, and a cross-border research consortium. Each context exposes different maturity baselines, requiring transport mechanisms that adapt while preserving canonical commitments.

The chapter opens with the Transit Authority onboarding. Governance liaisons unpack the Stack’s maturity model, which defines stages from Initiation to Stewardship. The authority is at Stage 1, possessing ad-hoc policies and limited audit capacity. The story shows liaisons conducting a governance inventory, mapping existing procedures to Appendices B, E, and I. They introduce governance transport toolkits: AEIP policy adapters that translate obligations into local directives, custodial playbooks that schedule compliance sprints, and civic interface guides that prepare Layer 8 participation. The onboarding culminates in a governance covenant where transit leaders commit to incremental milestones, public reporting, and co-governance with rider councils.

The hospital network sits at Stage 3, with established regulatory processes but fragmented transparency practices. Here, governance transport resembles harmonisation. The narrative follows a clinical safety officer who integrates Stack obligations with healthcare privacy rules, referencing Appendix K transparency tiers to differentiate between clinical disclosures and public dashboards. Maturity assessments identify gaps in adversarial resilience and interpretive continuity, prompting the hospital to adopt Appendices M and L practices. Through simulated audits, the network rehearses how AEIP manifests can accompany medical device updates and algorithmic triage tools. The story highlights how maturity advances when organisations internalise obligations as routine habits rather than external audits.

The research consortium operates across borders and is classified Stage 4, experimenting with custodianship innovation. Governance transport becomes federated choreography. Partner institutions exchange maturity attestations, align on Appendix G policy agreements, and co-design oversight rotations. The narrative depicts a moment where one partner proposes a novel interpretive forum that blends indigenous knowledge systems with scientific review, expanding Appendix L’s ledger to accommodate plural epistemologies. The consortium’s maturity progression is measured not only by compliance but by its capacity to invite civic partners into governance experiments.

The chapter closes by returning to the maturity model, which is updated with metrics gathered from the three contexts. Governance transport is reaffirmed as an iterative practice that must remain legible to citizens. The final scene shows a public dashboard where maturity indicators, AEIP commitments, and custodial contact points are published in layered formats, enabling communities to track progress and hold institutions accountable.
}
\normative{
Organisations adopting the Stack SHALL complete an initial governance maturity assessment covering policy alignment, custodianship capacity, AEIP integration, transparency tiers, and adversarial preparedness. Assessments MUST document baseline stage classifications and cross-reference Appendices B, E, G, I, K, L, and M as applicable. Governance transport plans SHALL outline incremental milestones, responsible owners, evidence requirements, and civic engagement commitments for advancing maturity.

Governance obligations SHALL be translated into local directives or procedures using AEIP policy adapters that preserve provenance and modal intent. Any localisation MUST maintain references to Appendix O signatures and SHALL be validated through interpretive review sessions recorded in Appendix L. Institutions SHALL publish maturity indicators and progress reports through transparency tiers defined in Appendix K, with at least annual updates accessible to civic partners.

Federated deployments SHALL exchange maturity attestations signed by custodial quorums per Appendix H. Partners MUST establish coordination mechanisms for incident response, interpretive deliberation, and audit support, referencing Appendices B, G, and N. Advancement to higher maturity stages SHALL require demonstration of sustained compliance, participatory governance practices, and resilience exercises documented in AEIP manifests and public ledgers.
}
