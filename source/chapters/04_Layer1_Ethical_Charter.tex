% AI OSI Stack v5 — Canonical Edition
% Daniel P. Madden — Independent AI Researcher
% Created: 2025-11-04
% File: 04_Layer1_Ethical_Charter.tex
% Purpose: Chapter 04 rewritten for Layer 1 Ethical Charter and Intent Disclosure
\narrative{
Layer 1 translates the civic mandate into an ethical charter that explains why the system exists and how it will behave under pressure. The narrative begins with convenings where custodians, subject-matter experts, and community delegates draft intent statements that honour the Civic Charter while expanding upon specific moral commitments. Ethical deliberation is documented as a participatory process: facilitators surface equity impacts, cultural sensitivities, accessibility requirements, and environmental considerations. The resulting charter names intended beneficiaries, articulates harm prevention strategies, and embeds the principle that transparency must never become surveillance. These statements become living guides for design, procurement, and oversight decisions.

\subsection*{Contextual Alignment}
Layer 1 ensures continuity between civic expectations and technical planning. It harmonises the narrative of public good with regulatory duties such as GDPR accountability, Annex IV conformity, and Appendix E §3 human-rights protections. Intent disclosures clarify legal bases, data minimisation commitments, explainability goals, and fallback options when models fail gracefully. The narrative highlights how interdisciplinary review boards annotate charters with references to prior incidents, research evidence, and comparative jurisprudence, keeping the ethical storyline grounded in reality rather than aspirational prose.

\subsection*{Ethical Rationale in Practice}
The charter is operationalised through ethical design controls, risk hypotheses, and persona empathy maps sourced from (AEIP §O.4) records. These artefacts inform Layer 2’s privacy thresholds, Layer 3’s development guardrails, and Layer 4’s instruction boundaries. By publishing charter updates in Governance Disclosure Systems, custodians demonstrate accountability to the public and provide implementers with clear guardrails. The narrative underscores that ethical intent must be revisited whenever deployment contexts shift, data sources evolve, or civic feedback introduces new obligations.
}
\normative{
\begin{itemize}
  \item Ethical charters SHALL be co-authored with documented participation from civic representatives, domain experts, and custodians, capturing intent statements, risk hypotheses, and value commitments within AEIP intent disclosure bundles (AEIP §O.4).
  \item Each charter SHALL cite applicable legal and rights frameworks (including Appendix E §3 and Annex IV crosswalks) and SHALL publish an accessibility-oriented summary for public review via the Governance Disclosure System.
  \item Ethical assumptions, exclusions, and trade-offs SHALL be recorded as Signalled Interpretive Records; any deviation from the charter during implementation MUST be approved through a documented exception process referencing Appendix H §2 oversight procedures.
  \item Layer 1 artefacts SHALL interface with Layer 2 privacy controls, Layer 3 development policies, and Layer 4 instruction governance via versioned references; lack of current linkage SHALL be treated as a compliance gap requiring remediation before deployment continues.
\end{itemize}
}
\clearpage
