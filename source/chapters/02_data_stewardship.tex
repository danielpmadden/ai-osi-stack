% SPDX-License-Identifier: CC-BY-NC-ND-4.0
\chapter{Layer 02 — Data Stewardship}
\section{Narrative Intent}
\subsection*{Overview}
Layer 02 specifies how data entering, residing within, and exiting the Stack is governed. It separates
consent handling from custodial metadata, reflecting the crosswalk guidance that turned broad data
obligations into discrete, auditable controls. The intent is to guarantee lawful collection, transparent
use, and accountable retention strategies for all civic datasets.

\section{Normative Clauses}
\subsection*{Governance Requirements}
\begin{enumerate}
  \item \textbf{Consent Ledger.} Custodians SHALL maintain a ledger of consent artifacts, capturing purpose
        limitation, renewal cadence, and withdrawal pathways for each dataset.
  \item \textbf{Custodial Metadata.} Every dataset SHALL include Appendix I custodial metadata fields so that
        provenance, access decisions, and policy lineage can be reconstructed.
  \item \textbf{Data Minimization.} Collection activities SHALL document necessity tests and disposal triggers,
        ensuring the Stack only retains data essential to the civic services authorized.
  \item \textbf{Incident Escalation.} Data incidents SHALL invoke Appendix B escalation protocols with public
        disclosure timelines proportionate to the civic impact.
\end{enumerate}

\section{Plain-Speak Summary}
This layer organizes data rules into precise obligations. People must know why their information is used,
who touched it, and how long it will stay. If something goes wrong, there is a predetermined plan for
alerts and fixes. The goal is clear: data is treated as a public trust, not an expendable commodity.

\emph{License: Creative Commons Attribution–NonCommercial–NoDerivatives 4.0 International.}

\cleardoublepage

% Authored and maintained solely by the Custodial Editorial Committee.
% This is a non-operational, publication-grade governance artifact.
% No AEIP runtime specs or machine-readable schemas are included.
