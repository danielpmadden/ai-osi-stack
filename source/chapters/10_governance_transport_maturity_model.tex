% SPDX-License-Identifier: CC-BY-NC-ND-4.0
\chapter{Layer 10 — Governance Transport Maturity Model}
\section{Narrative Intent}
\subsection*{Overview}
Layer 10 converts Version 4 maturity checkpoints into tiered governance transport pathways. It explains
how policy updates move across jurisdictions, organizations, and technical environments while retaining
traceability. The narrative underscores responsible adoption that respects local civic commitments and
avoids unilateral deployments.

\section{Normative Clauses}
\subsection*{Transport Obligations}
\begin{enumerate}
  \item \textbf{Tier Definition.} Custodians SHALL maintain a tiered model describing readiness levels for
        governance transport, including prerequisite controls and civic prerequisites for each tier.
  \item \textbf{Pathway Documentation.} Every transport initiative SHALL include route maps detailing
        stakeholders, legal considerations, and rollback conditions.
  \item \textbf{Reciprocity Agreements.} When transporting governance artifacts across borders or institutions,
        the custodian SHALL secure reciprocal oversight agreements that honor local civic mandates.
  \item \textbf{Maturity Assessment.} Transport readiness SHALL be assessed using standardized scorecards, with
        results published in Appendix A vocabulary to avoid ambiguity.
\end{enumerate}

\section{Plain-Speak Summary}
Moving governance practices from one place to another requires planning. This layer sets the rules for
judging whether a community is ready, documenting who is involved, and making sure local values are
respected. No one should be surprised by a policy rollout.

\emph{License: Creative Commons Attribution–NonCommercial–NoDerivatives 4.0 International.}

\cleardoublepage
