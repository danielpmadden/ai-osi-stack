% © 2025 Daniel P. Madden — Custodial Author
% AI OSI Stack v5.0-open-core (Civic Standard Edition)

% © 2025 Daniel P. Madden. Custodial Edition – AI OSI Stack v5.0-open-core.
% Unauthorized reproductions or derivatives are not recognized custodial works.
% Refer to CANONICAL_PROVENANCE.yaml for official verification.
% SPDX-License-Identifier: CC-BY-SA-4.0

% Implements: tools/verify-aeip-signatures.py
% AI OSI Stack v5 — Canonical Deepbuild
% Chapter 00 — Introduction and Purpose
% Sources: Version 5 Draft, AEIP v1, Update Plans 1–10, Persona Architecture v2

\narrative{
The deepbuild introduction grounds the AI OSI Stack in its civic origins. Between 2022 and 2025, successive drafts responded to
public hearings, pilot deployments, and academic critiques. Version 5 consolidates that discourse into a layered constitution for
socio-technical governance. The introduction explains the problem the stack solves: democratic societies needed a reference
architecture that binds legal mandates, engineering practice, and civic participation into a single verifiable system. Without it,
transparency devolved into sporadic disclosures that rarely empowered citizens. With it, every commitment is tracked through AEIP
receipts, cross-referenced to appendices, and open to contestation.
}

\subsection*{Triple Register}
\textbf{Narrative Intent:} This chapter confronts the civic confusion that emerges when AI governance lacks a shared constitution, explaining how the Stack anchors obligations in a traceable civic mandate so communities know who is answering for each promise.
\textbf{Normative Clauses:}
\begin{itemize}
\item Custodians \shall{} map introductory assertions to \texttt{schemas/aeip/aeip-frame-schema.json} so that AEIP records tie narrative claims to enforceable controls.
\item Stewards \shall{} catalogue adoption rationales using \texttt{schemas/aeip-template.yaml} to preserve provenance for external auditors.
\item Briefing teams \should{} surface glossary anchors from \texttt{schemas/svc/semantic-registry.jsonld} when translating the chapter for civic audiences.
\end{itemize}
\textbf{Plain-Speak Summary:} The introduction lays out why the AI OSI Stack exists and whom it protects. It reminds readers that civic consent, legal mandates, and technical controls move together. It also points to the artefacts that document each promise so anyone can verify the origin. By following those records, a community can see why the Stack insists on openness.


\section*{Mandate for Reconstruction}
\narrative{
Historical files reveal an urgent demand to replace placeholder prose with enforceable obligations. Update Plans 6 through 10, the
Persona Architecture v2 dossier, and the Version 5 Draft all insisted that introductory materials do more than summarise—they must
instruct. The deepbuild methodology, therefore, stitches together narrative, normative, and evidentiary strands. The introduction
sets expectations for the entire work: each chapter will present a story of why the obligation matters and a rule set detailing how
it is enforced. Citizens should feel welcomed into the governance studio, not excluded by jargon or secrecy.
}
\normative{
Custodians \shall{} maintain an auditable linkage between introductory claims and the obligations enumerated in subsequent layers.
Compilation pipelines \shall{} include the historical citations enumerated in \texttt{versions/historical/}. Implementers MUST record
in (AEIP §O.3) their adoption rationale, referencing this introduction to demonstrate comprehension of the stack’s civic purpose.
Public briefings \should{} reference this chapter when explaining why the stack refuses to separate technical assurance from civic
legitimacy.
}

\section*{Layer Overview}
\narrative{
Layers 0 through 5 form the civic, ethical, data, model, instruction, and reasoning foundations. They establish a flow of
obligations: legitimacy arises from community consent (Layer 0); ethical duties anchor design decisions (Layer 1); data stewardship
builds trust (Layer 2); model development enforces accountable engineering (Layer 3); instruction governance ensures deliberative
alignment (Layer 4); reasoning exchange maintains safe operational dialogue (Layer 5). Higher layers extend this structure to
publication, participation, and interpretive continuity. The introduction previews these relationships so readers can navigate the
stack as an interconnected constitution rather than siloed policies.
}
\normative{
Readers \shall{} interpret every layer as mutually reinforcing. Implementers \shall{not} cherry-pick layers; partial adoption undermines
both compliance and legitimacy. AEIP submissions MUST reference the relevant layer identifiers when recording obligations and \shall{}
cite Appendix E §3 when privacy controls intersect with data or reasoning obligations. Oversight bodies \may{} sequence their reviews
according to local risk, but they \shall{} eventually attest to every layer before declaring full alignment.
}

\section*{Registers and Tone}
\narrative{
The dual-register structure is deliberate. Narrative passages document context, moral stakes, and historical lineage. Normative
passages specify enforceable obligations using ISO-2119 modals. This introduction commits to the plain-technical tone mandated by
the authoring profile so that multi-disciplinary audiences can engage without sacrificing precision. AEIP hooks embedded throughout
the chapters allow developers and auditors to trace requirements directly into lifecycle tooling while enabling civic readers to
understand their rights and responsibilities.
}
\normative{
Authors contributing to future revisions \shall{} preserve the dual-register format. Narrative text \may{} elaborate with case studies or
illustrative stories, but normative clauses MUST remain explicit, testable, and free from ambiguity. When adding new obligations,
writers \shall{} register the change in (AEIP §O.5) modification logs and \shall{} cite the consultation events that justified the shift.
Educational derivatives \shall{} retain both registers, even when paraphrasing, to avoid collapsing moral context into mere
compliance checklists.
}

\section*{Interpretive Anchors}
\narrative{
Interpretive principles, detailed in the front matter, are reiterated here because they guide how readers resolve conflicts. The
introduction emphasises that every transparency measure must protect against surveillance. It references Appendix E privacy
safeguards, Appendix B remediation protocols, and Appendix G participation norms as the guardrails for implementation. By placing
these anchors in the introduction, the deepbuild ensures that readers cannot ignore the broader constitutional logic while diving
into technical layers.
}
\normative{
Any ambiguity encountered in subsequent chapters \shall{} be resolved by returning to the interpretive principles. Implementers MUST
document their interpretive choices in (AEIP §O.4) reasoning logs, explicitly linking decisions to the non-revocable clause and the
relevant appendices. Oversight bodies \shall{} audit these logs annually and \shall{} issue remediation directives when choices drift
from the mandated principles. Civic participants \may{} submit interpretive challenges, which custodians \shall{} catalogue under
Appendix G procedures.
}

\section*{Reading Roadmap}
\narrative{
The introduction concludes with a roadmap for diverse audiences. Citizens are invited to begin with the narrative passages of each
layer before consulting the normative obligations. Policymakers can follow the cross references to Appendices D through H for
implementation guidance. Engineers and data custodians can trace normative statements into AEIP evidence templates. Academics and
legal scholars can compare the governance architecture to constitutional analogues archived in the historical corpus. This roadmap
ensures that the stack operates as a shared civic resource rather than a specialist manual.
}
\normative{
Dissemination programs \shall{} use this roadmap when onboarding new stakeholders. Training materials MUST cite the relevant sections
and \shall{} include exercises that walk participants through AEIP evidence creation. Custodians \shall{} monitor feedback channels to
identify confusion or misinterpretation and \shall{} file clarifications in (AEIP §O.7) communication logs. Public-facing summaries
\may{} condense the roadmap but \shall{} retain explicit references to the dual-register structure.
}

\n\subsection*{Verification and Enforcement}
Conformance is evidenced through artefacts \texttt{schemas/aeip/aeip-frame-schema.json}, \texttt{schemas/aeip-template.yaml}, and \texttt{schemas/svc/semantic-registry.jsonld} and corresponding AEIP audit records.
\clearpage