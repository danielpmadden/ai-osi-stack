% © 2025 Daniel P. Madden. Custodial Edition – AI OSI Stack v5.0-open-core.
% Unauthorized reproductions or derivatives are not recognized custodial works.
% Refer to CANONICAL_PROVENANCE.yaml for official verification.
% SPDX-License-Identifier: CC-BY-SA-4.0

% Implements: schemas/aeip/ccm-schema.json
% AI OSI Stack v5 — Canonical Deepbuild
% Chapter 04 — Layer 1: Ethical Charter
% Sources: Version 5 Draft, AEIP v1, Update Plans 2, 6, 9, Persona Architecture v2, Appendix E

\narrative{
Layer 1 transforms the civic mandate into an actionable ethical charter. It defines the values, rights, and duties that govern every
technical decision. The charter articulates what harms must be prevented, what dignities must be upheld, and how competing interests
are reconciled. Drawing from the Version 5 Draft and AEIP lifecycle guidance, the narrative explains how ethical commitments are
translated into operational controls, review criteria, and public accountabilities. Ethics is presented not as aspirational rhetoric
but as a binding contract with the communities whose lives are impacted by intelligent systems.
}

\subsection*{Triple Register}
\textbf{Narrative Intent:} Layer 1 tackles the frequent gap between stated corporate values and enforceable duties by translating the ethical charter into testable commitments that teams cannot ignore.
\textbf{Normative Clauses:}
\begin{itemize}
\item Stewards \shall{} encode charter clauses in \texttt{schemas/aeip/ccm-schema.json} before any development milestone is approved.
\item Compliance leads \shall{} update charter change logs through \texttt{schemas/ccm-template.yaml} whenever values evolve or conflicts are resolved.
\item Narrative editors \should{} cross-reference terms with \texttt{schemas/svc/semantic-registry.jsonld} so that ethical language stays consistent across layers.
\end{itemize}
\textbf{Plain-Speak Summary:} This chapter turns value statements into obligations people can check. It tells teams where to store the formal charter and how to record updates. The guidance keeps words like fairness and dignity tied to specific controls. Readers can see how ethical promises become requirements they can test.


\section*{Charter Composition}
\narrative{
Charters are co-authored documents that encode values into enforceable clauses. Update Plan 6 introduced a template linking each
ethical principle to evidence expectations, ensuring every clause can be audited. Persona Architecture v2 ensures that persona
behaviours align with charter values, preventing automated agents from undermining human intent. The narrative describes drafting
workshops where communities, engineers, and legal experts iterate on charter language until it balances ambition with practicality.
}
\normative{
Charter documents \shall{} enumerate core values, prohibited harms, mandatory mitigations, and appeal procedures. Each clause MUST map
to AEIP evidence requirements and \shall{} cite relevant appendices, including Appendix E §3 for privacy and Appendix F for equity.
Charters \shall{} be ratified through the same participatory processes used for mandates and MUST be published in accessible formats.
Implementers \shall{} log charter references in (AEIP §O.4) when making design decisions.
}

\section*{Ethical Risk Assessment}
\narrative{
Risk assessment operationalises the charter. Teams evaluate potential harms, rights impacts, and societal implications before
building or deploying systems. Update Plan 9 emphasised scenario analysis to anticipate edge cases, especially where marginalized
groups might bear disproportionate risk. The narrative explores how charter clauses translate into assessment checklists, community
consultations, and persona-driven simulations that reveal unintended consequences.
}
\normative{
Implementers \shall{} conduct ethical risk assessments at project inception, prior to major changes, and during periodic reviews.
Assessments MUST involve community representatives and \shall{} document findings in (AEIP §O.3) evidence bundles. Identified risks
\shall{} be categorised by severity and likelihood, with mitigation plans aligned to charter clauses. Projects \shall{not} progress to
Layer 2 until risk mitigations have been validated by oversight councils and recorded in Appendix B remediation trackers.
}

\section*{Ethical Governance in Operations}
\narrative{
Ethical charters guide day-to-day operations. Monitoring systems compare actual behaviour against charter expectations, triggering
alerts when deviations occur. The narrative recounts how civic monitors use ethical dashboards to trace decision outcomes, ensuring
they align with commitments around fairness, privacy, and agency. Persona Architecture v2 supports this by embedding ethical
constraints into agent behaviours, ensuring they reflect community-approved norms.
}
\normative{
Operational teams \shall{} integrate charter metrics into monitoring pipelines. Deviations MUST trigger incident workflows defined in
Appendix B, and corrective actions \shall{} reference the specific charter clauses implicated. AEIP operational logs \shall{} capture
ethical performance indicators, escalation outcomes, and stakeholder communications. If deviations persist, oversight bodies \shall{}
consider mandate suspension under Layer 0 authority.
}

\section*{Ethical Education and Capacity}
\narrative{
Charters require people capable of interpreting and applying them. The narrative discusses training programs for engineers,
policymakers, auditors, and civic participants. Update Plan 2 advocated for shared curricula anchored in AEIP records, ensuring
ethics education remains grounded in real governance artifacts. Community-led workshops allow residents to learn how to read charters
and hold institutions accountable.
}
\normative{
Custodians \shall{} provide ongoing ethics education tailored to each stakeholder group. Training materials MUST reference the
canonical charter and \shall{} include case studies documented in AEIP repositories. Participation in training \shall{} be recorded in
(AEIP §O.5) workforce logs. Institutions \shall{not} delegate responsibilities to personnel who have not completed required ethical
training. Civic education programs \should{} compensate participants for their time and expertise.
}

\section*{Ethical Amendment and Sunset}
\narrative{
Charters evolve as communities learn. Amendment processes incorporate new insights, address emergent harms, and retire clauses that
no longer serve the public. The narrative reflects on amendment debates captured in Update Plan 9, where communities negotiated the
balance between algorithmic efficiency and human oversight. Sunset clauses ensure charters remain relevant; expired provisions must
be renewed or replaced through participatory processes.
}
\normative{
Amendments to the ethical charter \shall{} follow documented procedures aligned with Appendix H succession and change control. Proposed
changes MUST include rationale, impact analysis, and public consultation plans. AEIP (AEIP §O.6) \shall{} record amendment debates,
voting outcomes, and implementation timelines. Sunset clauses \shall{} specify review dates, and if a clause lapses without renewal,
operations dependent on that clause MUST pause until replacement guidance is ratified. Emergency amendments \may{} be enacted under
Appendix B provisions but \shall{} undergo retrospective review within 30 days.
}

\n\subsection*{Verification and Enforcement}
Conformance is evidenced through artefacts \texttt{schemas/aeip/ccm-schema.json}, \texttt{schemas/ccm-template.yaml}, and \texttt{schemas/svc/semantic-registry.jsonld} and corresponding AEIP audit records.
% \begin{autogenerated}
\begin{autogenerated}
\begin{longtable}{p{0.22\textwidth}p{0.73\textwidth}}
\toprule
\textbf{Purpose} & Translate institutional values into enforceable, testable commitments for development teams. \ 
\midrule
\textbf{Obligations} & \begin{itemize}
\item Charter custodians \shall{encode each value as a control clause} with measurable acceptance criteria.
\item Compliance leads \shall{log amendments within seven days} of approval in the charter change ledger.
\item Narrative editors \should{synchronise terminology with the semantic registry} to avoid drift.
\end{itemize} \ 
\midrule
\textbf{Verification Artefact} & Ethical charter ledger with amendment diff, approval signatures, and semantic crosswalk. \ 
\midrule
\textbf{AEIP Schema} & \texttt{schemas/aeip/ccm-schema.json} \ 
\midrule
\textbf{Evidence Fields} & \begin{itemize}
\item \texttt{charter_values[].control_clause}
\item \texttt{amendments[].effective_on}
\item \texttt{glossary.references[]}
\end{itemize} \ 
\bottomrule
\end{longtable}
\textbf{Verification Example}\label{tab:layer1-verification}\par
\begin{lstlisting}[language=json,caption={Verification artefact for layer1},label={lst:layer1-example}]
{\n  "charter_values": [\n    {\n      "name": "Due Process",\n      "control_clause": "Appeal routes maintained for all automated denials",\n      "evidence": "appendix-k/tier-3"\n    }\n  ],\n  "amendments": [\n    {\n      "change_id": "ETH-2025-11-01",\n      "effective_on": "2025-11-15",\n      "approved_by": ["custodian", "community_review"]\n    }\n  ]\n}
\end{lstlisting}
\end{autogenerated}
% \end{autogenerated}

\clearpage