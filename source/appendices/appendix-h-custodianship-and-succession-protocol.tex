% AI OSI Stack v5 — Canonical Edition
% Appendix H — Custodianship & Succession Protocol
\narrative{
Appendix H describes how custodianship is conferred, exercised, and transitioned. The narrative follows the Custodial Assembly as it prepares for a succession ceremony. Outgoing custodian Mara has stewarded the Stack for a decade; incoming custodian Idris was selected through a civic nomination process. The Assembly convenes in a public hall where witnesses, auditors, and community delegates observe the handover.

Mara recounts the responsibilities of custodianship: maintaining integrity ledgers, convening transparency forums, and enforcing the non-revocable principle. She presents a continuity dossier referencing Appendices C and L, summarising major decisions and unresolved challenges. Idris pledges to uphold the canon, acknowledging accountability to citizens and federated partners. The narrative details rituals—key rotation, signing of AEIP succession manifests, and reading of the custodial oath. Civic witnesses validate the process, ensuring legitimacy and trust.

The appendix also covers contingency plans. It narrates a scenario where a custodian must be removed for cause after failing to remediate repeated human rights violations. The Custodial Assembly activates emergency protocols: an interim council assumes duties, public notices are issued, and an independent inquiry documents findings. Succession is portrayed as both ceremonial and pragmatic, designed to protect the Stack from capture or negligence.

Training and mentorship are highlighted. Prospective custodians complete apprenticeships, participate in meta-audits, and lead resilience exercises. Appendices E, F, and N feature prominently in their curriculum. The narrative emphasises that custodianship is a service role, not a seat of power; authority flows from community consent and adherence to the canon.
}
\normative{
Custodians SHALL be selected through transparent, participatory processes that include civic nominations, eligibility vetting, and public deliberation. Appointments MUST be ratified by a Custodial Assembly and recorded in AEIP succession manifests with provenance signatures referencing Appendix O. Terms of service SHALL be defined, with renewal contingent on performance reviews documented in the meta-audit ledger.

Succession events, whether planned or emergent, SHALL include key rotation, asset inventory, continuity briefings, and public attestation per Appendix N. Outgoing custodians MUST provide comprehensive dossiers covering decisions, open risks, and compliance status. Incoming custodians MUST acknowledge obligations and commit to uphold the non-revocable principle, human rights safeguards, and transparency requirements.

Removal for cause SHALL follow due-process procedures that include investigation, evidence review, opportunity for response, and public reporting. Interim custodianship arrangements MUST prevent gaps in governance and SHALL prioritise continuity of critical services. Training programmes for custodial candidates SHALL cover Appendices B–O, ensuring readiness to manage the Stack’s technical, legal, and civic dimensions.
}
