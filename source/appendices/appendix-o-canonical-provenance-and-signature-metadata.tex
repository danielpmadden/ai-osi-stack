% AI OSI Stack v5 — Canonical Edition
% Appendix O — Canonical Provenance & Signature Metadata
\narrative{
Appendix O chronicles how the canonical edition is sealed. The narrative follows provenance engineer Lior and legal custodian Amara as they assemble the signature metadata that anchors the canon’s legitimacy. They begin by consolidating evidence: AEIP manifests, integrity ledgers, and signed attestations from custodial quorums. Each artefact is hashed, timestamped, and linked to the master provenance statement.

Lior demonstrates the cryptographic workflow. Keys are generated, stored in hardware security modules, and rotated according to Appendix H protocols. Signatures are applied to the master PDF, ledger snapshots, and appendices. Amara ensures that legal notices reflect licensing terms (CC BY-NC-ND 4.0) and that signature ceremonies comply with jurisdictional requirements. Civic witnesses observe and co-sign, ensuring that provenance is not solely a technical exercise but a communal affirmation.

The appendix details metadata schemas capturing signer identities, roles, key fingerprints, signature algorithms, and validity periods. It explains how provenance data integrates with Appendix N verification guides, enabling citizens to confirm authenticity. The narrative concludes with archival practices: signed artefacts are deposited in redundant repositories, and periodic re-signing ceremonies renew trust for future generations.
}
\normative{
Canonical artefacts—including PDFs, AEIP manifests, ledger snapshots, and appendices—SHALL be signed using cryptographic algorithms that meet or exceed NIST-recommended security levels. Signature metadata MUST include signer identity, custodial role, key fingerprint, timestamp, algorithm, validity period, and reference hashes. Metadata SHALL be published in machine-readable formats accessible to public verification tools.

Key management SHALL follow Appendix H custodianship protocols, including generation within secure modules, multi-person control for activation, and scheduled rotation. Compromise or suspicion of compromise SHALL trigger immediate key revocation, re-signing of affected artefacts, and public notification via Appendix N channels. Signature ceremonies SHALL involve civic witnesses and be documented in Appendix L interpretive records.

Provenance archives SHALL maintain redundant storage, periodic integrity checks, and disaster recovery plans. Re-signing reviews SHALL occur at least every two years or upon major custodial transitions. All provenance activities SHALL be recorded in AEIP manifests and cross-referenced in Appendix C change logs to preserve traceability.
}
