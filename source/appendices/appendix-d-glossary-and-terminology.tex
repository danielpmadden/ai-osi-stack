% AI OSI Stack v5 — Canonical Edition
% Appendix D — Glossary & Terminology
\narrative{
Appendix D expands the lexicon introduced in Appendix A, providing detailed descriptions of canonical terms and the contexts in which they operate. The narrative frames the glossary as a guided reference walk led by archivist-poet Mireille, who believes that definitions should resonate with lived experience. She escorts readers through thematic clusters—Governance Roles, Technical Artefacts, Civic Processes, and Interpretive Instruments—illustrating each term with short vignettes.

When Mireille defines “Civic Mandate,” she recalls a town hall where residents co-authored Layer 0 obligations, linking the term to democratic legitimacy. “Interpretive Ledger” becomes a story about elders and youth annotating decisions with cultural meaning. “Federated Policy Partnership” is explained through a case where two cities coordinate transparency standards. The narrative includes sidebars that reference relevant appendices, enabling readers to dive deeper. Terms are cross-linked to AEIP schema fields, ensuring that technical implementers can map language into machine-readable structures.

The glossary also captures contested or evolving terminology. Mireille documents alternative interpretations, noting when a term carries different implications across jurisdictions. These entries include references to hermeneutic debates and public comment records, demonstrating that definitions remain open to revision through civic participation. The appendix closes with guidance on proposing new terms or requesting clarifications via Appendix N channels.
}
\normative{
The glossary SHALL provide authoritative yet evolving definitions for canonical terms. Each entry MUST include the term, narrative description, normative interpretation, relevant appendices, AEIP schema references, and version history. Glossary updates SHALL be reviewed by the Lexicon Commons and recorded in Appendix C change logs and Appendix L interpretive notes.

Terms flagged as contested SHALL include documentation of differing perspectives, citation of deliberation records, and guidance for interim usage. Implementers MUST reference the glossary when drafting policies, manifests, or public communications. Any deviation from glossary definitions SHALL be justified within the relevant artefact and cross-referenced to an approved interpretive note.

Glossary maintenance SHALL occur continuously, with quarterly publications of updated editions. Machine-readable exports (e.g., JSON-LD, CSV) SHALL accompany narrative PDFs to support AEIP integration and accessibility. Public contributions to the glossary MUST receive acknowledgement and disposition within 60 days, with responses published through Appendix N transparency tiers.
}
