% SPDX-License-Identifier: CC-BY-SA-4.0

% AI OSI Stack v5 — Canonical Edition
% Appendix D — Glossary & Terminology
\narrative{
Appendix D expands the lexicon introduced in Appendix A, providing detailed descriptions of canonical terms and the contexts in which they operate. The narrative frames the glossary as a guided reference walk led by archivist-poet Mireille, who believes that definitions should resonate with lived experience. She escorts readers through thematic clusters—Governance Roles, Technical Artefacts, Civic Processes, and Interpretive Instruments—illustrating each term with short vignettes.

When Mireille defines “Civic Mandate,” she recalls a town hall where residents co-authored Layer 0 obligations, linking the term to democratic legitimacy. “Interpretive Ledger” becomes a story about elders and youth annotating decisions with cultural meaning. “Federated Policy Partnership” is explained through a case where two cities coordinate transparency standards. The narrative includes sidebars that reference relevant appendices, enabling readers to dive deeper. Terms are cross-linked to AEIP schema fields, ensuring that technical implementers can map language into machine-readable structures.

The glossary also captures contested or evolving terminology. Mireille documents alternative interpretations, noting when a term carries different implications across jurisdictions. These entries include references to hermeneutic debates and public comment records, demonstrating that definitions remain open to revision through civic participation. The appendix closes with guidance on proposing new terms or requesting clarifications via Appendix N channels.
}
\narrative{
Mireille now presents a crosswalk of the Stack’s most frequently cited terms. Each entry pairs the canonical word with a one-line definition and the primary AEIP reference so readers can translate conversation into evidence. The crosswalk keeps Layer~7 disclosures, Layer~8 participation, and Appendix~N attestations aligned on language.}
\egin{longtable}{p{0.22\textwidth}p{0.68\textwidth}}
\toprule
\textbf{Term} & \textbf{One-Line Definition and AEIP Anchor} \ \midrule
AEIP & Assurance Evidence Integration Protocol that ties every obligation to verifiable artefacts (see \texttt{schemas/aeip/aeip-frame-schema.json}). \ \midrule
Custodian & Named steward accountable for governance duties and signatures, recorded in AEIP witness and custodian fields. \ \midrule
Canon & Ratified edition of the Stack, sealed through Appendix~O provenance manifests and integrity notices. \ \midrule
Manifest & Machine-readable listing of release artefacts and hashes, published in \texttt{meta/v5-manifest.yaml}. \ \midrule
Integrity Notice & Public verification log in 	exttt{INTEGRITY\_NOTICE.md} summarising commands, hashes, and witness statements. \ \midrule
Governance Integration Index & Quantitative score describing maturity posture per Appendix~C ladder stages. \ \midrule
Participation Ledger & Layer~8 record of civic inputs, feedback, and custodial responses stored in AEIP TEC-L schemas. \ \midrule
Triple Register & Narrative, normative, and plain-speak alignment block inserted in every chapter via 	exttt{ops/scripts/update_chapters.py}. \ \midrule
Hermeneutic Ledger & Appendix~L interpretive record capturing debate, dissent, and contextual annotations for governance decisions. \ \midrule
Override Authority & Escalation mandate defined in Appendix~B and recorded in 	exttt{schemas/decision-rationale-record.jsonld}. \ \midrule
\bottomrule
\end{longtable}

\normative{
The glossary \shall{} provide authoritative yet evolving definitions for canonical terms. Each entry MUST include the term, narrative description, normative interpretation, relevant appendices, AEIP schema references, and version history. Glossary updates \shall{} be reviewed by the Lexicon Commons and recorded in Appendix C change logs and Appendix L interpretive notes.

Terms flagged as contested \shall{} include documentation of differing perspectives, citation of deliberation records, and guidance for interim usage. Implementers MUST reference the glossary when drafting policies, manifests, or public communications. Any deviation from glossary definitions \shall{} be justified within the relevant artefact and cross-referenced to an approved interpretive note.

Glossary maintenance \shall{} occur continuously, with quarterly publications of updated editions. Machine-readable exports (e.g., JSON-LD, CSV) \shall{} accompany narrative PDFs to support AEIP integration and accessibility. Public contributions to the glossary MUST receive acknowledgement and disposition within 60 days, with responses published through Appendix N transparency tiers.
}
