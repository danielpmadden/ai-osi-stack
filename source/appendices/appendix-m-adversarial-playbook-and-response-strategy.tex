% SPDX-License-Identifier: CC-BY-SA-4.0

% AI OSI Stack v5 — Canonical Edition
% Appendix M — Adversarial Playbook & Response Strategy
\narrative{
Appendix M equips the Stack to anticipate and counter adversarial behaviour. The narrative follows the Adversarial Response Collective, a multidisciplinary team of security researchers, sociologists, civic defenders, and storytellers. They gather in the Simulation Arena to rehearse adversarial campaigns ranging from data poisoning and prompt injection to disinformation and legal harassment.

A simulation begins with a coordinated attempt to corrupt the hermeneutic ledger. Attackers inject forged interpretive records that misrepresent past decisions. The Collective traces the intrusion through AEIP audit trails, deploys integrity verification scripts, and convenes custodians to publish clarifications. Another scenario tests resilience against malicious actors flooding transparency channels with false incident reports. Civic defenders collaborate with community moderators to validate claims while preserving openness. The narrative also covers socio-technical threats such as coercion of custodians or exploitation of legal processes to chill oversight.

Throughout the playbook, the Collective emphasises ethical resilience. They refuse to meet hostility with secrecy, instead reinforcing transparency-without-surveillance. Response strategies integrate Appendices B (remediation), E (human rights), I (security), and K (transparency). Lessons are documented as adversarial patterns with recommended countermeasures, detection signals, and escalation paths. The appendix concludes with a call to share threat intelligence across federated partners and civic networks.
}
\normative{
The Stack \shall{} maintain an adversarial playbook cataloguing threat scenarios, detection methods, response actions, and recovery procedures. Playbooks MUST cover technical, social, legal, and reputational adversaries, with each entry linked to AEIP manifests, security controls (Appendix I), and transparency commitments (Appendix K). Updates \shall{} incorporate lessons from incidents, exercises, and intelligence-sharing forums.

Adversarial exercises \shall{} occur at least twice per year, involving cross-functional teams and civic representatives. Exercises MUST document hypotheses, observed tactics, response effectiveness, and improvement actions. Outcomes \shall{} feed into Appendices B, C, L, and N to ensure remediation, lineage, interpretive context, and public communication reflect new knowledge.

Threat intelligence sharing with federated partners \shall{} follow Appendix G agreements, including confidentiality safeguards, attribution norms, and reciprocal support. Any adversarial campaign affecting human rights \shall{} trigger immediate consultation with Appendix E custodians and may warrant public advisories through Appendix N transparency tiers.
}
