% © 2025 Daniel P. Madden — Custodial Author
% AI OSI Stack v5.0-open-core (Civic Standard Edition)

% © 2025 Daniel P. Madden. Custodial Edition – AI OSI Stack v5.0-open-core.
% Unauthorized reproductions or derivatives are not recognized custodial works.
% Refer to CANONICAL_PROVENANCE.yaml for official verification.
% SPDX-License-Identifier: CC-BY-SA-4.0

% AI OSI Stack v5 — Canonical Edition
% Appendix A — Vocabulary & Modal Definitions
\narrative{
Appendix A curates the canon’s vocabulary so that every reader can parse its commitments with precision. The narrative welcomes citizens, policymakers, engineers, and custodians into the Lexicon Commons, a collaborative space where words are negotiated, not assumed. A linguist named Saanvi chairs the session. She invites participants to trace the lineage of key terms—“custodian,” “persona,” “ledger,” “attestation,” “shall,” “must,” and “may”—drawing on AEIP v1, Persona v2, and prior editions of the Stack. Each term is contextualised with stories about its civic origins: how “custodian” emerged from debates about stewardship versus ownership; how “ledger” became an interpretive instrument rather than merely a technical database.

The narrative emphasises the dual-register design. Saanvi shares narrative etymologies, recounting how community advocates resisted opaque jargon by insisting on language that honoured agency. Engineers respond by mapping those narratives into modal definitions that encode obligations within AEIP manifests. Cross-references to Appendices E and O show how language shapes rights and provenance. When disagreements surface—for example, whether “audit” implies adversarial scrutiny or collaborative learning—the Commons convenes a mini-deliberation. Participants annotate the hermeneutic ledger with interpretive notes, demonstrating how vocabulary evolves transparently.

The appendix also introduces modal verbs aligned with ISO-2119 conventions. The narrative recounts a workshop where standardisation experts harmonised “\shall{},” “MUST,” “\should{},” and “\may{}” with civic expectations. They discuss the risks of modal inflation—overusing strong requirements can erode credibility—and describe safeguards such as periodic lexicon reviews and public feedback loops. The Lexicon Commons closes by publishing a living glossary that will be maintained through Appendix L’s ledger, ensuring future revisions remain accountable to the communities they affect.
}
\normative{
Canonical vocabulary \shall{} be maintained in the Lexicon Commons and versioned through Appendix L interpretive records. Definitions of core roles, artefacts, and procedures (including “custodian,” “persona,” “ledger,” “attestation,” “audit,” and “manifest”) MUST include narrative context, normative obligations, and AEIP linkage identifiers referencing Appendices E, F, and O. Revisions to vocabulary \shall{} undergo public notice and comment through Appendix N transparency channels before adoption.

Modal verbs \shall{} align with ISO-2119 semantics: “\shall{}” and “MUST” indicate binding requirements; “\should{}” denotes recommended practices subject to documented justification if not followed; “\may{}” indicates discretionary actions that must remain consistent with canonical principles. Any deviation from these definitions MUST be explicitly annotated in the relevant document section and cross-referenced in Appendix C change logs.

Lexicon audits \shall{} occur annually, convening linguistic experts, civic representatives, and implementers. Audit findings MUST document term usage patterns, identify ambiguity risks, and recommend updates. All vocabulary artifacts \shall{} be published in machine-readable formats for AEIP integration and in accessible narratives for civic audiences, ensuring language remains a shared governance instrument.
}