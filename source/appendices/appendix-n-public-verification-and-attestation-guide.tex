% © 2025 Daniel P. Madden. Custodial Edition – AI OSI Stack v5.0-open-core.
% Unauthorized reproductions or derivatives are not recognized custodial works.
% Refer to CANONICAL_PROVENANCE.yaml for official verification.
% SPDX-License-Identifier: CC-BY-SA-4.0

% AI OSI Stack v5 — Canonical Edition
% Appendix N — Public Verification & Attestation Guide
\narrative{
Appendix N invites citizens to verify the canon. The narrative follows librarian-activist Tarek as he hosts a public verification workshop at the Civic Technology Lab. Participants receive laptops, printed guides, and access to the integrity ledger. Tarek walks them through the attestation process: downloading the canonical PDF, checking SHA-512 hashes, verifying GPG signatures, and cross-referencing AEIP manifest indices. He emphasises that verification is a civic ritual, not a technical chore.

Workshop attendees share experiences. A community health worker verifies that remediation reports are authentic before discussing them with colleagues. A journalist confirms that appendices referenced in a news article are genuine. Students explore interpretive records and add annotations to the hermeneutic ledger. The narrative showcases accessibility features—screen-reader friendly instructions, multilingual support, and public kiosks for those without personal devices.

The appendix also covers escalation pathways. If a verification fails, Tarek explains how to report anomalies through the Attestation Service Desk. The desk coordinates with custodians, federated partners, and law enforcement if tampering is suspected. Transparency remains central: any incident triggers public notices and follow-up reports in Appendix C change logs. The story concludes with Tarek encouraging participants to become attestation mentors, expanding civic capacity for oversight.
}
\normative{
Public verification guides \shall{} include step-by-step instructions for obtaining canonical materials, verifying hashes, checking signatures, and cross-referencing AEIP manifests. Guides MUST be accessible, multilingual, and available in both digital and physical formats. Verification instructions \shall{} reference Appendices O (provenance), F (AEIP operations), and K (transparency tiers).

Attestation services \shall{} maintain support channels for reporting anomalies, including online forms, hotlines, and in-person assistance. Reports MUST be acknowledged within 24 hours and investigated promptly. Outcomes, including confirmation of authenticity or remediation actions, \shall{} be published through Appendix N transparency updates and recorded in AEIP manifests.

Custodians \shall{} host periodic public workshops, at least twice per year, to train citizens, journalists, and partners in verification practices. Workshop materials, attendance records, and feedback \shall{} be documented in the hermeneutic ledger. Failure to maintain attestation support constitutes a governance breach subject to meta-audit review.
}