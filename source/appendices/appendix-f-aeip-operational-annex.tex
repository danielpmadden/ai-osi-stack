% AI OSI Stack v5 — Canonical Edition
% Appendix F — AEIP Schema Set (v1.3)
\narrative{
Appendix F presents the Assurance Evidence Integration Protocol (AEIP) as the canonical evidence spine for the Stack. The narrative
returns to operations engineer Lin and civic archivist Priyanka, now walking through the schema registry room restored from
version~4. They trace how every "\shall{}" clause binds to a specific schema, how custodial signatures seal each record, and how the
manifest orchestrates deterministic field ordering. The annex emphasises that AEIP is not a library of templates but a living
protocol: manifests, instruction logs, model cards, civic charters, and participation receipts form an interoperable chain that can
be re-verified from first principles.
}

\narrative{
Lin demonstrates the AEIP Frame outline. Field order is fixed—\texttt{canonical\_metadata}, \texttt{subject}, \texttt{temporal\_seal},
\texttt{witnesses}, \texttt{evidence\_hashes}, and \texttt{custodian\_signatures}. Priyanka highlights the hash algorithm (SHA-512) and the
signing suite (Ed25519 canonical keys). When a new artefact enters the stack, custodians instantiate the AEIP Frame, map the
specialised payload (e.g., modelcard or incident report), and sign the bundle. Appendix~F publishes the full schemas, the registry
table with checksums, and an illustrative deterministic ordering snippet so practitioners can script the same validation sequence.
}

\normative{
AEIP custodians \shall{use the canonical field order defined in \texttt{schemas/aeip/aeip-frame-schema.json}}. Hashes \shall{be computed
with SHA-512 over UTF-8 normalised bytes}, recorded in \texttt{evidence\_hashes[*].value}, and validated against manifest entries.
Witnesses \shall{include role, jurisdiction, and verification channel}, ensuring that Appendix~O provenance audits can confirm custodial
continuity. Deviations \shall{be documented in Appendix~L ledgers with remediation deadlines}. No artefact \may{enter production without a
complete AEIP Frame accompanied by signed custody metadata}.
}

\normative{
Schema evolution \shall{follow the registry workflow captured below}. Release managers \shall{bump manifest entries, regenerate the
schema registry table, and publish new checksums in Appendix~N}. Operations teams \should{run regression validators against every
schema listed in this appendix before ratifying a revision}. Civic observers \may{reproduce the verification procedure} by fetching the
manifest, running the deterministic ordering script, and comparing published hashes with locally computed values.
}

\begin{quote}
\textbf{Deterministic Ordering Example.} AEIP frame ordering is preserved by serialising YAML keys exactly as shown below. Scripts
that alter key order or introduce additional whitespace invalidate signatures and \shall{trigger rejection alerts}.\end{quote}

\begin{lstlisting}[language=yaml,caption={AEIP Frame deterministic order example}]
canonical_metadata:
  aeip_version: "1.3"
  canonical_version: "AI OSI Stack v5"
subject:
  schema: "aeip/modelcard"
  reference_id: "MC-2025-0041"
temporal_seal:
  created_at: "2025-11-04T18:03:12Z"
  jurisdiction: "Ward 7"
witnesses:
  - role: "custodian"
    name: "Lin Okafor"
    verification_channel: "https://records.example.gov/aeip"
evidence_hashes:
  - algorithm: "sha512"
    value: "<hex-digest>"
custodian_signatures:
  - signer: "stack-custodian-ed25519"
    signature: "<base64>"
\end{lstlisting}

% SPDX-License-Identifier: CC-BY-SA-4.0

% \begin{autogenerated}
% Rationale: Publish AEIP schema registry inventory with optional checksum guidance for Phase 1+ synthesis.
% [SYNTHESIZED v5 PH1+]
\begin{autogenerated}
\subsection*{AEIP Schema Registry (v1.3)}
\begin{longtable}{p{0.25\textwidth}p{0.15\textwidth}p{0.15\textwidth}p{0.35\textwidth}}
\toprule
\textbf{Schema} & \textbf{Version} & \textbf{Type} & \textbf{SHA-256 Checksum (optional verification)} \ 
\midrule
\texttt{aeip-frame-schema.json} & AEIP 1.3 & JSON & 710324e35a363a3337ebe98af3b6ca6d2729a9d4204bd38954751f41fc9cf9d0 \ 
\midrule
\texttt{ccm-schema.json} & AEIP 1.3 & JSON & 29a1945cf1f98ee1ab99d8d23f852f3dd9b6ffe4a4538453fcd5f464f3ee0783 \ 
\midrule
\texttt{civic-charter-schema.json} & AEIP 1.3 & JSON & 5be1575052bbafe3afa44bc0471103696fe12b551d7b5ec79549b406b45d350a \ 
\midrule
\texttt{gds-schema.json} & AEIP 1.3 & JSON & 5506f0eab5ca4ba74eac3c9383b199fa43673db4081b6f8e1724d60181dfc15e \ 
\midrule
\texttt{incident-report-schema.json} & AEIP 1.3 & JSON & 276be0f2c03770b2c542e48376839ba9c027eea53e4cfe5a5d7d9e7db77413e4 \ 
\midrule
\texttt{instruction-log-schema.json} & AEIP 1.3 & JSON & 472d77fdc9d12a6fc7d2599c854837fca4f53bbe8c831740b350e181ae9166f9 \ 
\midrule
\texttt{modelcard-schema.json} & AEIP 1.3 & JSON & 584c2ef824b3f5032f703d8991398b2bccf615d2f714cf29ff936125b37e2bcf \ 
\midrule
\texttt{tecl-schema.json} & AEIP 1.3 & JSON & d5ab4524524418139fb390349052cd8b593edb16847741921fa237eedaa56659 \ 
\bottomrule
\end{longtable}
\subsection*{Normative Schema Listings}
\lstinputlisting[caption={AEIP schema: aeip-frame-schema.json}]{../../schemas/aeip/aeip-frame-schema.json}
\lstinputlisting[caption={AEIP schema: ccm-schema.json}]{../../schemas/aeip/ccm-schema.json}
\lstinputlisting[caption={AEIP schema: civic-charter-schema.json}]{../../schemas/aeip/civic-charter-schema.json}
\lstinputlisting[caption={AEIP schema: gds-schema.json}]{../../schemas/aeip/gds-schema.json}
\lstinputlisting[caption={AEIP schema: incident-report-schema.json}]{../../schemas/aeip/incident-report-schema.json}
\lstinputlisting[caption={AEIP schema: instruction-log-schema.json}]{../../schemas/aeip/instruction-log-schema.json}
\lstinputlisting[caption={AEIP schema: modelcard-schema.json}]{../../schemas/aeip/modelcard-schema.json}
\lstinputlisting[caption={AEIP schema: tecl-schema.json}]{../../schemas/aeip/tecl-schema.json}
\end{autogenerated}
% \end{autogenerated}


\normative{
AEIP schema listings \shall{be treated as normative artefacts}. Any institution claiming conformance \shall{reference the exact checksum}
from the registry table. Custodians \shall{store verification logs demonstrating checksum validation at release time}. Oversight actors
\should{sample schema listings during audits} and \may{publish independent hashes to Appendix~N}. Implementation guides \shall{note the
schema path, version, and validation command} for every operational playbook.
}
