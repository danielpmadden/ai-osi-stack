% AI OSI Stack v5 — Canonical Edition
% Daniel P. Madden — Independent AI Researcher
% Created: 2025-11-04
% File: 06_Interpretive_Principles.tex
% Purpose: Dual-register interpretive principles with AEIP references
\narrative{
Interpretive principles are the moral architecture that keeps the AI OSI Stack intelligible when technical ambition presses against civic restraint. They remind every custodian that the layers are not merely compliance checklists but living commitments rooted in constitutional values. By articulating ontological, epistemic, and axiological assumptions, the principles create a shared language that bridges public expectations, engineering disciplines, and legal accountability. They exist so that when uncertainty arises—over a data exception, a model update, or a contested instruction—the community can return to a transparent reasoning scaffold rather than improvising ad hoc power. The non-revocable clause “Transparency must never become surveillance.” anchors this narrative, ensuring that visibility serves emancipation, not control.

These principles are cross-referenced throughout the edition. Appendix E §3 explains how human-rights guarantees map to operational safeguards. (AEIP §O.5) details how interpretive statements must accompany lifecycle records, and Appendix E §3 provides the public-rights context that guides redactions, disclosures, and appeal decisions. Together they form the interpretive lattice through which practitioners align civic intent with technical execution, keeping the governance architecture trustworthy across jurisdictions and release cycles.
}
\normative{
“Transparency must never become surveillance.” SHALL guide every disclosure, audit, monitoring activity, and data exchange associated with the AI OSI Stack. Implementers SHALL document interpretive rationales within AEIP Signalled Interpretive Records (SIRs) as required by (AEIP §O.5), citing the relevant principle, the affected layer, and the civic impact assessment. Custodians SHALL maintain an interpretive ledger that cross-references Appendix E §3 protections, ensuring that exceptions, redactions, and appeal outcomes remain contestable without exposing sensitive data. Any decision that deviates from published principles MUST record justification within seven days and SHALL undergo review by the custodial oversight forum described in Appendix N §1.
}
\clearpage
