% AI OSI Stack v5 — Canonical Edition
% Daniel P. Madden — Independent AI Researcher
% Created: 2025-11-04
% File: 02_Abstract.tex
% Purpose: Dual-register abstract ~400 words for Phase 3 style
\narrative{
The AI OSI Stack v5 is narrated as the civic blueprint for trustworthy automation in a century where computational power intersects with democratic legitimacy. The abstract begins with the civic storyline: communities demanded that algorithmic services should not outpace constitutional guarantees. Layers 0 through 8 answer that demand by staging governance as an architecture of trust. Each layer translates civic expectations into operational rituals—charters, disclosures, registers, and audit trails—so that rights are preserved not by promise alone but by repeatable practice. The dual-register approach addresses audiences ranging from citizens and policymakers to implementers, legal custodians, and academic reviewers, inviting them to read the stack as both institutional memory and deployment manual.

Narratively, the stack recounts a journey from mandate to interface. Layer 0 roots authority in ratified civic consent. Layers 1 and 2 pair ethical commitment with data stewardship. Layers 3 through 5 convert development, instruction, and reasoning exchange into verifiable workflows, while Layers 6 through 8 (held constant from earlier editions) broadcast operations, public accountability, and civic participation. V5 consolidates update plans, public consultations, and provenance attestations recorded throughout 2024–2025, ensuring continuity with the canonical ledger while integrating refinements such as AEIP lifecycle evidence, persona architecture crosswalks, and Annex alignment. The narrative clarifies that transparency must never become surveillance; therefore, evidence is structured to be auditable without exposing citizens.

The abstract closes on the civic horizon that Version 5 opens. The governance architecture is positioned as a living constitution for socio-technical systems, reinforcing that custodianship is a public trust, that appeals must remain accessible, and that epistemic humility is engineered through interpretive principles. As jurisdictions and organisations adapt the stack, they inherit a responsibility to keep the narrative legible to those governed. The dual register ensures the work remains readable to the public while binding to practitioners charged with implementation.
}
\normative{
\begin{itemize}
  \item Adopting jurisdictions and operators SHALL implement Layers 0–8 of the AI OSI Stack with documented evidence artefacts anchored to AEIP lifecycle receipts, ensuring verifiable continuity across releases (AEIP §O.3).
  \item Custodians SHALL maintain governance disclosure systems (GDS) that publish canonical hashes, charters, risk assessments, and change logs in alignment with Appendix E §3 and Appendix I crosswalk obligations.
  \item Implementers SHALL preserve AEIP evidence bundles for audits, ensuring persona architectures, data stewardship controls, and reasoning interfaces remain contestable by civic stakeholders (AEIP §O.5).
  \item Civic appeal and remediation pathways SHALL be supported and publicly documented, enabling communities to challenge deployments, request redress, and observe corrective outcomes without degrading privacy protections (Appendix H §2).
\end{itemize}
}
\clearpage
