% Interpretive Principles — AI OSI Stack v5 Deepbuild
\narrative{
Interpretive principles provide the hermeneutic backbone for the stack. They articulate how the non-revocable clause
``Transparency must never become surveillance.'' translates into practical guidance for architects, auditors, policymakers, and
civic custodians. Each principle is sourced from the AEIP lifecycle, Appendices E and G, and the deliberative findings captured in
Persona Architecture v2. They instruct readers to balance disclosure with privacy, to privilege contestability over convenience,
and to recognise that governance artifacts are living instruments shaped by continual civic feedback.
}
\normative{
All interpretations of the AI OSI Stack SHALL begin with the non-revocable clause. Any disclosure, telemetry, or monitoring
control MUST document why it is necessary, SHALL reference Appendix E §3 privacy safeguards, and SHALL record the justification in
(AEIP §O.4) evidence trails. Decision makers SHALL articulate which interpretive principle guided their judgement and SHALL submit
the rationale to the hermeneutic ledger for audit. Exceptions MAY be requested under Appendix B emergency provisions but SHALL
expire automatically without renewed public review. Implementers SHALL design interfaces that expose these interpretive linkages
so civic participants can trace obligations without specialist tooling.
}
\clearpage
