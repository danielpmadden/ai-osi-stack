% Canonical Edition: AI OSI Stack v5.0
% Section: Part IV – Interpretive and Applied Canon
% Created: 2025-11-07
% Placement: After Layer 8 / Before Appendices
% Related Layers: 0–8
% Related Schemas: see /schemas/
\narrative{
Chapter 19A studies how millions of people invoke foundation models in daily decision-making, civic discourse, and crisis response. It distils empirical telemetry from global ChatGPT deployments into narratives about expectation, reliance, and emergent etiquette. Usage data is translated into civic evidence: which prompts become policy artefacts, how communities negotiate translation, and where semantic drift undermines trust. The chapter maps behavioural cohorts—public servants, educators, mutual-aid volunteers—and traces how they anchor meaning within the Stack’s custodial guarantees. Evidence from AEIP attestations and ledger annotations demonstrates why semantic governance must expand beyond engineering controls into social observatories.

These stories surface the social reality of model mediation. They show how citizens evaluate disclosure cues, compare machine and human testimony, and challenge inconsistencies through participatory audits. The narrative closes on a federated workshop where stewards reconcile conflicting interpretations using Appendix L hermeneutic protocols, proving that interpretive stewardship is inseparable from trustworthy deployment.
}

\subsection*{Triple Register}
\textbf{Narrative Intent:} This interpretive chapter tackles the human anxiety that usage agreements are empty words, translating the stack’s operational norms into daily practice for participants.
\textbf{Normative Clauses:}
\begin{itemize}
\item Custodians \shall{} capture trust covenants using \texttt{schemas/aeip/tecl-schema.json} before onboarding new communities.
\item Operators \shall{} log breaches or exceptions via \texttt{schemas/aeip/incident-report-schema.json} so restitution obligations activate.
\item Trust monitors \should{} update public dashboards referencing \texttt{schemas/integrity-ledger-entry.jsonld} to prove adherence over time.
\end{itemize}
\textbf{Plain-Speak Summary:} This chapter explains how everyday users can rely on the stack. It lists the agreements, incident processes, and transparency tools they can point to. People see exactly where promises live and how to enforce them. Trust is earned by recording proof, not slogans.

\normative{
Custodians \shall{} maintain interpretive observatories that correlate high-volume usage patterns with semantic integrity metrics. Participating institutions MUST publish AEIP evidence of how prompt taxonomies, refusal policies, and disclosure statements evolve under civic feedback. When empirical data reveals semantic drift, custodians \shall{} convene cross-layer councils to apply remediation protocols defined in Appendices B and L before updating public commitments.

Deployments leveraging Stack-aligned models MUST ensure that trust signals—provenance banners, refusal rationales, and policy glossaries—remain verifiable against ledger records. Civic feedback loops \shall{} be open to independent researchers under transparency-with-consent principles so that social reality remains legible, contestable, and governed.
}
\begin{quote}
\textit{Operational Note. Canon 19A maps to AEIP tags and custodian fields in TEC-L participation records and Layer~8 control tables, ensuring usage audits tie trust signals to recorded civic participation evidence.}
\end{quote}

\n\subsection*{Verification and Enforcement}
Conformance is evidenced through artefacts \texttt{schemas/aeip/tecl-schema.json}, \texttt{schemas/aeip/incident-report-schema.json}, and \texttt{schemas/integrity-ledger-entry.jsonld} and corresponding AEIP audit records.
