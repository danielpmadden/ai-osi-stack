% Canonical Edition: AI OSI Stack v5.0
% Section: Part IV – Interpretive and Applied Canon
% Created: 2025-11-07
% Placement: After Layer 8 / Before Appendices
% Related Layers: 0–8
% Related Schemas: see /schemas/
\narrative{
Chapter 20 codifies rhetoric as infrastructure. It narrates convenings where linguists, ethicists, and civic archivists confront how language architectures steer public meaning. The storyline follows a semantic incident review in which divergent policy glossaries caused contradictory model responses. Stewards deploy semantic version control (SVC) registries, reconcile translation chains, and publish diffable meaning statements so that rhetoric remains accountable. Each intervention demonstrates that linguistic integrity—consistency of terms, tone, and obligations—is as critical as cryptographic integrity.

The chapter illustrates verifiable language governance in action. AEIP bundles record speech-act provenance, schema-aligned claims, and consent for quotation. Semantic registrars oversee updates, enforce context windows, and prevent rhetorical weaponisation. Communities participate through controlled vocabularies, contested term hearings, and public semantic audits. By the conclusion, rhetoric is repositioned as a shared civic asset requiring maintenance, traceability, and rights of appeal.
}
\normative{
Canonical deployments SHALL maintain semantic registries using verifiable version control, ensuring every normative statement, refusal rationale, and disclosure notice links to immutable provenance records. Updates to rhetorical baselines MUST pass through participatory review with ledger-backed deliberation trails. Custodians SHALL provide multilingual reconciliations and publish diff reports that trace semantic evolution across jurisdictions.

Implementers MUST integrate linguistic integrity checks into release pipelines so that model updates cannot ship with ambiguous or conflicting obligations. External auditors SHALL be granted read access to semantic registries under transparency-with-consent safeguards, enabling independent verification of rhetorical commitments.
}
