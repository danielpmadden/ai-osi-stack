% © 2025 Daniel P. Madden. Custodial Edition – AI OSI Stack v5.0-open-core.
% Unauthorized reproductions or derivatives are not recognized custodial works.
% Refer to CANONICAL_PROVENANCE.yaml for official verification.
% SPDX-License-Identifier: CC-BY-SA-4.0

% Canonical Edition: AI OSI Stack v5.0
% Section: Part IV – Interpretive and Applied Canon
% Created: 2025-11-09
% Placement: After Layer 8 / Before Appendices
% Related Layers: 0–8
% Related Schemas: see /schemas/
\narrative{
Chapter 20 codifies rhetoric as infrastructure. It narrates convenings where linguists, ethicists, and civic archivists confront how language architectures steer public meaning. The storyline follows a semantic incident review in which divergent policy glossaries caused contradictory model responses. Stewards deploy semantic version control (SVC) registries, reconcile translation chains, and publish diffable meaning statements so that rhetoric remains accountable. Each intervention demonstrates that linguistic integrity—consistency of terms, tone, and obligations—is as critical as cryptographic integrity.

The chapter illustrates verifiable language governance in action. AEIP bundles record speech-act provenance, schema-aligned claims, and consent for quotation. Semantic registrars oversee updates, enforce context windows, and prevent rhetorical weaponisation. Communities participate through controlled vocabularies, contested term hearings, and public semantic audits. By the conclusion, rhetoric is repositioned as a shared civic asset requiring maintenance, traceability, and rights of appeal.
}

\subsection*{Triple Register}
\textbf{Narrative Intent:} This chapter tackles semantic drift and rhetorical manipulation, ensuring language across the stack remains faithful to civic intent rather than marketing spin.
\textbf{Normative Clauses:}
\begin{itemize}
\item Semantic stewards \shall{} register contested or novel terms in \texttt{schemas/svc/semantic-registry.jsonld} before publication.
\item Authors \shall{} attach annotated discourse records via \texttt{schemas/interpretive-trace-package.jsonld} to expose persuasive context.
\item Drafting teams \should{} log AI-assisted edits through \texttt{schemas/governance/ai-assisted-drafting.jsonld} to disclose machine contributions.
\end{itemize}
\textbf{Plain-Speak Summary:} This chapter keeps the stack’s language honest. It records definitions, rhetorical context, and AI editing trails. Readers can see how words are chosen and challenged. That clarity protects the public from subtle semantic shifts.

\normative{
Canonical deployments \shall{} maintain semantic registries using verifiable version control, ensuring every normative statement, refusal rationale, and disclosure notice links to immutable provenance records. Updates to rhetorical baselines MUST pass through participatory review with ledger-backed deliberation trails. Custodians \shall{} provide multilingual reconciliations and publish diff reports that trace semantic evolution across jurisdictions.

Implementers MUST integrate linguistic integrity checks into release pipelines so that model updates cannot ship with ambiguous or conflicting obligations. External auditors \shall{} be granted read access to semantic registries under transparency-with-consent safeguards, enabling independent verification of rhetorical commitments.
}
% Rationale: Anchor Canon operational linkage to AEIP evidence pathways for Phase 1+ synthesis.
% [SYNTHESIZED v5 PH1+]
\begin{quote}
\textit{Operational Note. Canon 20 aligns with AEIP semantic registry status metadata and Layer~5 reasoning exchange controls, keeping linguistic commitments traceable to registries and dialogue archives.}
\end{quote}

\n\subsection*{Verification and Enforcement}
Conformance is evidenced through artefacts \texttt{schemas/svc/semantic-registry.jsonld}, \texttt{schemas/interpretive-trace-package.jsonld}, and \texttt{schemas/governance/ai-assisted-drafting.jsonld} and corresponding AEIP audit records.