% Canonical Edition: AI OSI Stack v5.0
% Section: Part IV – Interpretive and Applied Canon
% Created: 2025-11-07
% Placement: After Layer 8 / Before Appendices
% Related Layers: 0–8
% Related Schemas: see /schemas/
\narrative{
Chapter 22 formalises persona architecture as the primary design unit for responsible AI. It recounts the evolution from monolithic assistants to bounded, role-specific personas with accountable provenance. Through case studies—an urban-planning analyst, a crisis-translation liaison, and a youth-safety moderator—the narrative shows how persona manifests integrate Layer 0–8 obligations, limit operational scope, and advertise escalation routes. Custodians emphasise persona charters, capability manifests, and retirement ceremonies documented in Appendix L ledgers.

The chapter details governance patterns for persona lifecycle management. Stakeholders co-design guardrails, align training corpora with civic mandates, and enforce separation-of-duties across persona portfolios. Meta-governance sessions illustrate how conflicting persona behaviours are reconciled via schema-driven manifests and public attestation. Persona architecture thus becomes an ethical scaffold: it anchors accountability, clarifies authority, and ensures human stewards remain visible in every interaction.
}
\normative{
Implementers SHALL define persona manifests that specify mandate, scope, escalation procedures, and alignment evidence. Each persona MUST reference relevant schemas, including semantic registries, AEIP controls, and care obligations. Custodians SHALL publish persona registries with versioned change logs and deprecation notices so communities can verify authority boundaries.

Persona deployment pipelines MUST include pre-release conformity checks against canonical obligations, with results published to hermeneutic ledgers. When personas intersect with sensitive domains—safety, education, health—institutions SHALL convene participatory reviews to validate cultural competence, accessibility, and duty-of-care commitments before activation.
}
