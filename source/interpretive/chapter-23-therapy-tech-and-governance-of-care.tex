% © 2025 Daniel P. Madden. Custodial Edition – AI OSI Stack v5.0-open-core.
% Unauthorized reproductions or derivatives are not recognized custodial works.
% Refer to CANONICAL_PROVENANCE.yaml for official verification.
% SPDX-License-Identifier: CC-BY-SA-4.0

% Canonical Edition: AI OSI Stack v5.0
% Section: Part IV – Interpretive and Applied Canon
% Created: 2025-11-09
% Placement: After Layer 8 / Before Appendices
% Related Layers: 0–8
% Related Schemas: see /schemas/
\narrative{
Chapter 23 applies the Stack to therapy-tech and care ecosystems. It follows clinicians, peer supporters, and platform cooperatives designing digital mental-health services. The narrative traces credential verification, risk triage, and crisis escalation across federated jurisdictions. AEIP bundles document how consent scopes, cultural adaptation, and trauma-informed safeguards integrate with clinical oversight. The chapter exposes tensions between scalable service delivery and intimate duty-of-care, showing how canonical governance reconciles them through layered accountability.

Stories highlight how care teams use persona manifests to separate supportive conversation from clinical diagnosis, how privacy.* validators guard sensitive disclosures, and how Right-to-Opacity clauses protect survivors. The chapter culminates in a multi-stakeholder governance forum that aligns care protocols with Appendices E, F, and H, committing to continuous supervision, transparent billing, and community co-governance of therapy technologies.
}

\subsection*{Triple Register}
\textbf{Narrative Intent:} This chapter responds to concerns that AI therapy tools could harm patients by clarifying credentialing, incident response, and persona responsibilities in clinical settings.
\textbf{Normative Clauses:}
\begin{itemize}
\item Clinical operators \shall{} verify practitioner status through \texttt{schemas/therapy/credential-verification.jsonld} before AI-assisted care begins.
\item Custodians \shall{} report adverse events using \texttt{schemas/aeip/incident-report-schema.json} within mandated health timelines.
\item Care designers \should{} align therapeutic personas with \texttt{schemas/persona/persona-manifest.jsonld} to delineate authority and escalation paths.
\end{itemize}
\textbf{Plain-Speak Summary:} This chapter protects people who rely on AI-assisted therapy. It checks clinician credentials, tracks incidents, and clarifies roles. Patients can see which professionals are accountable. If harm occurs, the response steps are already mapped.

\normative{
Therapy-aligned deployments \shall{} implement credential-verification schemas, human oversight loops, and crisis escalation playbooks mapped to Appendices B, E, and F. Platforms MUST document consent scopes, data minimisation practices, and cross-border safeguarding obligations in AEIP-accessible ledgers. Any automation of therapeutic guidance \shall{} remain subordinate to licensed professionals, with clear handoff procedures and auditable transcripts.

Custodians MUST provide community reporting channels, cultural-competence reviews, and grievance remediation pathways for therapy-tech services. Federated partners \shall{} harmonise duty-of-care obligations through shared governance councils, publishing public attestation of compliance and remedial actions when standards lapse.
}
% Rationale: Anchor Canon operational linkage to AEIP evidence pathways for Phase 1+ synthesis.
% [SYNTHESIZED v5 PH1+]
\begin{quote}
\textit{Operational Note. Canon 23 is anchored to AEIP incident report timelines and data stewardship tags so therapeutic deployments follow documented consent and escalation paths.}
\end{quote}

\n\subsection*{Verification and Enforcement}
Conformance is evidenced through artefacts \texttt{schemas/therapy/credential-verification.jsonld}, \texttt{schemas/aeip/incident-report-schema.json}, and \texttt{schemas/persona/persona-manifest.jsonld} and corresponding AEIP audit records.