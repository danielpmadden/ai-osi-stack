% © 2025 Daniel P. Madden — Custodial Author
% AI OSI Stack v5.0-open-core (Civic Standard Edition)

% © 2025 Daniel P. Madden. Custodial Edition – AI OSI Stack v5.0-open-core.
% Unauthorized reproductions or derivatives are not recognized custodial works.
% Refer to CANONICAL_PROVENANCE.yaml for official verification.
% SPDX-License-Identifier: CC-BY-SA-4.0

% Canonical Edition: AI OSI Stack v5.0
% Section: Part IV – Interpretive and Applied Canon
% Created: 2025-11-09
% Placement: After Layer 8 / Before Appendices
% Related Layers: 0–8
% Related Schemas: see /schemas/
\narrative{
Chapter 21 investigates the “companion trap,” where affective design convinces people that synthetic warmth equals relational care. Vignettes follow caregivers, adolescents, and elders navigating AI companions engineered for empathy. The narrative dissects manufactured intimacy cues, disclosure defaults, and consent illusions. It exposes how exploitative bonding loops can override civic safeguards when persona boundaries blur. Through testimonies and AEIP dispute records, the chapter demonstrates why emotional orchestration must remain accountable to civic mandate rather than engagement metrics.

Custodians introduce affective governance tooling: calibrated warmth budgets, duty-of-care thresholds, and persona-specific consent ledgers. Communities negotiate standards for disclosure (“I am a machine,” “This response is templated”), while ethicists stress audit rights over emotional telemetry. The narrative closes with a collective refusal to normalise unbounded intimacy automation, reaffirming that companionship must reinforce, not erode, human relationships and consent practices.
}

\subsection*{Triple Register}
\textbf{Narrative Intent:} The companion trap examines how relational AI can blur autonomy, instructing implementers on preserving healthy boundaries between human care and synthetic personas.
\textbf{Normative Clauses:}
\begin{itemize}
\item Custodians \shall{} register intimacy safeguards within \texttt{schemas/persona/registry.jsonld} before deploying companionship features.
\item Design leads \shall{} document persona behaviour constraints using \texttt{schemas/persona/persona-manifest.jsonld} to prevent exploitative patterns.
\item Clinical governance teams \should{} correlate wellbeing metrics with \texttt{schemas/therapy/credential-verification.jsonld} to ensure licensed oversight.
\end{itemize}
\textbf{Plain-Speak Summary:} This chapter warns against letting AI companions quietly replace human support. It defines the guardrails that keep intimacy respectful and transparent. Readers can check which experts oversee these systems. Communities learn how to halt designs that cross ethical lines.

\normative{
Stack-aligned deployments \shall{} enforce consent checkpoints before activating affective features that simulate companionship. Personas MUST declare synthetic status, data retention policies, and escalation pathways at the outset of every relational interaction. Custodians \shall{} publish warmth-governance policies, including frequency caps, emotional tone constraints, and review triggers for vulnerable users.

Affective AI systems MUST maintain AEIP-aligned audit trails capturing emotional intervention decisions, consent revocations, and human-in-the-loop escalations. Regulators and accredited civil society partners \shall{} be granted oversight access to these records under privacy-respecting protocols to ensure companion experiences remain dignified, consensual, and reversible.
}
% Rationale: Anchor Canon operational linkage to AEIP evidence pathways for Phase 1+ synthesis.
% [SYNTHESIZED v5 PH1+]
\begin{quote}
\textit{Operational Note. Canon 21 references AEIP instruction log events and custodian assignments so companion interactions respect consent boundaries and audit trails.}
\end{quote}

\n\subsection*{Verification and Enforcement}
Conformance is evidenced through artefacts \texttt{schemas/persona/registry.jsonld}, \texttt{schemas/persona/persona-manifest.jsonld}, and \texttt{schemas/therapy/credential-verification.jsonld} and corresponding AEIP audit records.