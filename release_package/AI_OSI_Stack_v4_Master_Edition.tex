\documentclass[11pt]{article}
\usepackage[T1]{fontenc}
\usepackage[utf8]{inputenc}
\usepackage{lmodern}
\usepackage[margin=1in]{geometry}
\usepackage{setspace}
\usepackage{hyperref}
\usepackage{longtable}
\usepackage{array}
\usepackage{booktabs}
\usepackage{fancyhdr}
\usepackage{tikz}
\usepackage{enumitem}
\setstretch{1.2}
\pagestyle{fancy}
\fancyhf{}
\rfoot{\thepage}
\lhead{The AI OSI Stack v4 Master Edition}
\rhead{Daniel P. Madden}
\title{The AI OSI Stack: Version 4\\\large Master Edition — Canonical Blueprint Integration}
\author{Daniel P. Madden}
\date{November 2025}
\newcommand{\licenseinfo}{CC BY-NC-ND 4.0}
\begin{document}
\maketitle
\thispagestyle{empty}
\begin{center}
\textbf{License:} \licenseinfo\\[0.5em]
\textbf{DOI:} \emph{To be assigned}\\[1em]
\begin{minipage}{0.9\textwidth}
\small
\textbf{Abstract.} Artificial intelligence is moving from discrete products to systemic infrastructure. Yet most AI is still designed and governed as if it were a single opaque system. This master edition consolidates the AI OSI Stack Version 4 into a single normative artifact that harmonizes conceptual blueprints, transport specifications, governance artifacts, implementation notes, and forward-looking expansion plans. The document integrates Persona Architecture, Epistemology by Design, and the AI Epistemic Infrastructure Protocol (AEIP) into a cohesive governance framework suitable for publication, stewardship, and citation.
\end{minipage}
\end{center}
\vspace{1em}
\noindent\fbox{\parbox{0.95\textwidth}{\textbf{Normative Language Notice.} This specification uses normative language consistent with ISO/IEC~42010 and NIST conventions. ``SHALL'' denotes mandatory requirements, ``SHOULD'' denotes strong recommendations, and ``MAY'' denotes optional practices. Interpretations SHALL preserve authorial intent: layered accountability, epistemic integrity, and human dignity as binding design constraints.}}
\clearpage
\tableofcontents
\clearpage
\listoffigures
\clearpage
\listoftables
\clearpage
\section{Introduction}
\emph{Scope: Establishes lineage, purpose, and foundational context for the Master Edition, integrating textual sources across prior releases.}
The AI OSI Stack v4 Master Edition consolidates Daniel~P.~Madden's governance architecture into a single canonical reference that aligns operational controls, epistemic integrity, and human dignity across the entire lifecycle of intelligent systems. This Master Edition resolves fragmentation from earlier versions by embedding Persona Architecture, Epistemology by Design, and the AI Epistemic Infrastructure Protocol (AEIP) directly within the layered stack. The intent is to provide institutions with a deployable governance blueprint compatible with the NIST AI Risk Management Framework, ISO/IEC~42001, the European Union AI Act, and the TRUST framework while remaining adaptable to sector-specific constraints.

\subsection{Lineage of Versions v1--v3}
\emph{Scope: Documents the evolution from foundational concept through epistemic integration, referencing changelog history.}
The release history captured in Appendix~C traces the maturation of the architecture:\begin{itemize}[leftmargin=*]
    \item \textbf{v1.0 --- Foundational Stack Overview (2019-09-01).} Introduced the seven-layer governance metaphor, integrity ledger concept, and initial stewardship obligations for AI services.\item \textbf{v2.0 --- Persona Architecture Expansion (2020-12-15).} Added Persona Architecture briefs, affective constraint models, and role-specific accountability patterns.\item \textbf{v3.0 --- Epistemology Alignment (2022-03-10).} Integrated Epistemology by Design, introduced Interpretive Trace Packages, and prepared AEIP compatibility guidance.\item \textbf{v4.0 --- Integrated Canonical Specification (2024-07-22).} Published the AI OSI Stack v4 Test Integrated edition, harmonizing governance artifacts, temporal legitimacy, and the Implementation Maturity Model.\item \textbf{v4 Protocol Blueprint --- Blueprint Complete Reference Implementation (2025-11-30).} Finalized the offline-first blueprint codebase, added governance maps, deferred system registry, validation artifacts, and Zenodo-ready release package for archival DOI issuance.
\end{itemize}

\subsection{Purpose of Version 4}
\emph{Scope: Presents the publication's objectives and alignment with the offline-first blueprint implementation notes.}
The AI OSI Stack operates as a governance-first reference implementation. Architectural intent is finalized in this blueprint; operational deployment is deferred until custodial councils confirm readiness. Each module, schema, and artifact is engineered to demonstrate how stewardship evidence MAY be produced, validated, and archived. Implementers SHALL treat the codebase as a canonical rehearsal environment rather than a production network. All tooling, tests, and examples SHALL execute without network access. Offline execution preserves audit reproducibility, protects unfinished persona credentials, and prevents accidental data egress. Where integrations with external systems are envisioned (e.g., Persona-PKI, Open Governance Registry), the repository provides placeholders and schema contracts only. Operators SHALL stage integrations in isolated sandboxes and SHALL NOT connect this blueprint to live infrastructure without explicit custodial authorization.

\subsection{Relationship to Persona Architecture and Epistemology by Design}
\emph{Scope: Describes integration of Persona Architecture and Epistemology by Design within the stack.}
Persona Architecture anchors Layer~4 by establishing accountable instruction sets, while AEIP operationalizes Layer~5 as the protocol spine that transmits epistemic commitments. Governance artifacts at Layer~7 distribute power by exposing stewardship evidence to external auditors and public stakeholders. Epistemology by Design supplies interpretive traceability, ensuring that reasoning states, decision artifacts, and socio-technical context remain inseparable. Dignity operates as a non-negotiable constraint: Persona Architecture requires agents to declare identity and intent at session initiation, ensuring clarity of role and accountability. The module prohibits intimacy simulation, manipulative sentiment scripts, and coercive affective cues. Violations trigger automatic Oversight Action Memoranda and public disclosure through Layer~7 portals. This preserves the ``dignity as constraint'' principle articulated across Daniel P.~Madden's corpus while providing enforceable levers for auditors.

\subsection{Jurisdictional Neutrality and Interpretive Authority}
\emph{Scope: Reinforces interpretive rules across jurisdictions.}
This framework is jurisdiction-neutral and intended for international application. In the event of conflicting legal or linguistic interpretations, interpretive authority SHALL defer to the author's defined principles of transparency, accountability, and dignity by design --- not to localized or translated semantics.
\section{Abstract}
\emph{Scope: Summarizes scope, purpose, and integrated sources for the Master Edition.}
Artificial intelligence is moving from discrete products to systemic infrastructure. Yet most AI is still designed and governed as if it were a single opaque system. This creates concentration risk, weak accountability, and an environment in which a few model or API providers can shape whole sectors. The AI OSI Stack separates the AI ecosystem into seven governance layers and an optional civic precursor, enabling targeted regulation, auditable decision artifacts, persona-based safety layers, and protection against interface monopolies. This Master Edition integrates the conceptual reference, the blueprint-complete offline implementation, and the AEIP reasoning handshake, enabling design-time assurance and audit-ready provenance.

\subsection{Repository Overview Narrative}
\emph{Scope: Incorporates the root README overview and abstract.}
\subsubsection*{\textbf{The AI OSI Stack: A Governance Blueprint for Scalable and Trusted AI}}
\subparagraph{\textbf{Author:} Daniel P. Madden -- Independent AI Researcher}
\subparagraph{\textbf{Version:} 4.0 (Expanded --- November 2025)}

---

\paragraph{\textbf{Abstract}}

Artificial intelligence is moving from discrete products to systemic infrastructure. Yet most AI is still designed and governed as if it were a single opaque system. This creates concentration risk, weak accountability, and an environment in which a few model or API providers can shape whole sectors.

This paper proposes the \textbf{AI OSI Stack}, a seven-layer architectural and governance framework for clarifying how AI is built, where risks concentrate, and how trust can be made portable. Inspired by the Open Systems Interconnection (OSI) model in networking, the AI OSI Stack separates the AI ecosystem into:

\begin{enumerate}[leftmargin=*]
\item \textbf{Physical / Hardware}
\item \textbf{Model Architecture}
\item \textbf{Training / Optimization}
\item \textbf{Instruction / Control}
\item \textbf{Interface / Protocol}
\item \textbf{Application}
\item \textbf{Governance / Trust}
\end{enumerate}

This separation of concerns turns AI governance from a reactive afterthought into a \textbf{design-time feature}. It enables targeted regulation, auditable decision artifacts, persona-based safety layers, and protection against interface monopolies.  

The Stack is intended for \textbf{AI labs, policymakers, enterprise architects, and standards bodies} that need an operational map for trustworthy AI.

---

\paragraph{\textbf{Overview}}

The \textbf{AI OSI Stack} defines a layered model aligning technical, ethical, and institutional responsibilities. Version 4 consolidates previous releases into a single reference that integrates \emph{Persona Architecture}, \emph{Epistemology by Design}, and the \emph{AI Epistemic Infrastructure Protocol (AEIP)}.  

The Stack treats governance as a foundational design constraint and enables \textbf{trust to remain portable} across heterogeneous AI services, organizations, and jurisdictions. It transforms governance into a form of infrastructure---where accountability becomes auditable, composable, and measurable.

---

\paragraph{\textbf{Purpose and Scope}}

This publication serves research, policy, and enterprise implementation communities.  

The repository contains both:
\begin{itemize}[leftmargin=*]
\item the \textbf{conceptual reference}, and
\item the \textbf{blueprint-complete offline implementation} --- operable without databases, APIs, or live network services.
\end{itemize}

The blueprint demonstrates \textbf{executable governance concepts} through machine-readable schemas, a local integrity ledger, and the \textbf{AEIP reasoning handshake}, enabling design-time assurance and audit-ready provenance.

---

\paragraph{\textbf{Core Layers}}

| \textbf{Layer} | \textbf{Designation} | \textbf{Governance Focus} |
|:---:|:---|:---|
| 1 | \textbf{Physical / Hardware} | Compute provenance, energy accountability, and secure supply chain. |
| 2 | \textbf{Model Architecture} | Architecture disclosure, capability bounding, and artifact traceability. |
| 3 | \textbf{Training / Optimization} | Dataset governance, optimization transparency, and performance guarantees. |
| 4 | \textbf{Instruction / Control} | Persona policies, prompt governance, and intervention channels. |
| 5 | \textbf{Interface / Protocol} | Interoperable exchanges, handshake integrity, and API stewardship. |
| 6 | \textbf{Application} | Service obligations, duty-of-care enforcement, and operational safeguards. |
| 7 | \textbf{Governance / Trust} | Assurance attestation, oversight engagement, and accountability ledgers. |

---

\paragraph{\textbf{Implementation Blueprint}}

The \textbf{Blueprint Implementation} anchors the conceptual stack in executable artifacts. It demonstrates how layered governance can be expressed in machine-readable, testable form.

\textbf{Core Components:}

\begin{itemize}[leftmargin=*]
\item \textbf{\texttt{/protocol/}} --- AEIP v1 handshake flows and a local governance ledger for trusted evidence capture.
\item \textbf{\texttt{/schemas/}} --- Machine-readable definitions for:
\end{itemize}
  - Interpretive Trace Packages (\textbf{ITP})  
  - Decision Rationale Records (\textbf{DRR})  
  - Governance Disclosure Statements (\textbf{GDS})  
  - Oversight Action Memoranda (\textbf{OAM})  
  - Integrity Ledger Entries (\textbf{ILE})  
\begin{itemize}[leftmargin=*]
\item \textbf{\texttt{/src/layer1--8/}} --- Modular validators aligning with each stack layer to enforce design-time governance controls.
\item \textbf{\texttt{/tools/}} --- Artifact generation and validation utilities for deterministic governance rehearsals.
\item \textbf{\texttt{/tests/}} --- Conformance and handshake verification suites demonstrating AEIP compliance and ledger reproducibility.
\item \textbf{\texttt{/examples/}} --- Notebooks illustrating local reasoning workflows and ledger integration for offline governance rehearsals.
\end{itemize}

All components operate \textbf{offline-first} and are engineered to illustrate \textbf{design-time governance patterns}, not production network services.

---

\paragraph{\textbf{Current Canonical Documents}}

\begin{itemize}[leftmargin=*]
\item \texttt{source/AI\_OSI\_Stack\_v4\_Test\_Integrated.md}
\item \texttt{docs/AI\_OSI\_Protocol\_Spec.md}
\item \texttt{docs/AEIP\_Spec\_v1.md}
\item \texttt{AEIP\_Artifact\_Schema\_Templates.md}
\item \emph{Persona Architecture v2 (PDF)}
\item \emph{Epistemology by Design v1 (PDF)}
\end{itemize}

---

\paragraph{\textbf{Repository Structure}}

```text
.
├── docs/          # Specifications and layer documentation
├── examples/      # Demonstration notebooks
├── protocol/      # AEIP handshake and ledger implementation
├── schemas/       # Governance artifact schemas
├── src/           # Layer modules (L1--L8)
├── tests/         # Conformance suite
├── tools/         # CLI utilities for artifact management
├── ledger/        # Local integrity ledger
└── versions/      # Historical releases (non-normative)
```

\subsubsection*{Citation}
```
Madden, D. (2025). \emph{The AI OSI Stack: A Governance Blueprint for Scalable and Trusted AI (v4 Expanded).} Zenodo. DOI: [placeholder]
```

\subsubsection*{Project Status}
Maintained by a single independent researcher. Blueprint-complete reference implementation; open for study and replication.

\subsubsection*{License}
This work is distributed under the Creative Commons Attribution-NonCommercial-NoDerivatives 4.0 International License.

\subsubsection*{Acknowledgements}
\begin{quote}“Developed as an open architecture for accountable intelligence. Released for public stewardship and future standardization.”\end{quote}

\subsubsection*{Header & License}

Copyright \textcopyright{} 2025 Daniel P. Madden  
Licensed under CC BY-NC-ND 4.0  

\begin{quote}Normative Language Notice\end{quote}
\begin{quote}This specification uses normative language consistent with ISO/IEC 42010 and NIST conventions.\end{quote}
\begin{quote}\textquotedbl{}SHALL\textquotedbl{} denotes mandatory requirements, \textquotedbl{}SHOULD\textquotedbl{} denotes strong recommendations, and \textquotedbl{}MAY\textquotedbl{} denotes optional practices.\end{quote}
\begin{quote}Interpretations of this document must preserve authorial intent --- fidelity to layered accountability, epistemic integrity, and human dignity as design constraints.\end{quote}
\section{Architecture Overview}
\emph{Scope: Provides the seven-plus-layer rationale, ethical foundations, and integration roadmap drawing on the architecture overview and conceptual stack narrative.}
The AI OSI Stack defines a seven-layer governance architecture that binds technical controls, institutional stewardship, and evidentiary accountability. It functions as the canonical scaffold for the AI Epistemic Infrastructure Protocol (AEIP), which provides the transport layer that conveys reasoning states, decisions, and temporal attestations between personas, auditors, and custodial nodes. AEIP inherits the stack's normative guardrails --- reasoning fidelity, traceable accountability, and dignity-first commitments --- while supplying deterministic handshakes that keep each layer's artifacts synchronized.

The blueprint encompasses seven mandatory governance layers with an optional civic mandate precursor. Each layer exposes interfaces, controls, and evidence channels that align with AEIP packet classes. Layer boundaries are permeable to guarantee system cohesion. Power geometry recognizes that decision authority, data provenance, and execution logic concentrate differently across layers. The stack addresses vertical power flow (from infrastructure to user interaction) and horizontal influence (across partner ecosystems) to mitigate capture and concentration risks.

\subsection{Ethical Foundations}
\emph{Scope: Articulates the dignity, epistemic integrity, and stewardship pillars.}
\begin{enumerate}[leftmargin=*]
    \item \textbf{Dignity by Design} --- Personas, affect controls, and refusal logic maintain human primacy and avert manipulative behaviors.
    \item \textbf{Epistemic Integrity} --- Interpretive Trace Packages (ITP) and Decision Rationale Records (DRR) ensure reasoning is reconstructable and contestable.
    \item \textbf{Accountable Stewardship} --- Integrity Ledger Entries (ILE) and Governance Disclosure Statements (GDS) expose temporally sealed evidence to auditors and the public.
\end{enumerate}

\subsection{Layer Summary Table}
\emph{Scope: Presents objectives, interfaces, and artifacts derived from the architecture overview and governance map.}
\begin{table}[h]
    \centering
    \begin{tabular}{p{2.2cm}p{3.5cm}p{4.2cm}p{4.2cm}}
        \toprule
        \textbf{Layer} & \textbf{Core Objective} & \textbf{Canonical Interfaces} & \textbf{Primary Artifacts} \\
        \midrule
        L0 --- Civic Mandate & Establish social license, legislative authorization, and participatory compacts & Civic consultation portals, public comment ledgers & Civic directives, community compacts, oversight charters \\
        L1 --- Physical Substrate & Secure compute foundations, supply integrity & Environmental attestations, facility provenance & GDS, DRR, facility inspection ILEs \\
        L2 --- Data Stewardship & Provenance, consent, and epistemic hygiene & Data lineage ledgers, consent registries & ITP, DRR, provenance appendices \\
        L3 --- Model Development & Training rigor, evaluation transparency & Evaluation harness APIs, persona regression hooks & OAM, GDS, evaluation annexes \\
        L4 --- Instruction \& Control & Persona governance, affect safeguards & Persona briefs, refusal logic, AEIP intents & ITP, DRR, persona governance matrices \\
        L5 --- Reasoning Exchange & AEIP handshake execution & Signed reasoning packets, counter-signatures & ILE, OAM, AEIP transcripts \\
        L6 --- Deployment \& Integration & Runtime assurance, change control & Release manifests, incident playbooks & TRR, OAM, change approvals \\
        L7 --- Governance Publication & External transparency, civic accountability & Disclosure portals, audit submission queues & GDS, ILE, oversight council minutes \\
        L8 --- Policy (optional) & Civic participation overlays, human rights clauses & Civic directive gateways, federation registries & Civic directives, policy concordance statements \\
        \bottomrule
    \end{tabular}
    \caption{AI OSI Stack Layer Objectives, Interfaces, and Artifacts}
\end{table}

\subsection{Offline-First Blueprint Scope}
\emph{Scope: Integrates implementation notes emphasizing offline execution and custodial governance.}
This reference implementation is intentionally offline-first. All utilities, schemas, and example ledgers SHALL operate without network dependencies to guarantee reproducible review, audit rehearsals, and archival resilience. Deployment, live registries, and production key material are out of scope. Stakeholders SHALL treat this repository as a normative blueprint: design decisions are finalized, but instantiation in live infrastructure remains deferred until governance coalitions ratify operational requirements. Organizations adopting the AI OSI Stack SHALL first internalize the seven-layer obligations, map existing controls to artifact expectations, and stage AEIP nodes in sandbox form. Only after governance councils approve contextual adaptations SHOULD production roll-out proceed. Continuous maturation follows the Implementation Maturity Model, ensuring every escalation preserves the ethical foundations summarized above.

\subsection{Architecture Overview Source Text}
\emph{Scope: Embeds the full architecture overview document for archival completeness.}
\textcopyright{} 2025 Daniel P. Madden  
\textbf{License:} CC BY-NC-ND 4.0

\subsubsection*{AI OSI Stack --- Architecture Overview}
\textbf{Author:} Daniel P. Madden  
\textbf{Version:} v4 -- Blueprint Integration  
\textbf{Date:} November 2025

\begin{quote}Normative Language Notice\end{quote}
\begin{quote}This document uses normative language consistent with ISO/IEC 42010 and NIST conventions.\end{quote}
\begin{quote}“SHALL” denotes mandatory requirements, “SHOULD” denotes strong recommendations, and “MAY” denotes optional practices.\end{quote}
\begin{quote}Interpretations SHALL preserve authorial intent: layered accountability, epistemic integrity, and human dignity as binding design constraints.\end{quote}

\paragraph{1. Purpose and Relationship to AEIP}
The AI OSI Stack defines a seven-layer governance architecture that binds technical controls, institutional stewardship, and evidentiary accountability. It functions as the canonical scaffold for the \textbf{AI Epistemic Infrastructure Protocol (AEIP)}, which provides the transport layer that conveys reasoning states, decisions, and temporal attestations between personas, auditors, and custodial nodes. AEIP inherits the stack’s normative guardrails---reasoning fidelity, traceable accountability, and dignity-first commitments---while supplying deterministic handshakes that keep each layer’s artifacts synchronized.

\paragraph{2. Seven-Layer Design Summary}
The blueprint encompasses seven mandatory governance layers (with an optional civic mandate precursor). Each layer exposes interfaces, controls, and evidence channels that align with AEIP packet classes.

| Layer | Core Objective | Canonical Interfaces | Primary Artifacts |
| --- | --- | --- | --- |
| L1 -- Physical Substrate | Secure compute foundations, supply integrity | Environmental attestations, facility provenance | GDS, DRR |
| L2 -- Data Stewardship | Provenance, consent, and epistemic hygiene | Data lineage ledgers, consent registries | ITP, DRR |
| L3 -- Model Development | Training rigor, evaluation transparency | Evaluation harness APIs, persona regression hooks | OAM, GDS |
| L4 -- Instruction & Control | Persona governance, affect safeguards | Persona briefs, refusal logic, AEIP intents | ITP, DRR, GDS |
| L5 -- Reasoning Exchange | AEIP handshake execution | Signed reasoning packets, counter-signatures | ILE, OAM |
| L6 -- Deployment & Integration | Runtime assurance, change control | Release manifests, incident playbooks | TRR, OAM |
| L7 -- Governance Publication | External transparency, civic accountability | Disclosure portals, audit submission queues | GDS, ILE |

\paragraph{3. Ethical Foundations}
The stack’s structure codifies Daniel P. Madden’s ethical triad:
\begin{enumerate}[leftmargin=*]
\item \textbf{Dignity by Design} --- Personas, affect controls, and refusal logic maintain human primacy and avert manipulative behaviors.
\item \textbf{Epistemic Integrity} --- Interpretive Trace Packages (ITP) and Decision Rationale Records (DRR) ensure reasoning is reconstructable and contestable.
\item \textbf{Accountable Stewardship} --- Integrity Ledger Entries (ILE) and Governance Disclosure Statements (GDS) expose temporally sealed evidence to auditors and the public.
\end{enumerate}

\paragraph{4. Offline-First Blueprint Scope}
This reference implementation is intentionally \textbf{offline-first}. All utilities, schemas, and example ledgers SHALL operate without network dependencies to guarantee reproducible review, audit rehearsals, and archival resilience. Deployment, live registries, and production key material are \textbf{out of scope}. Stakeholders SHALL treat this repository as a normative blueprint: design decisions are finalized, but instantiation in live infrastructure remains deferred until governance coalitions ratify operational requirements.

\paragraph{5. Integration Roadmap Considerations}
Organizations adopting the AI OSI Stack SHALL first internalize the seven-layer obligations, map existing controls to artifact expectations, and stage AEIP nodes in sandbox form. Only after governance councils approve contextual adaptations SHOULD production roll-out proceed. Continuous maturation follows the Implementation Maturity Model described in the canonical specification, ensuring every escalation preserves the ethical foundations summarized above.

\section{Detailed Layer Expositions (L0--L8)}
\emph{Scope: Expands each layer with risks, obligations, artifact flow, and normative instructions drawing on the integrated stack, risk taxonomy, and governance map.}

\subsection{Layer 0 --- Civic Mandate (Optional)}
\emph{Scope: Codifies legitimacy preconditions and civic obligations.}
Layer~0 establishes jurisdictional legitimacy and social license. It captures legislative authorization, community compacts, and public trust obligations. Governance councils SHALL document civic consultations, equity considerations, and participatory design sessions as Governance Disclosure Statements that frame the entire stack. Risks include legitimacy erosion and misaligned civic expectations. Controls include public comment ledgers, civic directive registries, and social impact attestations. Required artifacts encompass civic directives, oversight charters, and cross-community compacts recorded as Integrity Ledger Entries.

\subsection{Layer 1 --- Physical and Compute Substrate}
\emph{Scope: Details infrastructure risks, controls, and evidence requirements.}
Layer~1 covers data centers, chip governance, and reliability engineering aligned with ISO/IEC~27001 controls. Monitoring includes resilience audits and energy accountability. Risk themes encompass hardware tampering, supply chain disruption, and resilience degradation. Sentinel indicators include facility access anomalies, unscheduled firmware changes, and energy draw deviations. Controls include certified facility audits, redundant energy governance, and tamper-evident custody logs. Required evidence includes GDS Section~L1 narratives, DRR infrastructure appendices, and Integrity Ledger Entries containing temporal seals for facility inspections. Custodians SHALL ensure that all physical envelopes record `temporalSeal`, `personaSignature`, and `dignityCompliance` metadata prior to integration with higher layers.

\subsection{Layer 2 --- Data Stewardship}
\emph{Scope: Addresses provenance, consent, and epistemic hygiene.}
Layer~2 ensures provenance, consent, and representational justice. Epistemology by Design applies here through provenance ledgers and reasoning lineage documentation. Risk themes include data poisoning, consent erosion, and epistemic contamination. Sentinel indicators cover provenance gaps, unexplained distribution shifts, and contested consent revocations. Controls consist of consent traceability ledgers, dataset review boards, and epistemic hygiene playbooks. Required evidence includes Interpretive Trace Package provenance bundles, Decision Rationale Record data stewardship summaries, and AEIP lineage signatures. AEIP SHALL bind dataset commitments through `Justify` payloads that contain dataset hashes, licensing clauses, and dignity attestations.

\subsection{Layer 3 --- Model Development}
\emph{Scope: Summarizes training governance, evaluation rigor, and stewardship obligations.}
Layer~3 governs training protocols, evaluation harnesses, and red-teaming. It includes model cards, risk statements, and fairness reviews. Risk themes include bias amplification, evaluation blind spots, and adversarial regressions. Sentinel indicators include diverging fairness metrics, untested safety scenarios, and failed regression thresholds. Controls cover composite evaluation harnesses, adversarial benchmarking cadences, and persona-aligned regression suites. Required evidence includes Oversight Action Memorandum evaluation actions, Governance Disclosure Statement validation annexes, and Decision Rationale Record design decisions. Organizations SHALL conduct annual ``governance fire drills'' that simulate compound failures and document remediation quality, time-to-response, and lessons learned through AEIP transcripts.

\subsection{Layer 4 --- Instruction and Control}
\emph{Scope: Integrates Persona Architecture mandates, affect safeguards, and refusal logic.}
Layer~4 delivers structured roles, refusal logic, and affect management to ensure the AI acts within declared mandates. Persona Architecture requires agents to declare identity and intent at session initiation. Risk themes include prompt injection, persona drift, affect misalignment, and refusal degradation. Sentinel indicators include persona mismatch alerts, refusal override logs, and affect boundary excursions. Controls include Persona mandate enforcement, refusal logic penetration testing, and affect boundary audits. Required evidence includes Interpretive Trace Package instruction traces, Decision Rationale Record persona matrices, and Governance Disclosure Statement instruction governance statements. Dignity operates as a non-negotiable constraint: violations trigger automatic Oversight Action Memoranda and public disclosure through Layer~7 portals.

\subsection{Layer 5 --- Reasoning Exchange and Interface}
\emph{Scope: Describes AEIP handshake obligations, ledger synchronization, and failure modes.}
Layer~5 enforces structured dialogue between AI services, auditors, and human supervisors using signed reasoning packets and integrity fingerprints. Risk themes include ledger desynchronization, packet replay, counter-signature gaps, and handshake tampering. Sentinel indicators include hash mismatches, delayed acknowledgements, and anomalous replay counts. Controls include AEIP handshake validation, deterministic replay checkpoints, and cross-node quorum review. Required evidence includes Integrity Ledger Entry handshake bundles, Oversight Action Memorandum remediation logs, and AEIP transcript archives. Each AEIP step MUST include provenance fields and persona signatures covering canonical payloads and headers. Hash drift triggers `LedgerVerificationError` events within governance nodes and SHALL result in immediate remediation via Oversight Action Memoranda.

\subsection{Layer 6 --- Deployment and Integration}
\emph{Scope: Covers runtime assurance, change control, and incident governance.}
Layer~6 manages runtime environments, incident response, and change control across organizational ecosystems. Risk themes include runtime drift, integration conflicts, and incident opacity. Sentinel indicators include unauthorized configuration changes, failed deployment gates, and unreported incidents. Controls include change gating with governance approvals, blue/green stewardship protocols, and incident rehearsal cadences. Required evidence includes Temporal Review Record release attestations, Oversight Action Memorandum incident records, and Integrity Ledger Entry change approvals. Governance councils SHALL maintain change ledgers, ensure training for human overseers, and coordinate with regulators to synchronize compliance submissions.

\subsection{Layer 7 --- Governance Publication}
\emph{Scope: Mandates disclosure practices, oversight transparency, and civic accountability.}
Layer~7 releases Governance Disclosure Statement packages, Integrity Ledger Entry updates, and public accountability artifacts ensuring transparency, traceability, and trust. Risk themes include delayed disclosure, accountability gaps, and legitimacy erosion. Sentinel indicators include missed publication cadences, stakeholder complaints, and inconsistent disclosure metadata. Controls include public disclosure schedules, multi-stakeholder review councils, and archival notarization. Required evidence includes Governance Disclosure Statement transparency packs, Integrity Ledger Entry publication proofs, and custodial meeting minutes. Organizations SHALL publish Implementation Maturity Model levels annually as Integrity Ledger Entries to maintain transparency and accountability.

\subsection{Layer 8 --- Policy and Civic Participation}
\emph{Scope: Extends governance into participatory oversight and human rights alignment.}
Layer~8 provides civic participation overlays and human rights clauses. It connects governance publication outputs to regulatory submissions, public forums, and federated registries. Persona-PKI, Open Governance Registry, and RegOps Bridge expansion plans SHALL inform Layer~8 implementations. Civic directives SHALL document how governance evidence is interpreted by regulators, civil society coalitions, and participatory oversight structures. Custodians SHALL coordinate with civic councils to ensure policy updates remain grounded in the stack's normative commitments.

\subsection{Test Integration Narrative Supplement}
\emph{Scope: Reproduces the full Test Integration Draft within the Master Edition for archival fidelity.}
\textcopyright{} 2025 Daniel P. Madden
Licensed under Creative Commons Attribution-NonCommercial-NoDerivatives 4.0 International (CC BY-NC-ND 4.0)

\begin{quote}Normative Language Notice\end{quote}
\begin{quote}This specification uses normative language consistent with ISO/IEC 42010 and NIST conventions.\end{quote}
\begin{quote}“SHALL” denotes mandatory requirements, “SHOULD” denotes strong recommendations, and “MAY” denotes optional practices.\end{quote}
\begin{quote}Interpretations of this document must preserve authorial intent --- fidelity to layered accountability, epistemic integrity, and human dignity as design constraints.\end{quote}

\subsubsection*{The AI OSI Stack v4 --- Test Integration Draft}
\paragraph{A Governance Blueprint for Scalable and Trusted AI}
\textbf{Author:} Daniel P. Madden, Independent AI Researcher  
\textbf{Date:} November 2025  
\textbf{Version:} Version 4.0-Test (Supersedes all previous AI OSI Stack versions in this repository.)

\begin{quote}\textbf{NOTE:} This v4 integrates: maturity modeling, risk catalogue, temporal legitimacy, affective constraints, cognitive diversity, distributed oversight, sectoral extensions, and AEIP alignment.\end{quote}

\paragraph{Table of Contents}
\begin{enumerate}[leftmargin=*]
\item Introduction and Purpose
\item Normative Definitions and Acronyms
\item Architectural Foundations and Methodology
\item Layer Interdependence and Power Geometry
\item Layer-by-Layer Framework
\item Governance Artifacts and Audit Infrastructure
\item Temporal Integrity and Semantic Version Control
\item Global Framework Interoperability Map
\item Adaptive Governance and Foresight
\item Implementation and Adoption Pathways
\item Custodianship and Change Management
\item Implementation Maturity Model (IMM)
\item Risk Taxonomy and Control Catalogue
\item Temporal Legitimacy Framework
\item Adaptive Governance Metrics (AGM)
\item Human Dignity and Affective Constraint Module
\item Cognitive Diversity Index (CDI)
\item Governance Simulation and Stress Testing
\item Distributed Oversight Architecture
\item Ethical AI Supply Chain Framework
\item Human-in-the-Governance Loop
\item Societal Foresight and Temporal Pluralism
\item Open Implementation Registry (OIR)
\item Decommissioning and End-of-Life Governance
\end{enumerate}

Appendices:
\begin{itemize}[leftmargin=*]
\item Appendix A. Glossary
\item Appendix B. Layer-to-Standard Concordance
\item Appendix C. Version Change Log
\item Appendix D. References
\end{itemize}

\paragraph{1. Introduction and Purpose}
The AI OSI Stack v4 consolidates Daniel P. Madden’s governance architecture into a single canonical reference that aligns operational controls, epistemic integrity, and human dignity across the entire lifecycle of intelligent systems. This Test Integration Draft resolves fragmentation from earlier versions by embedding Persona Architecture, Epistemology by Design, and the AI Epistemic Infrastructure Protocol (AEIP) directly within the layered stack. The intent is to provide institutions with a deployable governance blueprint compatible with NIST AI RMF, ISO/IEC 42001, the EU AI Act, and the TRUST Framework while remaining adaptable to sector-specific constraints.

\subparagraph{Jurisdictional Neutrality and Interpretive Authority}
This framework is jurisdiction-neutral and intended for international application.  
In the event of conflicting legal or linguistic interpretations, interpretive authority SHALL defer  
to the author’s defined principles of transparency, accountability, and dignity by design ---  
not to localized or translated semantics.

\paragraph{2. Normative Definitions and Acronyms}
This document adopts normative language consistent with international governance standards. “SHALL” denotes mandatory requirements; “SHOULD” indicates strong recommendations; “MAY” introduces optional practices. Key acronyms include: AEIP (AI Epistemic Infrastructure Protocol), AGM (Adaptive Governance Metrics), CDI (Cognitive Diversity Index), GDS (Governance Disclosure Statement), ILE (Integrity Ledger Entry), IMM (Implementation Maturity Model), OAM (Oversight Action Memorandum), OIR (Open Implementation Registry), and Persona Architecture (PA). The lexicon harmonizes with prior publications such as \emph{Persona Architecture: Designing Role-Specific AI Systems for Accountability and Trust} and \emph{Epistemology by Design: Embedding Reasoning Integrity in AI Systems}.

\paragraph{3. Architectural Foundations and Methodology}
The stack is engineered as infrastructure rather than policy prose. Its methodology interleaves architectural decomposition with governance artifacts that provide audit-ready evidence. Layered controls are validated via AEIP-compliant exchanges, ensuring that reasoning states, decision artifacts, and socio-technical context remain inseparable. The methodology favors design patterns that instantiate accountability at system inception, echoing the stance articulated in \emph{The AI OSI Stack: A Governance Blueprint for Scalable and Trusted AI}. Each layer is treated as a contract surface that binds risk controls, cognitive safeguards, and affective constraints.

\paragraph{4. Layer Interdependence and Power Geometry}
Layer boundaries are permeable to guarantee system cohesion. Power geometry recognizes that decision authority, data provenance, and execution logic concentrate differently across layers. The stack addresses vertical power flow (from infrastructure to user interaction) and horizontal influence (across partner ecosystems) to mitigate capture and concentration risks. Persona Architecture anchors Layer 4 by establishing accountable instruction sets, while AEIP operationalizes Layer 5 as the protocol spine that transmits epistemic commitments. Governance artifacts at Layer 7 distribute power by exposing stewardship evidence to external auditors and public stakeholders.

\paragraph{5. Layer-by-Layer Framework}
\begin{enumerate}[leftmargin=*]
\item \textbf{Layer 0 -- Civic Mandate (Optional):} Establishes jurisdictional legitimacy and social license. It captures legislative authorization, community compacts, and public trust obligations.
\item \textbf{Layer 1 -- Physical and Compute Substrate:} Covers data centers, chip governance, and reliability engineering aligned with ISO/IEC 27001 controls. Monitoring includes resilience audits and energy accountability.
\item \textbf{Layer 2 -- Data Stewardship:} Ensures provenance, consent, and representational justice. Epistemology by Design applies here through provenance ledgers and reasoning lineage documentation.
\item \textbf{Layer 3 -- Model Development:} Governs training protocols, evaluation harnesses, and red-teaming. Includes model cards, risk statements, and fairness reviews.
\item \textbf{Layer 4 -- Instruction and Control:} Persona Architecture delivers structured roles, refusal logic, and affect management to ensure the AI acts within declared mandates.
\item \textbf{Layer 5 -- Reasoning Exchange and Interface:} AEIP enforces structured dialogue between AI services, auditors, and human supervisors using signed reasoning packets and integrity fingerprints.
\item \textbf{Layer 6 -- Deployment and Integration:} Covers runtime environments, incident response, and change control across organizational ecosystems.
\item \textbf{Layer 7 -- Governance Publication:} Releases GDS packages, ILE updates, and public accountability artifacts ensuring transparency, traceability, and trust.
\end{enumerate}

\paragraph{6. Governance Artifacts and Audit Infrastructure}
The stack relies on reusable artifacts: Governance Disclosure Statements (GDS) synthesize cross-layer obligations; Integrity Ledger Entries (ILE) notarize decisions and maturity attestations; Oversight Action Memoranda (OAM) document interventions and remediations; Temporal Review Records (TRR) capture version checks; Interpretive Trace Packages (ITP) reconstruct evidentiary reasoning; and Decision Rationale Records (DRR) preserve architectural justification. AEIP provides the transport for these artifacts, guaranteeing interoperable audit traces. Artifacts SHALL be version-controlled, cryptographically signed, and exposed through Layer 7 portals. Canonical JSON-LD definitions are maintained in the \texttt{schemas/} directory (AEIP-Schema-1.0-2025-11), with validation exemplars residing in \texttt{tests/aeip\_schema\_validation.py}.

\paragraph{7. Temporal Integrity and Semantic Version Control}
Governance is inherently temporal. Every change to Persona Architecture configurations, epistemic safeguards, or AEIP schemas requires semantic versioning anchored to custodial review. Temporal integrity is sustained through synchronized clocks, validity windows for each artifact, and automated drift alerts triggered by reasoning fingerprint deviation. Semantic version control couples model updates with governance obligations to prevent silent regression.

\paragraph{8. Global Framework Interoperability Map}
The stack crosswalks to prevailing frameworks: Layer 2 aligns with NIST AI RMF “Data and Data Governance”; Layer 3 maps to ISO/IEC 42001 operational controls; Layer 4 satisfies EU AI Act transparency and instruction mandates; Layer 5 implements TRUST Framework interoperability principles. The interoperability map establishes translation guides so that organizations can certify once and comply many times, minimizing audit friction while preserving rigor.

\paragraph{9. Adaptive Governance and Foresight}
Adaptive governance assumes AI systems encounter dynamic contexts. Scenario planning, foresight scanning, and socio-technical sensing are embedded at Layer 6 and Layer 7. Persona Architecture supplies the adjustable knobs, AEIP carries foresight signals, and Governance Councils interpret the signals to recalibrate controls. The stack mandates quarterly foresight reviews incorporating emergent risks, policy shifts, and cultural feedback loops.

\paragraph{10. Implementation and Adoption Pathways}
Organizations SHALL follow a staged adoption: inventory existing systems, map them to the stack, implement Persona Architecture for instruction surfaces, deploy AEIP nodes, and publish the first GDS. Sectoral extensions---such as GERDY v1 for education or ARCHY v1 for healthcare---serve as reference templates. Adoption pathways emphasize co-design with affected communities, aligning with the dignity-first ethos introduced in \emph{Persona Architecture}.

\paragraph{11. Custodianship and Change Management}
Custodianship rests with a designated Governance Council empowered to approve changes, review audits, and steward public accountability. Change management integrates AEIP change requests, DRR updates, and OAM tracking. Custodians SHALL maintain a change ledger, ensure training for human overseers, and coordinate with regulators to synchronize compliance submissions. Custodianship is designed to be portable across organizations while preserving Daniel P. Madden’s architectural intent.

\paragraph{12. Implementation Maturity Model (IMM)}
The IMM defines six discrete levels of governance capability:
\begin{itemize}[leftmargin=*]
\item \textbf{Level 0 -- No Governance:} Ad hoc AI usage without documented controls. Required artifacts: none. Expected audit frequency: incident-driven only.
\item \textbf{Level 1 -- Basic / Reactive Governance:} Minimal policies and reactive incident handling. Required artifacts: baseline GDS, initial DRR. Expected audit frequency: annual self-assessment.
\item \textbf{Level 2 -- Integrated Layer Controls:} Stack-aligned controls implemented across Layers 1--4 with Persona Architecture instantiated. Required artifacts: Persona briefs, AEIP readiness report, TRR baseline. Expected audit frequency: semiannual internal audits.
\item \textbf{Level 3 -- Adaptive / Temporal Governance:} Temporal integrity tooling, drift monitoring, and AEIP ledger integration across Layers 1--6. Required artifacts: temporal audit log, OAM register, IMM attestation. Expected audit frequency: quarterly governance reviews.
\item \textbf{Level 4 -- Trust Infrastructure:} Full AEIP deployment, public Layer 7 disclosures, and cross-jurisdictional interoperability. Required artifacts: comprehensive GDS, signed ILE packages, resilience playbooks. Expected audit frequency: continuous monitoring with biannual external audits.
\item \textbf{Level 5 -- Cognitive Stewardship:} Cognitive diversity orchestration, affective constraint modules, and inter-organizational oversight. Required artifacts: CDI reports, dignity compliance attestations, governance simulation results. Expected audit frequency: continuous AEIP validation with annual third-party certification.
\end{itemize}

Organizations SHALL publish their IMM level annually as an Integrity Ledger Entry (ILE) to maintain transparency and accountability.

\paragraph{13. Risk Taxonomy and Control Catalogue}
Each layer carries characteristic risks and prescribed controls:
\begin{itemize}[leftmargin=*]
\item \textbf{Layer 1:} Risks---hardware tampering, supply disruptions. Controls---certified facilities audits, redundant energy governance. Artifacts---GDS, DRR.
\item \textbf{Layer 2:} Risks---data poisoning, consent erosion. Controls---provenance validation, consent traceability. Artifacts---ITP (Interpretive Trace Package), DRR.
\item \textbf{Layer 3:} Risks---model bias, evaluation blind spots. Controls---composite testing, adversarial benchmarking. Artifacts---OAM, GDS.
\item \textbf{Layer 4:} Risks---prompt injection, persona drift, unethical affect. Controls---Heartwood Core enforcement, refusal logic tests, persona regression suite. Artifacts---ITP, DRR, GDS.
\item \textbf{Layer 5:} Risks---ledger desynchronization, reasoning packet corruption. Controls---AEIP handshake validation, cross-node replay defenses. Artifacts---ILE, OAM.
\item \textbf{Layer 6:} Risks---deployment drift, integration conflicts. Controls---change gating, blue/green governance checkpoints. Artifacts---TRR, OAM.
\item \textbf{Layer 7:} Risks---opaque reporting, delayed disclosures. Controls---public GDS cadence, audit-ready publishing pipeline. Artifacts---GDS, ILE.
\end{itemize}

Controls SHALL be catalogued with traceable references to the artifact classes: Interpretive Trace Package (ITP), Decision Rationale Record (DRR), Governance Decision Summary (GDS), Oversight Audit Memo (OAM), Integrity Ledger Entry (ILE).

\paragraph{14. Temporal Legitimacy Framework}
Temporal legitimacy binds authority to time. Each governance artifact receives a defined validity window during which its claims remain authoritative. Temporal amnesty clauses allow constrained grace periods to remediate violations without punitive escalation when self-disclosed promptly. Drift thresholds are codified: a 5\% change in reasoning fingerprint triggers mandatory review and potential rollback. Quarterly temporal audits evaluate alignment between declared versions and deployed systems. All temporal legitimacy events SHALL be recorded as AEIP-compatible ledger entries to guarantee verifiable lineage.

\paragraph{15. Adaptive Governance Metrics (AGM)}
Adaptive metrics provide continuous insight:
\begin{itemize}[leftmargin=*]
\item \textbf{Transparency Ratio (TR):} Published governance artifacts ÷ required artifacts. Recorded in GDS.
\item \textbf{Governance Coverage Score (GCS):} Implemented controls ÷ identified controls for the operational scope. Logged alongside IMM attestations.
\item \textbf{Drift Index (DI):} Aggregated delta of performance and reasoning fingerprints normalized over time. Stored in TRR and linked via AEIP.
\item \textbf{Dignity Compliance Rate (DCR):} Verified adherence to affective constraints ÷ total evaluated interactions. Captured within GDS and ILE updates.
\end{itemize}

Organizations SHALL report AGM values quarterly and expose them through Layer 7 dashboards for stakeholder review.

\paragraph{16. Human Dignity and Affective Constraint Module}
Dignity operates as a non-negotiable constraint. Persona Architecture requires agents to declare identity and intent at session initiation, ensuring clarity of role and accountability. The module prohibits intimacy simulation, manipulative sentiment scripts, and coercive affective cues. Violations trigger automatic Oversight Action Memoranda and public disclosure through Layer 7 portals. This preserves the “dignity as constraint” principle articulated across Daniel P. Madden’s corpus while providing enforceable levers for auditors.

\paragraph{17. Cognitive Diversity Index (CDI)}
The CDI measures epistemic plurality. Defined as CDI = (distinct epistemic families engaged) ÷ (total governance decisions), it encourages inclusive stewardship. Inspired by the cognitive scaffolding demonstrated in CASSI v1, organizations target a CDI of 0.4 or higher for high-stakes domains. CDI reporting SHALL accompany IMM and AGM disclosures, enabling auditors to detect monocultures that jeopardize resilience.

\paragraph{18. Governance Simulation and Stress Testing}
Annual “governance fire drills” validate readiness for compound failures. Simulations incorporate synthetic data breaches, sudden policy shifts, and model drift scenarios. Outputs are consolidated into an OAM-Resilience Report that documents response quality, time-to-remediation, and lessons learned. AEIP facilitates replay and external verification, ensuring simulations translate into tangible improvements.

\paragraph{19. Distributed Oversight Architecture}
Distributed oversight balances local autonomy with global assurance. The architecture comprises three tiers: Local Ledger Nodes maintain operational records; Regional Custodians aggregate sector or jurisdiction data; and a Global Trust Anchor synchronizes standards without centralizing control. AEIP secures inter-tier exchanges, enabling cross-jurisdictional audits and mutual recognition agreements.

\paragraph{20. Ethical AI Supply Chain Framework}
Supply chain governance requires all vendors to publish Stack-aligned Governance Cards detailing Persona Architecture conformance, epistemic safeguards, and AEIP integration status. Procurement processes SHALL mandate Stack-Aligned Reports assessing each supplier’s IMM level. All supply chain attestations are logged via ILE to preserve accountability across the lifecycle.

\paragraph{21. Human-in-the-Governance Loop}
Human stewards remain integral. Mandatory checkpoints occur during training data approval, instruction design, deployment readiness, and audit sign-off. Escalation paths route unresolved issues to the Governance Council, which can issue OAM directives or suspend operations. AEIP records human approvals to ensure traceability and to prevent automation from eroding oversight.

\paragraph{22. Societal Foresight and Temporal Pluralism}
The stack embeds temporal pluralism to account for divergent futures. Scenario planning explores three anchor futures for 2030--2040: \textbf{Fragmented Governance}, where jurisdictions diverge; \textbf{Unified AEIP}, where interoperability prevails; and \textbf{AI--Human Co-Governance}, where shared stewardship emerges. Inter-temporal arbitration mechanisms reconcile decisions that impact multiple timelines, ensuring present-day actions remain legitimate across futures.

\paragraph{23. Open Implementation Registry (OIR)}
The OIR is a public or semi-public ledger enumerating adopters of the AI OSI Stack. Each entry records organization name, IMM level, adoption date, and referenced artifacts. Entries SHALL link to published GDS packages and ILE hashes, enabling external verification while protecting sensitive details through selective disclosure.

\paragraph{24. Decommissioning and End-of-Life Governance}
Retiring AI systems requires deliberate closure. Organizations SHALL produce a Final GDS summarizing operational history, issue a Closure ILE indicating shutdown status, perform memory redaction consistent with privacy commitments, and publish a public “AI epitaph” explaining lessons learned. Ethical end-of-life practices honor affected communities and prevent orphaned systems from resurfacing without governance.

\paragraph{Appendix A. Glossary}
\begin{itemize}[leftmargin=*]
\item \textbf{AEIP:} Protocol for exchanging reasoning artifacts with audit guarantees. Detailed in \emph{AEIP-00: The AI Epistemic Infrastructure Protocol}.
\item \textbf{Governance Disclosure Statement (GDS):} Public artifact describing compliance posture.
\item \textbf{Integrity Ledger Entry (ILE):} Signed record anchoring governance claims in time.
\item \textbf{Oversight Action Memorandum (OAM):} Formal notice of intervention or remediation.
\item \textbf{Persona Architecture (PA):} Role-specific design methodology ensuring accountable AI personas.
\end{itemize}

\paragraph{Appendix B. Layer-to-Standard Concordance}
\begin{itemize}[leftmargin=*]
\item \textbf{Layer 1:} ISO/IEC 27001, NIST CSF.
\item \textbf{Layer 2:} NIST AI RMF “Data” function, GDPR Articles 13--22.
\item \textbf{Layer 3:} ISO/IEC 42001 Clause 8, NIST AI RMF “Measure”.
\item \textbf{Layer 4:} EU AI Act transparency requirements, Persona Architecture guidelines.
\item \textbf{Layer 5:} TRUST Framework interoperability principles, AEIP handshake specifications.
\item \textbf{Layer 6:} ITIL Change Management, NIST SP 800-53 IR controls.
\item \textbf{Layer 7:} Open Government data publication norms, public accountability statutes.
\end{itemize}

\paragraph{Appendix C. Version Change Log}
\begin{itemize}[leftmargin=*]
\item \textbf{v4.0-Test (November 2025):} Integrated maturity model, risk catalogue, temporal legitimacy, affective constraints, cognitive diversity, distributed oversight, supply chain governance, and new registries. Supersedes v1--v3 and prior drafts.
\end{itemize}

\paragraph{Appendix D. References}
\begin{enumerate}[leftmargin=*]
\item Daniel P. Madden, \emph{Persona Architecture: Designing Role-Specific AI Systems for Accountability and Trust}.
\item Daniel P. Madden, \emph{Epistemology by Design: Embedding Reasoning Integrity in AI Systems}.
\item Daniel P. Madden, \emph{The AI OSI Stack: A Governance Blueprint for Scalable and Trusted AI}.
\item Daniel P. Madden, \emph{AEIP-00: The AI Epistemic Infrastructure Protocol}.
\item NIST, \emph{AI Risk Management Framework} (2023).
\item ISO/IEC, \emph{42001:2023 Artificial Intelligence Management System}.
\item European Union, \emph{AI Act} (2024 provisional agreement).
\item U.S. Department of Commerce, \emph{TRUST Framework} (2024).
\end{enumerate}

\subparagraph{Custodianship and Authorship}
Daniel P. Madden retains moral and intellectual authorship of this framework.  
This work SHALL NOT be modified, translated, or reissued under altered terminology without written consent.  
Derived works or alternative semantic renderings that could misrepresent intent SHALL be considered  
non-conformant and unauthorized under the CC BY-NC-ND 4.0 License.

\section{AEIP v1 Transport Specification}
\emph{Scope: Integrates the AEIP v1 specification, protocol reference, and handshake governance logic.}
AEIP~v1 provides a deterministic, persona-signed negotiation channel that links reasoning traces to ledger-grade artifacts. The protocol is intentionally human-auditable and uses SHA3-512 hashes plus Ed25519-compatible signatures. The transport spine SHALL implement the five-step handshake `Intent \rightarrow Justify \rightarrow CounterSign \rightarrow Commit \rightarrow Update`. Headers MUST include `aeipVersion`, `temporalSeal`, `personaSignature`, and `governanceScope`. Successful handshakes SHALL emit an Integrity Ledger Entry conforming to `schemas/ile_schema.json`.

\subsection{Message Headers}
\emph{Scope: Enumerates canonical header fields.}
\begin{table}[h]
    \centering
    \begin{tabular}{p{3cm}p{7cm}p{2cm}}
        \toprule
        \textbf{Field} & \textbf{Description} & \textbf{Mandatory} \\
        \midrule
        `aeipVersion` & Protocol version identifier. Current: `1.0`. & YES \\
        `temporalSeal` & RFC3339 timestamp concatenated with SHA3-512 digest. & YES \\
        `personaSignature` & Ed25519 placeholder signature (persona bound). & YES \\
        `governanceScope` & Governance namespace (e.g., `demo`, `production`). & YES \\
        `dignityCompliance` & Boolean flag ensuring Persona guardrails. & YES \\
        `supportedVersions` & Array declaring optional downgrade paths during Intent. & SHOULD \\
        `x-*` extensions & Additional headers namespaced for extensibility. & MAY \\
        \bottomrule
    \end{tabular}
    \caption{AEIP v1 Header Fields}
\end{table}

\subsection{Handshake Steps}
\emph{Scope: Details the Intent-to-Update progression and obligations.}
\begin{enumerate}[leftmargin=*]
    \item \textbf{Intent} --- Initiator declares objective and attaches Instructional Task Plan (ITP).
    \item \textbf{Justify} --- Counterparty provides Decision Rationale Record elements and confirms constraints.
    \item \textbf{CounterSign} --- Initiator re-confirms, signs, and extends provenance metadata.
    \item \textbf{Commit} --- Counterparty finalizes obligations and creates Governance Directive Set (GDS).
    \item \textbf{Update} --- Initiator issues Oversight Assurance Metrics snapshot and requests ledger entry.
\end{enumerate}
Each step MUST include provenance fields and a persona signature covering the canonical payload plus headers. Missing or invalid signatures SHALL abort the handshake. `dignityCompliance=false` SHALL propagate as a refusal and MUST block ledger submission. Hash drift triggers `LedgerVerificationError` within the governance node. Additional headers MUST be namespaced under `x-` prefixes. Alternative persona signature algorithms SHALL declare their algorithm name in the signature object. Version negotiation occurs during Intent by including a `supportedVersions` array in the payload. Temporal governance SHALL be enforced by `temporalSeal` issuance per step; drift detection is achieved by comparing handshake timestamps with ledger arrival time.

\begin{figure}[h]
    \centering
    \begin{tikzpicture}[node distance=2.5cm]
        \tikzstyle{role}=[rectangle, draw, rounded corners, minimum width=3.5cm, minimum height=1cm, align=center]
        \node[role] (instr) {Instruction Node\\(Persona Aligned)};
        \node[role, right=5cm of instr] (gov) {Governance Node\\(Custodial Ledger)};
        \draw[->, thick] (instr) -- node[above]{Intent / ITP + Header Hash} (gov);
        \draw[<-, thick] (instr) -- ++(0,-1.5cm) node[midway, above]{Justify / DRR Snapshot} (gov |- instr);
        \draw[->, thick] (instr |- gov) -- node[above]{CounterSign / Persona Signature} (gov);
        \draw[<-, thick] (instr |- gov) -- ++(0,-1.5cm) node[midway, above]{Commit / GDS Bundle} (gov |- gov);
        \draw[->, thick] (instr |- gov |- instr) -- node[above]{Update / OAM Metrics} (gov |- gov |- instr);
        \node[below=5.3cm of instr, align=center] {All transmissions SHALL carry `temporalSeal`, `dignityCompliance`, and SHA3-512 digests.};
    \end{tikzpicture}
    \caption{AEIP v1 Handshake Flow between Persona and Governance Nodes}
\end{figure}

\subsection{Canonical Payload Example}
\emph{Scope: Provides a round-trippable JSON payload illustrating normative fields.}
\begin{verbatim}
{
  "aeipVersion": "1.0",
  "temporalSeal": "2025-11-30T16:42:51Z#4f12...",
  "personaSignature": {
    "algorithm": "Ed25519",
    "publicKey": "PERSONA-KEY-STUB",
    "signature": "cafe-feed-..."
  },
  "governanceScope": "production",
  "dignityCompliance": true,
  "intent": {
    "itpRef": "itp-2025-11-30-001",
    "objective": "Deploy Layer6 release candidate",
    "constraints": ["IMM>=3", "CDI>=0.4"],
    "supportedVersions": ["1.0", "1.1-draft"]
  },
  "justify": {
    "drrRef": "drr-2025-11-30-014",
    "riskStatement": "Layer6 change with blue/green fallback",
    "provenance": ["ile-2025-10-01-020"]
  },
  "counterSign": {
    "personaId": "governance.engineer",
    "hash": "b1b5...",
    "notes": "All controls satisfied"
  },
  "commit": {
    "gdsRef": "gds-2025-11-30-Release",
    "obligations": ["publish OAM", "update TRR"]
  },
  "update": {
    "oamRef": "oam-2025-11-30-incident-rehearsal",
    "metrics": {"TR": 0.92, "GCS": 0.87, "DI": 0.05, "DCR": 0.98}
  }
}
\end{verbatim}

\subsection{AEIP v1 Specification Source Text}
\emph{Scope: Embeds the canonical AEIP v1 specification.}
\subsubsection*{AEIP v1 Specification}

\textbf{Status:} Reference Implementation Alignment  
\textbf{Scope:} Transport handshake for reasoning integrity and artifact exchange

\paragraph{1. Protocol Summary}

AEIP v1 provides a deterministic, persona-signed negotiation channel that links reasoning traces to ledger-grade artifacts.  The protocol is intentionally human-auditable and uses SHA3-512 hashes plus Ed25519-compatible signatures.

\paragraph{2. Message Headers}

| Field | Description | Mandatory |
| --- | --- | --- |
| \texttt{aeipVersion} | Protocol version identifier. Current: \texttt{1.0}. | YES |
| \texttt{temporalSeal} | RFC3339 timestamp concatenated with SHA3-512 digest. | YES |
| \texttt{personaSignature} | Ed25519 placeholder signature (persona bound). | YES |
| \texttt{governanceScope} | Governance namespace (e.g., \texttt{demo}, \texttt{production}). | YES |
| \texttt{dignityCompliance} | Boolean flag ensuring Persona guardrails. | YES |

\paragraph{3. Handshake Steps}

\begin{enumerate}[leftmargin=*]
\item \textbf{Intent} -- Initiator declares objective and attaches Instructional Task Plan (ITP).
\item \textbf{Justify} -- Counterparty provides Decision Rationale Record (DRR) elements and confirms constraints.
\item \textbf{CounterSign} -- Initiator re-confirms, signs, and extends provenance metadata.
\item \textbf{Commit} -- Counterparty finalises obligations and creates Governance Directive Set (GDS).
\item \textbf{Update} -- Initiator issues Oversight Assurance Metrics (OAM) snapshot and requests ledger entry.
\end{enumerate}

Each step MUST include provenance fields and a persona signature covering the canonical payload plus headers.

\paragraph{4. Serialization Rules}

\begin{itemize}[leftmargin=*]
\item JSON is the canonical wire format.  YAML MAY be used for diagnostics and MUST round-trip to JSON without data loss.
\item JSON-LD contexts MAY be attached under \texttt{@context} for semantic interoperability.
\item Field ordering is irrelevant; canonicalization uses lexicographic JSON serialization prior to hashing.
\end{itemize}

\paragraph{5. Integrity Guarantees}

\begin{itemize}[leftmargin=*]
\item Hashing: SHA3-512 across canonical JSON payloads.
\item Signatures: Deterministic Ed25519 placeholder (ready for HSM swap).
\item Temporal Governance: \texttt{temporalSeal} is re-issued at every step and SHALL be monotonic.
\end{itemize}

\paragraph{6. Failure Modes}

\begin{itemize}[leftmargin=*]
\item Missing or invalid signatures SHALL abort the handshake.
\item \texttt{dignityCompliance=False} SHALL propagate as a refusal and MUST block ledger submission.
\item Hash drift triggers \texttt{LedgerVerificationError} within the governance node.
\end{itemize}

\paragraph{7. Extensibility}

\begin{itemize}[leftmargin=*]
\item Additional headers MUST be namespaced under \texttt{x-} prefixes.
\item Alternative persona signature algorithms SHALL declare their algorithm name in the signature object.
\item Version negotiation occurs during Intent by including \texttt{supportedVersions} array in the payload.
\end{itemize}

---

\emph{Refer to \texttt{/protocol/aeip\_handshake.py} and \texttt{/tests/test\_aeip\_handshake.py} for executable reference behaviour.}

\subsection{AI OSI Protocol Reference Source Text}
\emph{Scope: Integrates the AI OSI protocol reference document.}
\subsubsection*{AI OSI Protocol Reference (AI\_OSI-00)}

\textbf{Status:} Draft for Implementation Guidance  
\textbf{Editors:} AI Governance Reference Implementation Team  
\textbf{Related Normative Sources:} ISO/IEC 42001, NIST AI RMF, EU AI Act Recital 60, AI\_OSI\_Stack\_v4\_Test\_Integrated.md

---

\paragraph{1. Introduction}

The AI OSI protocol reference maps the conceptual AI OSI Stack v4 (Test Integrated) to a modular, testable implementation analogous to the TCP/IP family atop the OSI networking model.  Each layer exposes interoperable contracts, traceable dignity safeguards, and AEIP transport compatibility.  This document is the living RFC-style specification that SHALL evolve via governance ledger entries.

\paragraph{2. Layer Overview}

| Layer | Name | Key Responsibilities | Mandatory Artifacts |
| --- | --- | --- | --- |
| L1 | Physical | Telemetry normalization, energy stewardship, persona safety interlocks | Physical Envelope |
| L2 | Architecture | Persona topology, interface negotiation, capacity planning | Architecture Envelope |
| L3 | Training | Corpus governance, alignment specification | Training Spec |
| L4 | Instruction | Instruction packetization, AEIP handshake preparation | ITP, DRR |
| L5 | Interface | Header canonicalization, persona signature propagation | AEIP Headers |
| L6 | Application | Reasoning outputs, service responses | GDS, OAM |
| L7 | Governance | Oversight synthesis, ledger integration | ILE |
| L8 | Policy (opt) | Civic participation overlays, human rights clauses | Civic Directive |

All layers MUST expose a \texttt{dignity\_compliance} flag and SHALL refuse unsafe operations.  Provenance fields (\texttt{source}, \texttt{timestamp}, \texttt{personaId}, \texttt{hash}) are REQUIRED on every exchanged artifact.

\paragraph{3. Architectural Diagram}

```mermaid
graph TD
    L1[Layer 1\nPhysical] --> L2[Layer 2\nArchitecture]
    L2 --> L3[Layer 3\nTraining]
    L3 --> L4[Layer 4\nInstruction]
    L4 --> L5[Layer 5\nInterface]
    L5 --> L6[Layer 6\nApplication]
    L6 --> L7[Layer 7\nGovernance]
    L7 --> L8[Layer 8\nPolicy]
    L7 --> Ledger[Governance Ledger Node]
    Ledger --> Registry[Governance Namespace Registry]
```

\paragraph{4. Conformance Classes}

\begin{itemize}[leftmargin=*]
\item \textbf{Class A -- Conceptual compliance:} Component prototypes implement IDDs and dignity safeguards with offline validation.
\item \textbf{Class B -- AEIP-Lite node:} Implements AEIP v1 handshake and submits artifacts to a local governance ledger.
\item \textbf{Class C -- Federated ledger integration:} Synchronizes AEIP transcripts with distributed ledgers and publishes to \texttt{registry\_gns.yaml}.
\end{itemize}

\paragraph{5. AEIP Integration}

The AEIP transport spine SHALL implement the five-step handshake (\texttt{Intent → Justify → CounterSign → Commit → Update}).  Headers MUST include \texttt{aeipVersion}, \texttt{temporalSeal}, \texttt{personaSignature}, and \texttt{governanceScope}.  Successful handshakes SHALL emit an Integrated Ledger Entry (ILE) conforming to \texttt{schemas/ile\_schema.json}.

```mermaid
sequenceDiagram
    participant A as Instruction Node
    participant B as Governance Node
    A->>B: Intent
    B-->>A: Justify
    A->>B: CounterSign
    B-->>A: Commit
    A->>B: Update
    Note over A,B: All steps signed & dignity compliant
```

\paragraph{6. Governance Clauses}

Temporal governance SHALL be enforced by \texttt{temporalSeal} issuance per step.  Drift detection is achieved by comparing handshake timestamps with ledger arrival time.  Dignity constraints derive from the Persona Architecture normative language and SHALL trigger refusal pathways whenever \texttt{dignity\_compliance=False}.

\paragraph{7. Change Management}

Updates to this specification MUST be captured as GDS and DRR artifacts, anchored to the ledger using AEIP handshakes, and versioned in \texttt{/docs/AI\_OSI\_Protocol\_Spec.md}.  Derived implementations SHALL reference the ledger entry ID for traceability.

---

\emph{This living document is published under CC BY-NC-ND 4.0 and inherits the Normative Language Notice from INTEGRITY\_NOTICE.md.}
\section{Governance Artifacts and Schemas}
\emph{Scope: Harmonizes artifact definitions from the integrated stack, governance map, and repository schemas.}
The AI OSI Stack relies on reusable artifacts that provide audit-ready evidence across every layer. Governance Disclosure Statements (GDS) synthesize cross-layer obligations; Integrity Ledger Entries (ILE) notarize decisions and maturity attestations; Oversight Action Memoranda (OAM) document interventions and remediations; Temporal Review Records (TRR) capture version checks; Interpretive Trace Packages (ITP) reconstruct evidentiary reasoning; and Decision Rationale Records (DRR) preserve architectural justification. AEIP provides the transport for these artifacts, guaranteeing interoperable audit traces. Artifacts SHALL be version-controlled, cryptographically signed, and exposed through Layer~7 portals. Canonical JSON-LD definitions are maintained within the repository's `schemas/` directory, with validation exemplars in the test suite.

\subsection{Artifact Definitions}
\emph{Scope: Defines each artifact class and required fields.}
\begin{enumerate}[leftmargin=*]
    \item \textbf{Interpretive Trace Package (ITP).} Captures reasoning lineage, dataset provenance, and contextual constraints. Required fields include `itpId`, `layerId`, `temporalSeal`, `reasoningTrace`, and `dignityCompliance`.
    \item \textbf{Decision Rationale Record (DRR).} Documents architectural justification, trade-offs, and governance obligations. Required fields include `decisionContext`, `riskStatement`, `controls`, and `approverSignature`.
    \item \textbf{Governance Disclosure Statement (GDS).} Publishes consolidated governance posture, Implementation Maturity Model level, and Adaptive Governance Metrics. Required fields include `gdsId`, `immLevel`, `transparencyRatio`, `governanceCoverageScore`, `driftIndex`, and `dignityCompliance`.
    \item \textbf{Oversight Action Memorandum (OAM).} Records interventions, simulations, and remediations. Required fields include `oamId`, `triggerEvent`, `responseActions`, `lessonsLearned`, and `custodianSignature`.
    \item \textbf{Integrity Ledger Entry (ILE).} Serves as the authoritative ledger record linking AEIP handshakes to published evidence. Required fields include `ileId`, `aeipTranscriptHash`, `temporalSeal`, `governanceScope`, and `publicationPointer`.
    \item \textbf{Temporal Review Record (TRR).} Tracks version checkpoints, release approvals, and change control decisions. Required fields include `trrId`, `releaseVersion`, `reviewFindings`, `followUpActions`, and `temporalSeal`.
\end{enumerate}

\subsection{Sample JSON Fragments}
\emph{Scope: Demonstrates canonical structure with inline excerpts.}
\begin{verbatim}
{
  "itpId": "itp-2025-IMM-04",
  "layerId": "L4",
  "temporalSeal": "2025-11-30T12:00:00Z#9ab3...",
  "reasoningTrace": {
    "persona": "instruction.steward",
    "steps": ["collect policy briefs", "validate refusal logic", "record CDI"]
  },
  "dignityCompliance": true
}
\end{verbatim}
\begin{verbatim}
{
  "gdsId": "gds-2025-annual",
  "immLevel": 4,
  "transparencyRatio": 0.92,
  "governanceCoverageScore": 0.87,
  "driftIndex": 0.05,
  "dignityCompliance": true,
  "publications": ["ile-2025-11-01-001", "ile-2025-11-15-007"],
  "oversightActions": ["oam-2025-resilience", "oam-2025-affect"]
}
\end{verbatim}

\subsection{Validation Rules and Temporal Seals}
\emph{Scope: Enumerates invariants enforced by Interface Definition Documents.}
All Interface Definition Documents inherit invariants:\begin{enumerate}[leftmargin=*]
    \item \textbf{Provenance.} `layerId`, `temporalSeal`, `dignityCompliance`, and `hash` MUST be present.
    \item \textbf{Integrity.} Hashes are computed from canonical JSON with the `hash` field removed prior to hashing.
    \item \textbf{Extensibility.} Unknown fields are tolerated but SHALL NOT break canonicalization.
    \item \textbf{Dignity.} Setting `dignityCompliance=false` SHALL signal refusal upstream.
\end{enumerate}
`tests/test_layer_contracts.py` executes a synthetic end-to-end flow across all layers, ensuring each IDD is satisfied and that AEIP payloads generated by Layer~4 serialize temporal seals correctly. Updates to schemas SHALL be anchored via AEIP handshakes and captured as Decision Rationale Records paired with Governance Disclosure Statements before ledger publication.

\subsection{Interface Definition Document Guide Source Text}
\emph{Scope: Provides the full IDD guide.}
\subsubsection*{Interface Definition Document (IDD) Guide}

\paragraph{Purpose}
This guide codifies the Interface Definition Documents (IDDs) that connect adjacent layers in the AI OSI reference stack.  Each IDD is represented as a JSON Schema stored alongside layer packages under \texttt{/src/layer*/schema.json}.  Validators within each layer enforce these contracts and compute SHA3-512 hashes across canonical payloads.

\paragraph{Layer Interfaces}

| Producer | Consumer | Schema Path | Notes |
| --- | --- | --- | --- |
| Layer1Physical | Layer2Architecture | \texttt{src/layer1\_physical/schema.json} | Physical telemetry envelope, ensures \texttt{temporalSeal}. |
| Layer2Architecture | Layer3Training | \texttt{src/layer2\_architecture/schema.json} | Persona graph and AEIP interface negotiation. |
| Layer3Training | Layer4Instruction | \texttt{src/layer3\_training/schema.json} | Training spec including dataset provenance. |
| Layer4Instruction | Layer5Interface | \texttt{src/layer4\_instruction/schema.json} | AEIP-ready instruction packets, includes \texttt{aeipVersion}. |
| Layer5Interface | Layer6Application | \texttt{src/layer5\_interface/schema.json} | AEIP headers plus persona signature propagation. |
| Layer6Application | Layer7Governance | \texttt{src/layer6\_application/schema.json} | Application responses providing service outcome context. |
| Layer7Governance | Layer8Policy | \texttt{src/layer7\_governance/schema.json} | Governance decisions, ledger digest, normative clause pointers. |
| Layer8Policy | External Civic Nodes | \texttt{src/layer8\_policy/schema.json} | Civic directives for participatory governance feedback. |

\paragraph{Validation Contract}

All IDDs inherit the following invariants:

\begin{enumerate}[leftmargin=*]
\item \textbf{Provenance} -- \texttt{layerId}, \texttt{temporalSeal}, \texttt{dignityCompliance}, and \texttt{hash} MUST be present.
\item \textbf{Integrity} -- Hashes are computed from canonical JSON with the \texttt{hash} field removed prior to hashing.
\item \textbf{Extensibility} -- Unknown fields are tolerated but SHALL NOT break canonicalization.
\item \textbf{Dignity} -- Setting \texttt{dignityCompliance} to \texttt{false} SHALL signal refusal upstream.
\end{enumerate}

\paragraph{Conformance Testing}

\texttt{tests/test\_layer\_contracts.py} executes a synthetic end-to-end flow across all layers, ensuring each IDD is satisfied and that AEIP payloads generated by Layer 4 serialize temporal seals correctly.  Additional validator unit tests live within each \texttt{validator.py} module.

\paragraph{Updating an IDD}

\begin{enumerate}[leftmargin=*]
\item Update the target schema file and corresponding validator.
\item Regenerate artifact examples using \texttt{/tools/generate\_artifact.py}.
\item Document the change in \texttt{/docs/AI\_OSI\_Protocol\_Spec.md} and capture a DRR + GDS artifact pair.
\item Anchor the change via AEIP handshake and submit to the ledger node.
\end{enumerate}

All modifications SHALL maintain compatibility with the dignity and provenance requirements enumerated above.
\section{Implementation Blueprint}
\emph{Scope: Describes offline-first reference implementation, directory structure, tests, and toolchain with integrated narrative from the README and implementation notes.}
The blueprint implementation anchors the conceptual stack in executable artifacts. It demonstrates how layered governance can be expressed in machine-readable, testable form. All tooling, tests, and examples operate offline-first and are engineered to illustrate design-time governance patterns, not production network services. Implementers SHALL treat the repository as a canonical rehearsal environment: architecture is finalized, but deployment awaits custodial ratification.

\subsection{Directory Structure}
\emph{Scope: Enumerates core components derived from the repository overview.}
\begin{verbatim}
.
├── docs/          # Specifications and layer documentation
├── examples/      # Demonstration notebooks
├── protocol/      # AEIP handshake and ledger implementation
├── schemas/       # Governance artifact schemas
├── src/           # Layer modules (L1–L8)
├── tests/         # Conformance suite
├── tools/         # CLI utilities for artifact management
├── ledger/        # Local integrity ledger
└── versions/      # Historical releases (non-normative)
\end{verbatim}

\subsection{Toolchain and Tests}
\emph{Scope: Highlights executable components and rehearsal flows.}
The blueprint includes AEIP handshake flows under `/protocol/aeip_handshake.py` and `/tests/test_aeip_handshake.py`, machine-readable schema definitions within `/schemas/`, and validator modules inside each layer directory (e.g., `src/layer4_instruction/validator.py`). `tests/test_layer_contracts.py` performs synthetic end-to-end validation across Interface Definition Documents. Artifact generation utilities reside in `/tools/generate_artifact.py`, providing deterministic governance rehearsals. Examples within `/examples/` illustrate local reasoning workflows and ledger integration for offline governance rehearsals.

\subsection{Artifact Flow Diagram}
\emph{Scope: Visualizes blueprint exchanges from Layer~4 through ledger publication.}
\begin{figure}[h]
    \centering
    \begin{tikzpicture}[node distance=2.6cm]
        \tikzstyle{block}=[rectangle, draw, rounded corners, align=center, minimum width=3.6cm, minimum height=1.2cm]
        \node[block] (l4) {Layer 4\newline Instruction Node\newline (Persona Brief)};
        \node[block, below of=l4] (l5) {Layer 5\newline AEIP Interface\newline (Handshake Engine)};
        \node[block, below of=l5] (l6) {Layer 6\newline Deployment Control\newline (TRR + OAM)};
        \node[block, below of=l6] (l7) {Layer 7\newline Governance Portal\newline (GDS Publication)};
        \node[block, below of=l7] (ledger) {Integrity Ledger\newline (ILE Archive)};
        \draw[->, thick] (l4) -- node[right]{ITP + Persona Mandate} (l5);
        \draw[->, thick] (l5) -- node[right]{AEIP Transcript} (l6);
        \draw[->, thick] (l6) -- node[right]{TRR + OAM Snapshot} (l7);
        \draw[->, thick] (l7) -- node[right]{GDS Publication} (ledger);
        \node[right=4cm of l5, align=left] {Validation hooks:\\- Schema hashes\\- Temporal seals\\- Dignity compliance};
    \end{tikzpicture}
    \caption{Blueprint Artifact Flow from Instruction Design to Ledger Publication}
\end{figure}

\subsection{Implementation Maturity Model Integration}
\emph{Scope: Reproduces the six-level maturity ladder from the integrated specification.}
The Implementation Maturity Model (IMM) defines six governance capability levels:\begin{enumerate}[leftmargin=*]
    \item \textbf{Level 0 --- No Governance.} Ad hoc AI usage without documented controls. Required artifacts: none. Expected audit frequency: incident-driven only.
    \item \textbf{Level 1 --- Basic/Reactive Governance.} Minimal policies and reactive incident handling. Required artifacts: baseline GDS, initial DRR. Expected audit frequency: annual self-assessment.
    \item \textbf{Level 2 --- Integrated Layer Controls.} Stack-aligned controls implemented across Layers~1--4 with Persona Architecture instantiated. Required artifacts: Persona briefs, AEIP readiness report, TRR baseline. Expected audit frequency: semiannual internal audits.
    \item \textbf{Level 3 --- Adaptive/Temporal Governance.} Temporal integrity tooling, drift monitoring, and AEIP ledger integration across Layers~1--6. Required artifacts: temporal audit log, OAM register, IMM attestation. Expected audit frequency: quarterly governance reviews.
    \item \textbf{Level 4 --- Trust Infrastructure.} Full AEIP deployment, public Layer~7 disclosures, and cross-jurisdictional interoperability. Required artifacts: comprehensive GDS, signed ILE packages, resilience playbooks. Expected audit frequency: continuous monitoring with biannual external audits.
    \item \textbf{Level 5 --- Cognitive Stewardship.} Cognitive diversity orchestration, affective constraint modules, and inter-organizational oversight. Required artifacts: CDI reports, dignity compliance attestations, governance simulation results. Expected audit frequency: continuous AEIP validation with annual third-party certification.
\end{enumerate}
Organizations SHALL publish their IMM level annually as an Integrity Ledger Entry to maintain transparency and accountability.
\subsection{Global Framework Interoperability Map}
\emph{Scope: Aligns the stack with international standards and sectoral frameworks.}
The AI OSI Stack maintains compatibility with NIST AI RMF, ISO/IEC~42001, the EU AI Act, and the TRUST framework while remaining adaptable to sector-specific constraints. Interoperability depends on mapping Governance Disclosure Statement chapters to regulatory reporting templates, ensuring Interpretive Trace Packages capture provenance evidence acceptable to multiple jurisdictions, and using AEIP transcripts as canonical exchanges for supervisory dialogues. Federation nodes SHALL support translation of AEIP payloads into sectoral submission formats without diluting normative commitments, leveraging the RegOps Bridge adapters specified in the expansion register.
\subsection{Implementation and Adoption Pathways}
\emph{Scope: Integrates guidance from the Test Integrated specification on adoption staging.}
Organizations adopting the AI OSI Stack SHALL internalize layered obligations, map existing controls to artifact expectations, and stage AEIP nodes in sandbox form. Implementation proceeds through phased rehearsal: \textbf{Phase~1} inventory existing governance artifacts and align them with stack layers; \textbf{Phase~2} instantiate AEIP handshake simulations and ledger recordings; \textbf{Phase~3} execute governance fire drills that rehearse incident, drift, and regulatory escalation scenarios; and \textbf{Phase~4} publish public-facing Governance Disclosure Statements synchronized with Integrity Ledger Entries. Custodians SHALL maintain change ledgers, ensure training for human overseers, and coordinate with regulators to synchronize compliance submissions.

\subsection{Implementation Notes Source Text}
\emph{Scope: Embeds the implementation notes document.}
\textcopyright{} 2025 Daniel P. Madden  
\textbf{License:} CC BY-NC-ND 4.0

\subsubsection*{AI OSI Stack --- Implementation Notes}
\textbf{Author:} Daniel P. Madden  
\textbf{Version:} v4 -- Blueprint Integration  
\textbf{Date:} November 2025

\begin{quote}Normative Language Notice\end{quote}
\begin{quote}This document uses normative language consistent with ISO/IEC 42010 and NIST conventions.\end{quote}
\begin{quote}“SHALL” denotes mandatory requirements, “SHOULD” denotes strong recommendations, and “MAY” denotes optional practices.\end{quote}
\begin{quote}Interpretations SHALL preserve authorial intent: layered accountability, epistemic integrity, and human dignity as binding design constraints.\end{quote}

\paragraph{1. Design Philosophy --- “Design Now, Instantiate Later”}
The AI OSI Stack operates as a governance-first reference implementation. Architectural intent is finalized in this blueprint; operational deployment is deferred until custodial councils confirm readiness. Each module, schema, and artifact is engineered to demonstrate how stewardship evidence MAY be produced, validated, and archived. Implementers SHALL treat the codebase as a canonical rehearsal environment rather than a production network.

\paragraph{2. Offline-Only Execution Constraints}
All tooling, tests, and examples SHALL execute without network access. Offline execution preserves audit reproducibility, protects unfinished persona credentials, and prevents accidental data egress. Where integrations with external systems are envisioned (e.g., Persona-PKI, Open Governance Registry), the repository provides placeholders and schema contracts only. Operators SHALL stage integrations in isolated sandboxes and SHALL NOT connect this blueprint to live infrastructure without explicit custodial authorization.

\paragraph{3. Artifact Integrity and Temporal Governance}
Schemas within \texttt{/schemas/} and validation logic under \texttt{/tests/} enforce deterministic hashing, temporal seals, and cross-layer traceability. Generated artifacts SHALL include SHA3-512 fingerprints, persona identifiers, and timestamps bound to UTC. Temporal legitimacy reviews MAY be rehearsed using the provided ledger utilities; actual regulatory submissions SHALL reference the canonical documents in \texttt{/source/} and \texttt{/docs/}.

\paragraph{4. Authorship Statement}
All original research, specifications, and blueprint logic remain authored by \textbf{Daniel P. Madden}. This repository is released to preserve the canonical architecture for future standardization and public stewardship. Derivative works SHALL obtain explicit consent prior to modification. Attribution SHALL reference the title \emph{The AI OSI Stack: Version 4 (Expanded --- Blueprint Integration Release)} and SHALL cite the Zenodo DOI once issued.

\section{Governance Map and Risk Taxonomy}
\emph{Scope: Consolidates the governance risk map, condensed risk taxonomy, temporal legitimacy framework, and adaptive metrics.}
The governance risk map maintains a layer-to-control matrix that surfaces primary risks, example controls, and evidence artifacts. Governance councils SHALL maintain this matrix as a living register. Controls MAY be tailored to sectoral mandates provided equivalent rigor, transparency, and dignity-first obligations are preserved. Artifact classes SHALL remain immutable to ensure cross-layer auditability.

\begin{longtable}{p{2.8cm}p{4.5cm}p{4.5cm}p{3.5cm}}
    \caption{Layer-to-Governance Control Matrix}\label{tab:control-matrix}\\
    \toprule
    \textbf{Layer} & \textbf{Primary Governance Risks} & \textbf{Example Controls} & \textbf{Evidence Artifacts} \\
    \midrule
    L1 --- Physical Substrate & Hardware tampering, supply chain opacity, resilience failure & Certified facility audits, redundant energy and cooling governance, tamper-evident custody & GDS, DRR \\
    L2 --- Data Stewardship & Consent erosion, provenance drift, epistemic contamination & Consent traceability ledgers, dataset review boards, epistemic hygiene playbooks & ITP, DRR \\
    L3 --- Model Development & Bias amplification, evaluation blind spots, red-team blind coverage & Composite evaluation harnesses, adversarial benchmarking cadences, persona-aligned regression suites & OAM, GDS \\
    L4 --- Instruction \& Control & Prompt injection, persona drift, affect misalignment & Persona mandate enforcement, refusal logic penetration testing, affect boundary audits & ITP, DRR, GDS \\
    L5 --- Reasoning Exchange & Ledger desynchronization, packet replay, counter-signature gaps & AEIP handshake validation, deterministic replay checkpoints, cross-node quorum review & ILE, OAM \\
    L6 --- Deployment \& Integration & Runtime drift, change conflicts, incident opacity & Change gating with governance approvals, blue/green stewardship protocols, incident rehearsal & TRR, OAM \\
    L7 --- Governance Publication & Delayed disclosure, accountability gaps, civic legitimacy erosion & Public disclosure cadence, multi-stakeholder review councils, archival notarization & GDS, ILE \\
    \bottomrule
\end{longtable}

\subsection{Temporal Integrity and Semantic Version Control}
\emph{Scope: Reproduces the temporal legitimacy framework.}
Governance is inherently temporal. Every change to Persona Architecture configurations, epistemic safeguards, or AEIP schemas requires semantic versioning anchored to custodial review. Temporal integrity is sustained through synchronized clocks, validity windows for each artifact, and automated drift alerts triggered by reasoning fingerprint deviation. Temporal legitimacy binds authority to time: each governance artifact receives a defined validity window during which its claims remain authoritative. Temporal amnesty clauses allow constrained grace periods to remediate violations without punitive escalation when self-disclosed promptly. Drift thresholds are codified: a 5\% change in reasoning fingerprint triggers mandatory review and potential rollback. Quarterly temporal audits evaluate alignment between declared versions and deployed systems. All temporal legitimacy events SHALL be recorded as AEIP-compatible ledger entries to guarantee verifiable lineage.

\subsection{Adaptive Governance Metrics}
\emph{Scope: Documents the Adaptive Governance Metrics (AGM) and Cognitive Diversity Index (CDI).}
Adaptive metrics provide continuous insight:\begin{itemize}[leftmargin=*]
    \item \textbf{Transparency Ratio (TR).} Published governance artifacts divided by required artifacts; recorded in Governance Disclosure Statements.
    \item \textbf{Governance Coverage Score (GCS).} Implemented controls divided by identified controls for the operational scope; logged alongside Implementation Maturity Model attestations.
    \item \textbf{Drift Index (DI).} Aggregated delta of performance and reasoning fingerprints normalized over time; stored in Temporal Review Records and linked via AEIP transcripts.
    \item \textbf{Dignity Compliance Rate (DCR).} Verified adherence to affective constraints divided by total evaluated interactions; captured within Governance Disclosure Statements and Integrity Ledger Entry updates.
\end{itemize}
The Cognitive Diversity Index (CDI) measures epistemic plurality. Defined as the ratio of distinct epistemic families engaged to total governance decisions, it encourages inclusive stewardship. Organizations target a CDI of 0.4 or higher for high-stakes domains. CDI reporting SHALL accompany Implementation Maturity Model and Adaptive Governance Metric disclosures, enabling auditors to detect monocultures that jeopardize resilience.

\subsection{Risk Taxonomy Synopsis}
\emph{Scope: Summarizes layer-specific risk themes, sentinel indicators, and controls.}
Each layer carries characteristic risks and prescribed controls as detailed in the condensed risk taxonomy. Sentinel indicators highlight anomalies that warrant Oversight Action Memoranda, while required evidence maps to artifact classes enumerated in Section~\ref{tab:control-matrix}. Cross-cutting dynamics include temporal integrity enforcement, custodianship documentation via AEIP-compatible ledgers, and continuous monitoring of affective constraint modules for dignity preservation.

\subsection{Governance Map Source Text}
\emph{Scope: Includes the governance risk map document verbatim.}
\textcopyright{} 2025 Daniel P. Madden  
\textbf{License:} CC BY-NC-ND 4.0

\subsubsection*{AI OSI Stack --- Governance Risk Map}
\textbf{Author:} Daniel P. Madden  
\textbf{Version:} v4 -- Blueprint Integration  
\textbf{Date:} November 2025

\begin{quote}Normative Language Notice\end{quote}
\begin{quote}This document uses normative language consistent with ISO/IEC 42010 and NIST conventions.\end{quote}
\begin{quote}“SHALL” denotes mandatory requirements, “SHOULD” denotes strong recommendations, and “MAY” denotes optional practices.\end{quote}
\begin{quote}Interpretations SHALL preserve authorial intent: layered accountability, epistemic integrity, and human dignity as binding design constraints.\end{quote}

\paragraph{1. Layer-to-Governance Control Matrix}
| AI OSI Layer | Primary Governance Risks | Example Controls | Evidence Artifacts |
| --- | --- | --- | --- |
| L1 -- Physical Substrate | Hardware tampering, supply chain opacity, resilience failure | Certified facility audits, redundant energy and cooling governance, tamper-evident custody | GDS, DRR |
| L2 -- Data Stewardship | Consent erosion, provenance drift, epistemic contamination | Consent traceability ledgers, dataset review boards, epistemic hygiene playbooks | ITP, DRR |
| L3 -- Model Development | Bias amplification, evaluation blind spots, red-team blind coverage | Composite evaluation harnesses, adversarial benchmarking cadences, persona-aligned regression suites | OAM, GDS |
| L4 -- Instruction & Control | Prompt injection, persona drift, affect misalignment | Persona mandate enforcement, refusal logic penetration testing, affect boundary audits | ITP, DRR, GDS |
| L5 -- Reasoning Exchange | Ledger desynchronization, packet replay, counter-signature gaps | AEIP handshake validation, deterministic replay checkpoints, cross-node quorum review | ILE, OAM |
| L6 -- Deployment & Integration | Runtime drift, change conflicts, incident opacity | Change gating with governance approvals, blue/green stewardship protocols, incident rehearsal | TRR, OAM |
| L7 -- Governance Publication | Delayed disclosure, accountability gaps, civic legitimacy erosion | Public disclosure cadence, multi-stakeholder review councils, archival notarization | GDS, ILE |

\paragraph{2. Stewardship Notes}
Governance councils SHALL maintain the above matrix as a living register. Controls MAY be tailored to sectoral mandates provided equivalent rigor, transparency, and dignity-first obligations are preserved. Artifact classes SHALL remain immutable to ensure cross-layer auditability.


\subsection{Risk Taxonomy Source Text}
\emph{Scope: Includes the condensed risk taxonomy in full.}
\textcopyright{} 2025 Daniel P. Madden  
\textbf{License:} CC BY-NC-ND 4.0

\subsubsection*{AI OSI Stack --- Condensed Risk Taxonomy}
\textbf{Author:} Daniel P. Madden  
\textbf{Version:} v4 -- Blueprint Integration  
\textbf{Date:} November 2025

\begin{quote}Normative Language Notice\end{quote}
\begin{quote}This document uses normative language consistent with ISO/IEC 42010 and NIST conventions.\end{quote}
\begin{quote}“SHALL” denotes mandatory requirements, “SHOULD” denotes strong recommendations, and “MAY” denotes optional practices.\end{quote}
\begin{quote}Interpretations SHALL preserve authorial intent: layered accountability, epistemic integrity, and human dignity as binding design constraints.\end{quote}

\paragraph{1. Layer 1 --- Physical and Compute Substrate}
\begin{itemize}[leftmargin=*]
\item \textbf{Risk Themes:} Hardware tampering, supply chain disruption, resilience degradation.
\item \textbf{Sentinel Indicators:} Facility access anomalies, unscheduled firmware changes, energy draw deviations.
\item \textbf{Primary Controls:} Certified facility audits, redundant energy governance, tamper-evident custody logs.
\item \textbf{Required Evidence:} GDS section L1, DRR infrastructure appendix, ILE temporal seals for facility inspections.
\end{itemize}

\paragraph{2. Layer 2 --- Data Stewardship}
\begin{itemize}[leftmargin=*]
\item \textbf{Risk Themes:} Data poisoning, consent erosion, epistemic contamination.
\item \textbf{Sentinel Indicators:} Provenance gaps, unexplained distribution shifts, contested consent revocations.
\item \textbf{Primary Controls:} Consent traceability ledgers, dataset review boards, epistemic hygiene playbooks.
\item \textbf{Required Evidence:} ITP provenance bundles, DRR data stewardship summaries, AEIP lineage signatures.
\end{itemize}

\paragraph{3. Layer 3 --- Model Development}
\begin{itemize}[leftmargin=*]
\item \textbf{Risk Themes:} Bias amplification, evaluation blind spots, adversarial regressions.
\item \textbf{Sentinel Indicators:} Diverging fairness metrics, untested safety scenarios, failed regression thresholds.
\item \textbf{Primary Controls:} Composite evaluation harnesses, adversarial benchmarking cadences, persona-aligned regression suites.
\item \textbf{Required Evidence:} OAM evaluation actions, GDS validation annex, DRR design decisions.
\end{itemize}

\paragraph{4. Layer 4 --- Instruction and Control}
\begin{itemize}[leftmargin=*]
\item \textbf{Risk Themes:} Prompt injection, persona drift, affect misalignment, refusal degradation.
\item \textbf{Sentinel Indicators:} Persona mismatch alerts, refusal override logs, affect boundary excursions.
\item \textbf{Primary Controls:} Persona mandate enforcement, refusal logic penetration testing, affect boundary audits.
\item \textbf{Required Evidence:} ITP instruction traces, DRR persona matrices, GDS instruction governance statements.
\end{itemize}

\paragraph{5. Layer 5 --- Reasoning Exchange and Interface}
\begin{itemize}[leftmargin=*]
\item \textbf{Risk Themes:} Ledger desynchronization, packet replay, counter-signature gaps, handshake tampering.
\item \textbf{Sentinel Indicators:} Hash mismatches, delayed acknowledgements, anomalous replay counts.
\item \textbf{Primary Controls:} AEIP handshake validation, deterministic replay checkpoints, cross-node quorum review.
\item \textbf{Required Evidence:} ILE handshake bundles, OAM remediation logs, AEIP transcript archives.
\end{itemize}

\paragraph{6. Layer 6 --- Deployment and Integration}
\begin{itemize}[leftmargin=*]
\item \textbf{Risk Themes:} Runtime drift, integration conflicts, incident opacity.
\item \textbf{Sentinel Indicators:} Unauthorized configuration changes, failed deployment gates, unreported incidents.
\item \textbf{Primary Controls:} Change gating with governance approvals, blue/green stewardship protocols, incident rehearsal cadences.
\item \textbf{Required Evidence:} TRR release attestations, OAM incident records, ILE change approvals.
\end{itemize}

\paragraph{7. Layer 7 --- Governance Publication}
\begin{itemize}[leftmargin=*]
\item \textbf{Risk Themes:} Delayed disclosure, accountability gaps, legitimacy erosion.
\item \textbf{Sentinel Indicators:} Missed publication cadences, stakeholder complaints, inconsistent disclosure metadata.
\item \textbf{Primary Controls:} Public disclosure schedules, multi-stakeholder review councils, archival notarization.
\item \textbf{Required Evidence:} GDS transparency packs, ILE publication proofs, custodial meeting minutes.
\end{itemize}

\paragraph{8. Cross-Cutting Dynamics}
\begin{itemize}[leftmargin=*]
\item \textbf{Temporal Integrity:} Validity windows SHALL be enforced with temporal seals and quarterly audits.
\item \textbf{Custodianship:} Governance councils SHALL document every change request, remediation, and escalation in AEIP-compatible ledgers.
\item \textbf{Human Dignity:} Affective constraint modules SHALL be monitored for erosion; violations trigger immediate OAM issuance and persona retraining.
\end{itemize}

\section{Future Expansion Blueprints}
\emph{Scope: Integrates the expansion blueprint register and forward-looking initiatives.}
The expansion blueprint register enumerates proposed systems that extend the AI OSI Stack while preserving its normative guardrails. Governance councils SHALL review each proposal quarterly; progress MAY advance only after risk controls, human oversight protocols, and archival requirements are validated against this Master Edition.

\subsection{Persona-PKI}
\emph{Scope: Establishes credential authority requirements.}
\begin{itemize}[leftmargin=*]
    \item \textbf{Purpose.} Establish a distributed credential authority that binds Persona Architecture roles to cryptographic identities, ensuring refusals, affect constraints, and mandate boundaries remain verifiable across deployments.
    \item \textbf{Inputs.} Persona briefs, mandate definitions, key provenance attestations, AEIP node registry metadata.
    \item \textbf{Outputs.} Signed persona certificates, revocation manifests, cross-layer trust anchors for AEIP sessions.
    \item \textbf{Implementation Status.} Proposed --- specification drafted, awaiting custodial review and interoperability testing.
\end{itemize}

\subsection{Open Governance Registry}
\emph{Scope: Provides federated disclosure catalogues.}
\begin{itemize}[leftmargin=*]
    \item \textbf{Purpose.} Offer a federated disclosure catalogue through which institutions MAY publish Governance Disclosure Statements, Integrity Ledger Entries, and oversight actions to satisfy transparency obligations.
    \item \textbf{Inputs.} Validated governance artifacts, custodial approvals, temporal legitimacy attestations.
    \item \textbf{Outputs.} Searchable registry entries, subscription feeds for auditors, notarized snapshots for archival partners.
    \item \textbf{Implementation Status.} Proposed --- registry schema aligned; hosting, access policies, and data minimization controls pending ratification.
\end{itemize}

\subsection{RegOps Bridge}
\emph{Scope: Defines regulatory integration adapters.}
\begin{itemize}[leftmargin=*]
    \item \textbf{Purpose.} Translate AEIP artifacts into sectoral regulatory submission formats without diluting normative commitments.
    \item \textbf{Inputs.} AEIP handshake transcripts, Decision Rationale Records, Oversight Action Memoranda, jurisdiction-specific compliance templates.
    \item \textbf{Outputs.} Deterministic export bundles (XML/JSON/PDF), submission logs, regulator acknowledgment receipts.
    \item \textbf{Implementation Status.} Proposed --- adapter interface defined; awaiting liaison agreements with regulatory bodies.
\end{itemize}

\subsection{Federation Testnet}
\emph{Scope: Prepares multi-jurisdiction simulations.}
\begin{itemize}[leftmargin=*]
    \item \textbf{Purpose.} Simulate multi-jurisdiction AEIP exchanges across sandboxed nodes to validate resilience, consensus on ledger replay, and escalation workflows before production deployment.
    \item \textbf{Inputs.} Reference AEIP node configurations, signed test artifacts, persona PKI stubs, governance council adjudication scripts.
    \item \textbf{Outputs.} Replayable handshake archives, drift diagnostics, federation readiness reports for governance councils.
    \item \textbf{Implementation Status.} Proposed --- network topology drafted; requires hardware allocation and custodial oversight approval.
\end{itemize}

\subsection{Expansion Blueprint Register Source Text}
\emph{Scope: Embeds the expansion blueprint register.}
\textcopyright{} 2025 Daniel P. Madden  
\textbf{License:} CC BY-NC-ND 4.0

\subsubsection*{AI OSI Stack --- Expansion Blueprint Register}
\textbf{Author:} Daniel P. Madden  
\textbf{Version:} v4 -- Blueprint Integration  
\textbf{Date:} November 2025

\begin{quote}Normative Language Notice\end{quote}
\begin{quote}This document uses normative language consistent with ISO/IEC 42010 and NIST conventions.\end{quote}
\begin{quote}“SHALL” denotes mandatory requirements, “SHOULD” denotes strong recommendations, and “MAY” denotes optional practices.\end{quote}
\begin{quote}Interpretations SHALL preserve authorial intent: layered accountability, epistemic integrity, and human dignity as binding design constraints.\end{quote}

\paragraph{1. Persona-PKI}
\begin{itemize}[leftmargin=*]
\item \textbf{Purpose:} Establish a distributed credential authority that binds Persona Architecture roles to cryptographic identities, ensuring refusals, affect constraints, and mandate boundaries remain verifiable across deployments.
\item \textbf{Inputs:} Persona briefs, mandate definitions, key provenance attestations, AEIP node registry metadata.
\item \textbf{Outputs:} Signed persona certificates, revocation manifests, cross-layer trust anchors for AEIP sessions.
\item \textbf{Implementation Status:} Proposed --- specification drafted, awaiting custodial review and interoperability testing.
\end{itemize}

\paragraph{2. Open Governance Registry (OGR)}
\begin{itemize}[leftmargin=*]
\item \textbf{Purpose:} Provide a federated disclosure catalogue through which institutions MAY publish Governance Disclosure Statements (GDS), Integrity Ledger Entries (ILE), and oversight actions to satisfy transparency obligations.
\item \textbf{Inputs:} Validated governance artifacts (GDS, ILE, OAM), custodial approvals, temporal legitimacy attestations.
\item \textbf{Outputs:} Searchable registry entries, subscription feeds for auditors, notarized snapshots for archival partners.
\item \textbf{Implementation Status:} Proposed --- registry schema aligned; hosting, access policies, and data minimization controls pending ratification.
\end{itemize}

\paragraph{3. RegOps Bridge (API Adapters)}
\begin{itemize}[leftmargin=*]
\item \textbf{Purpose:} Translate AEIP artifacts into sectoral regulatory submission formats (e.g., supervisory portals, NIS2 registers) without diluting normative commitments.
\item \textbf{Inputs:} AEIP handshake transcripts, DRR and OAM payloads, jurisdiction-specific compliance templates.
\item \textbf{Outputs:} Deterministic export bundles (XML/JSON/PDF), submission logs, regulator acknowledgment receipts.
\item \textbf{Implementation Status:} Proposed --- adapter interface defined; awaiting liaison agreements with regulatory bodies.
\end{itemize}

\paragraph{4. Federation Testnet}
\begin{itemize}[leftmargin=*]
\item \textbf{Purpose:} Simulate multi-jurisdiction AEIP exchanges across sandboxed nodes to validate resilience, consensus on ledger replay, and escalation workflows before production deployment.
\item \textbf{Inputs:} Reference AEIP node configurations, signed test artifacts, persona PKI stubs, governance council adjudication scripts.
\item \textbf{Outputs:} Replayable handshake archives, drift diagnostics, federation readiness reports for governance councils.
\item \textbf{Implementation Status:} Proposed --- network topology drafted; requires hardware allocation and custodial oversight approval.
\end{itemize}

\paragraph{5. Stewardship Roadmap Governance}
Governance councils SHALL review each proposed system quarterly. Progress MAY advance only after risk controls, human oversight protocols, and archival requirements are validated against the canonical AI OSI Stack specification.

\section{Federation and RegOps Roadmap}
\emph{Scope: Outlines distributed oversight, regulatory operations bridges, and societal foresight trajectories.}
Distributed oversight balances local autonomy with global assurance. The architecture comprises three tiers: Local Ledger Nodes maintain operational records; Regional Custodians aggregate sector or jurisdiction data; and a Global Trust Anchor synchronizes standards without centralizing control. AEIP secures inter-tier exchanges, enabling cross-jurisdictional audits and mutual recognition agreements. Supply chain governance requires all vendors to publish Stack-aligned Governance Cards detailing Persona Architecture conformance, epistemic safeguards, and AEIP integration status. Procurement processes SHALL mandate Stack-Aligned Reports assessing each supplier's Implementation Maturity Model level, with attestations logged via Integrity Ledger Entries.

RegOps bridges SHALL translate AEIP artifacts into supervisory submission formats, ensuring regulators receive deterministic evidence bundles without manual transcription. Federation participants SHALL implement replayable handshake archives and drift diagnostics, enabling governance councils to compare reasoning fingerprints across jurisdictions. Scenario planning explores three anchor futures for 2030--2040: \textbf{Fragmented Governance}, where jurisdictions diverge; \textbf{Unified AEIP}, where interoperability prevails; and \textbf{AI--Human Co-Governance}, where shared stewardship emerges. Inter-temporal arbitration mechanisms reconcile decisions that impact multiple timelines, ensuring present-day actions remain legitimate across futures.

The Open Implementation Registry (OIR) functions as a public or semi-public ledger enumerating adopters of the AI OSI Stack. Each entry records organization name, Implementation Maturity Model level, adoption date, and referenced artifacts. Entries SHALL link to published Governance Disclosure Statement packages and Integrity Ledger Entry hashes, enabling external verification while protecting sensitive details through selective disclosure. Persona-PKI integration SHALL provide cryptographic attestation of registry submissions, while RegOps adapters SHALL feed regulator acknowledgments back into custodial dashboards.

\section{Supplemental Canonical Commentary}
\emph{Scope: Provides extended narrative analysis drawn from canonical sources to contextualize governance obligations.}
\subsection{Adaptive Governance and Foresight}
\emph{Scope: Elaborates foresight practices using Test Integrated narratives.}
Adaptive governance treats foresight as a living practice. Institutions SHALL convene multidisciplinary councils that rehearse divergent futures, using AEIP transcripts to anchor commitments against uncertainty. The Test Integration Draft details scenario matrices that compare regulatory divergence, interoperable AEIP regimes, and co-governance futures. Each scenario is translated into Oversight Action Memoranda so that commitments are explicit, reviewable, and reversible.

Governance councils SHALL publish annual foresight reports summarizing which scenario indicators manifested, how drift thresholds were managed, and what remedial actions were authorized. These reports cross-reference Governance Disclosure Statements and Integrity Ledger Entries, ensuring stakeholders can trace how predictions informed operational changes.

Adaptive governance also synchronizes with Implementation Maturity Model levels. Organizations at IMM Level~3 or higher SHALL maintain foresight playbooks that map to Temporal Review Records, illustrating how each planned intervention aligns with semantic version checkpoints.

\subsection{Layer Interdependence and Power Geometry Deep Dive}
\emph{Scope: Expands on interdependence insights from the Test Integration Draft.}
Layer boundaries are designed to remain permeable so that accountability flows across vertical and horizontal axes. Power geometry analysis reveals how decision authority, data provenance, and execution logic concentrate differently across layers. Layer~4 personas enforce mandates that originate in civic Layer~0 directives, while Layer~5 AEIP exchanges propagate those directives into runtime obligations.

To prevent capture, governance councils SHALL document cross-layer dependencies in Decision Rationale Records, including explicit references to consent registries, persona briefs, and ledger checkpoints. When a single organizational unit spans multiple layers, custodians SHALL institute separation-of-duties controls recorded as Integrity Ledger Entries to demonstrate impartial oversight.

Power geometry also informs federation strategy. Regional custodians coordinate Layer~6 deployment decisions with civic mandates captured at Layer~0, ensuring that local adaptations do not dilute global accountability commitments.

\subsection{Temporal Pluralism and Scenario Translation}
\emph{Scope: Extends the Societal Foresight narrative with operational detail.}
Societal foresight acknowledges that governance decisions reverberate across timelines. The Test Integration Draft articulates three anchor futures for 2030--2040. Institutions SHALL translate these futures into concrete policies: Fragmented Governance scenarios mandate resilience investments and redundancy in disclosure pipelines; Unified AEIP futures prioritize interoperability testing and joint audit exercises; AI--Human Co-Governance futures emphasize participatory oversight councils and cognitive diversity commitments.

Temporal pluralism requires inter-temporal arbitration boards that record deliberations via AEIP transcripts. Arbitration outcomes SHALL be published as Governance Disclosure Statement addenda so external stakeholders can assess how present-day choices respect future legitimacy.

Custodians SHALL maintain scenario evidence packages within Interpretive Trace Packages, linking each foresight outcome to metrics such as Transparency Ratio, Drift Index, and Dignity Compliance Rate.

\subsection{Human Dignity and Affective Constraint Amplification}
\emph{Scope: Builds on the dignity module to illustrate enforcement patterns.}
Dignity operates as a non-negotiable constraint across all personas. Instruction nodes SHALL include affect boundary checklists, refusal logic validation, and sentiment audit logs. When violations occur, Oversight Action Memoranda document remediation, persona retraining, and public disclosure steps.

The Human Dignity and Affective Constraint Module described in the Test Integration Draft requires custodians to monitor for manipulative sentiment scripts, intimacy simulations, and coercive cues. Organizations SHALL integrate automated detectors and human review panels, recording findings within Temporal Review Records.

Layer~7 governance portals publish Dignity Compliance Rates to demonstrate accountability to affected communities. Civic partners MAY request additional disclosures, triggering AEIP Update steps that refresh ledger entries with contextual explanations.

\subsection{Cognitive Diversity Stewardship}
\emph{Scope: Provides deeper guidance on the Cognitive Diversity Index.}
The Cognitive Diversity Index (CDI) encourages epistemic plurality across governance decisions. Institutions SHALL catalog epistemic families engaged in oversight, linking each decision to a CDI ledger reference.

The Test Integration Draft recommends a baseline CDI of 0.4 for high-stakes deployments. To achieve this threshold, governance councils SHALL recruit stakeholders from diverse disciplinary, cultural, and lived-experience backgrounds. Each council session SHALL produce Interpretive Trace Packages capturing perspectives contributed, dissent recorded, and resolution paths agreed.

CDI trends are published alongside Adaptive Governance Metrics within Governance Disclosure Statements, empowering auditors and communities to evaluate whether stewardship resists monoculture risks.

\subsection{Open Implementation Registry Operationalization}
\emph{Scope: Expands on the OIR concept from the Test Integration Draft.}
The Open Implementation Registry serves as a public ledger enumerating AI OSI Stack adopters. Entries record Implementation Maturity Model levels, adoption dates, artifact references, and persona attestations.

Governance councils SHALL ensure registry entries cite Integrity Ledger Entry hashes and Governance Disclosure Statement locations. Persona-PKI integrations SHALL sign submissions, enabling auditors to verify authenticity.

OIR data feeds into RegOps Bridge adapters so regulators receive consistent updates without manual transcription. Civic observers MAY subscribe to registry notifications, reinforcing transparency and layered accountability.


\subsection{Layer Accountability Records}
\emph{Scope: Summarizes evidence expectations per layer using integrated sources.}
Layer accountability is expressed through artifact bundles that cross-reference AEIP transcripts, Governance Disclosure Statements, and ledger seals. Layer~1 packages SHALL include facility attestations, energy governance metrics, and tamper-evident custody logs. Layer~2 packages consolidate consent registries, provenance ledgers, and Interpretive Trace Package appendices. Layer~3 bundles integrate evaluation harness outputs, adversarial benchmarking notes, and Oversight Action Memoranda summarizing red-team results.

Layers~4 and 5 depend on Persona Architecture briefs, refusal logic assessments, AEIP handshake transcripts, and Decision Rationale Records documenting instruction mandates. Layer~6 bundles aggregate Temporal Review Records, deployment manifests, incident retrospectives, and change approvals. Layer~7 evidence sets include publication cadences, civic consultation records, Integrity Ledger Entry hashes, and Implementation Maturity Model attestations. Optional Layer~0 civic dossiers collect participatory deliberations and social license documentation.

\subsection{Custodian Training Framework}
\emph{Scope: Defines training artifacts that operationalize governance roles.}
Custodians SHALL complete training curricula that cover Persona Architecture fundamentals, AEIP handshake mechanics, ledger operations, and affective constraint enforcement. Training materials include scenario walkthroughs, ledger replay exercises, and interpretive trace reconstruction drills. Certifications are logged as Integrity Ledger Entries referencing the curricula used, instructors involved, and expiration dates for competency refreshers.

Post-training assessments MAY incorporate open book reviews of Governance Disclosure Statements, requiring custodians to trace obligations to corresponding artifacts. Remediation plans address knowledge gaps and ensure every steward understands escalation paths, refusal triggers, and cross-layer dependencies.

\subsection{Persona Architecture Enforcement Patterns}
\emph{Scope: Explains how personas, mandates, and affect limits remain auditable.}
Persona briefs declare role identity, objectives, refusal clauses, and affect boundaries. Enforcement patterns tie these briefs to AEIP `Intent` payloads, ensuring each handshake begins with persona identification and dignity attestations. Regression suites evaluate persona compliance under normal and adversarial conditions. Results feed into Oversight Action Memoranda and Governance Disclosure Statement updates.

Custodians SHALL monitor for persona drift using metrics aligned with the Cognitive Diversity Index and Dignity Compliance Rate. When drift is detected, persona retraining plans specify updated instructions, human oversight checkpoints, and ledger entries documenting the transition.

\subsection{Cross-Jurisdiction Concordance Management}
\emph{Scope: Addresses harmonization of multi-jurisdiction deployments.}
Organizations operating across jurisdictions SHALL maintain concordance matrices mapping stack artifacts to regulatory citations. These matrices derive from the Layer-to-Standard Concordance appendix and include references to ISO/IEC standards, NIST frameworks, EU AI Act provisions, and local civic mandates. AEIP transcripts incorporate jurisdiction tags so auditors can filter evidence relevant to their authority.

When regulatory changes occur, custodians update concordance matrices and issue Decision Rationale Records describing the adaptation. Governance Disclosure Statements publish change logs, while Integrity Ledger Entries capture timestamps, approvers, and evidence pointers.


\subsection{Scenario Metric Mapping}
\emph{Scope: Connects foresight scenarios to quantitative indicators.}
Each foresight scenario from the Test Integration Draft aligns with a metrics bundle. Fragmented Governance scenarios emphasize energy resilience indices, redundancy counts, and time-to-disclosure benchmarks. Unified AEIP futures track handshake success rates, ledger synchronization latency, and cross-node replay audit completion. AI--Human Co-Governance futures prioritize Cognitive Diversity Index trends, dignity incident reports, and community feedback sentiment. These metrics feed into Adaptive Governance Metric dashboards and are referenced within Governance Disclosure Statements.

\subsection{Transparency Dashboard Expectations}
\emph{Scope: Defines visualization requirements for public governance portals.}
Layer~7 portals SHALL present transparency dashboards that expose Implementation Maturity Model levels, Adaptive Governance Metrics, ongoing Oversight Action Memoranda, and upcoming disclosure cadences. Dashboards provide public access to Integrity Ledger Entry hashes, persona mandates, and regulatory concordance matrices. Civic stakeholders MAY submit feedback, which is processed as AEIP `Intent` messages and tracked through ledgered response workflows.

\subsection{Ledger Operations Workflow}
\emph{Scope: Details the lifecycle of ledger entries within the offline blueprint.}
Ledger operations begin when AEIP `Update` steps request entry creation. Governance nodes validate signatures, verify `dignityCompliance`, and check temporal seals. Upon acceptance, nodes append Integrity Ledger Entries containing payload hashes, persona identifiers, and publication pointers. Subsequent amendments reference the original entry via cryptographic digests, preserving historical lineage. Ledger pruning policies SHALL respect archival obligations and include notarized snapshots for long-term storage.

\subsection{Federation Readiness Criteria}
\emph{Scope: Establishes conditions for joining the federation testnet.}
Organizations seeking federation participation SHALL demonstrate IMM Level~4 status, successful AEIP handshake conformance across five consecutive simulations, and publication of a Governance Disclosure Statement covering cross-border obligations. Custodians SHALL submit persona credential proofs via Persona-PKI stubs, share incident rehearsal logs, and agree to shared escalation protocols. Federation readiness reviews issue Decision Rationale Records summarizing findings and Integrity Ledger Entries documenting approval or remediation requirements.


\subsection{Adaptive Metrics Reporting Cadence}
\emph{Scope: Defines publication intervals for key indicators.}
Organizations SHALL publish Adaptive Governance Metrics quarterly, synchronizing updates with Governance Disclosure Statement releases. Interim monthly digests MAY highlight significant drift, transparency, or dignity deviations and SHALL be logged as Integrity Ledger Entries referencing the full quarterly report. Metric definitions SHALL remain stable unless Decision Rationale Records justify adjustments aligned with persona mandates and regulatory expectations.

\subsection{Persona Registry Maintenance}
\emph{Scope: Describes lifecycle management for persona credentials.}
Persona registries catalog mandate descriptions, affect boundaries, refusal patterns, and signature keys. Custodians SHALL review registries after every major release or within 90 days, whichever occurs first. Revoked or superseded personas SHALL retain archival records linked to AEIP transcripts, ensuring auditors can reconstruct historical behavior even after retirement.

\subsection{Risk Signal Escalation}
\emph{Scope: Establishes triggers for elevating risk signals across layers.}
Risk signals originate from sentinel indicators cataloged in the condensed risk taxonomy. When thresholds are crossed, automated alerts escalate through AEIP `Justify` messages to custodial teams. Escalations SHALL include recommended controls, affected artifacts, and proposed remediation timelines. Governance councils document final decisions in Oversight Action Memoranda, ensuring transparency regarding accepted risks, mitigations, or system rollbacks.


\subsection{Regulatory Liaison Coordination}
\emph{Scope: Describes interaction patterns with supervisory authorities.}
Regulatory liaisons SHALL maintain contact matrices linking stack layers to jurisdictional points of contact. AEIP transcripts capture pre-filing consultations, request-for-information exchanges, and supervisory feedback. Governance Disclosure Statements reference liaison outcomes, while Integrity Ledger Entries store acknowledgment receipts and agreed timelines.

\subsection{Community Accountability Forums}
\emph{Scope: Highlights civic engagement mechanisms.}
Layer~0 civic mandates encourage periodic accountability forums where affected communities review governance artifacts, question custodians, and propose refinements. Forums generate public summaries stored as Governance Disclosure Statement annexes, and AEIP Update steps disseminate responses. Civic insights feed into Adaptive Governance Metrics and may trigger scenario reassessments or persona adjustments.

\subsection{Cross-Layer Audit Trails}
\emph{Scope: Explains how audit trails traverse multiple artifacts.}
Audit trails weave through Interpretive Trace Packages, Decision Rationale Records, Oversight Action Memoranda, and Integrity Ledger Entries. Auditors SHALL be able to reconstruct a decision by following linked identifiers across these artifacts. Cross-layer audits confirm that persona mandates, temporal seals, and dignity constraints remain aligned during lifecycle events. Findings are published as Governance Disclosure Statement supplements and recorded in ledger entries for future verification.


\subsection{Adaptive Metric Remediation Plans}
\emph{Scope: Links metric deviations to mandated responses.}
When Transparency Ratio or Governance Coverage Score drop below thresholds declared in Governance Disclosure Statements, custodians SHALL initiate remediation plans detailing corrective actions, responsible teams, and timelines. Plans reference Oversight Action Memoranda and Temporal Review Records to ensure follow-through. AEIP Update steps broadcast remediation status to auditors and civic stakeholders.

\subsection{Persona-PKI Rollout Milestones}
\emph{Scope: Chronicles staged deployment of persona credentials.}
Persona-PKI implementations progress through milestone gates: prototype issuance, sandbox federation testing, cross-organization interoperability trials, and production readiness reviews. Each milestone generates Decision Rationale Records and Integrity Ledger Entries summarizing findings, residual risks, and next steps. Governance councils SHALL validate that persona certificates remain aligned with affective constraints and refusal logic definitions.

\subsection{Public Accountability Narrative}
\emph{Scope: Articulates messaging strategies for public trust.}
Public communications SHALL translate technical governance controls into accessible narratives emphasizing dignity, transparency, and stewardship. Governance Disclosure Statements include executive summaries, community impact assessments, and frequently asked questions. AEIP transcripts of public briefings provide verifiable evidence that messaging matches recorded commitments.


\subsection{Custodial Accountability Scorecards}
\emph{Scope: Introduces scorecards aligning custodial performance with stack metrics.}
Custodial councils SHALL maintain scorecards tracking response times, remediation completion rates, civic feedback satisfaction, and audit closure efficiency. Scorecards draw data from Oversight Action Memoranda, Governance Disclosure Statements, and AEIP transcripts. Results inform training updates and are published annually as Integrity Ledger Entry attachments.

\subsection{Knowledge Preservation Archives}
\emph{Scope: Ensures stewardship knowledge survives personnel changes.}
Knowledge archives consolidate interpretive traces, persona evolution records, regulatory correspondences, and simulation lessons. Archives are stored offline with cryptographic digests and stewardship instructions. Custodians SHALL verify archive integrity during temporal audits, guaranteeing future teams can reconstruct prior decisions without ambiguity.


\subsection{Continuous Improvement Backlog}
\emph{Scope: Maintains prioritized enhancements derived from audits and community input.}
Continuous improvement backlogs capture remediation tasks, innovation pilots, and policy harmonization initiatives. Each backlog item links to originating artifacts (e.g., Oversight Action Memoranda, civic forum notes, regulator correspondence) and includes target IMM levels, responsible custodians, and expected ledger updates. Backlog status reviews occur monthly and are summarized in Governance Disclosure Statements.

\subsection{Knowledge Transfer Workshops}
\emph{Scope: Coordinates cross-team sharing of governance practices.}
Workshops convene data stewards, persona designers, engineers, compliance leads, and civic liaisons to share lessons from recent incidents, simulations, and audits. Proceedings are transcribed into Interpretive Trace Packages and appended to knowledge archives. Participation metrics feed into Cognitive Diversity Index reporting, ensuring knowledge exchange remains inclusive.


\subsection{Governance Analytics Pipelines}
\emph{Scope: Outlines data processing required for continuous oversight analytics.}
Analytics pipelines consolidate telemetry from AEIP transcripts, persona audits, incident rehearsals, and civic feedback. Pipelines SHALL operate offline-first, producing dashboards, anomaly alerts, and trend analyses stored in Integrity Ledger Entries. Pipeline code is versioned alongside schema definitions, and updates require Decision Rationale Records plus regression evidence demonstrating metric continuity.

\subsection{Cross-Organizational Collaboration Protocols}
\emph{Scope: Formalizes collaboration with partner institutions and consortia.}
When multiple institutions co-govern AI services, they SHALL establish collaboration protocols covering artifact exchange, dispute resolution, and shared custody of federated ledgers. Protocols reference Persona-PKI trust anchors, AEIP handshake expectations, and concordance matrices. Joint Governance Disclosure Statements summarize collaborative obligations, while each custodian retains accountability for local layer controls.


\subsection{Stewardship Accountability Index}
\emph{Scope: Introduces a composite indicator aggregating stewardship performance.}
The Stewardship Accountability Index (SAI) aggregates Adaptive Governance Metrics, Cognitive Diversity Index scores, remediation timeliness, and civic feedback ratings into a normalized 0--1 scale. Institutions SHALL publish SAI values annually, accompanied by Decision Rationale Records explaining improvements or regressions. SAI calculations reference Integrity Ledger Entries and publicly auditable data sources.


\subsection{Custodial Peer Review Cadence}
\emph{Scope: Establishes cross-organization peer review of governance practices.}
Custodial councils SHALL participate in annual peer reviews where partner institutions evaluate artifact quality, ledger integrity, and persona compliance. Peer review findings are memorialized in Oversight Action Memoranda and published summaries, strengthening shared accountability.


\subsection{Governance Benchmark Exchanges}
\emph{Scope: Establishes peer benchmarking of governance metrics.}
Institutions SHALL participate in anonymized benchmark exchanges that compare Adaptive Governance Metrics, Stewardship Accountability Index values, and scenario preparedness scores. Exchanges operate under confidentiality agreements but publish aggregated insights to drive sector-wide improvement.

\subsection{Custodial Ethics Charter}
\emph{Scope: Codifies ethical commitments for governance custodians.}
The ethics charter enumerates commitments to transparency, non-retaliation toward whistleblowers, respect for affected communities, and adherence to dignity-first design. Custodians sign the charter annually, and charter violations trigger Oversight Action Memoranda and potential removal from governance roles.


\subsection{Custodial Peer Learning Exchanges}
\emph{Scope: Facilitates recurring knowledge swaps across institutions.}
Peer learning exchanges convene governance custodians on a bi-monthly cadence to review anonymized case studies, share audit methodologies, and evaluate emerging risks. Sessions include structured briefings, facilitated retrospectives, and working groups that map lessons to stack layers. Proceedings are recorded as Interpretive Trace Packages and published summaries accompany Governance Disclosure Statements to demonstrate openness.

\subsection{Scenario Stress Metric Library}
\emph{Scope: Catalogues quantitative stressors for foresight exercises.}
The metric library enumerates stress parameters such as supply chain disruption indices, policy volatility scores, persona drift rates, and ledger synchronization variance. Each metric references canonical artifacts, measurement scripts, and normalization formulas. Custodians SHALL update the library after each simulation cycle and expose changes through Integrity Ledger Entries for reproducibility.

\subsection{Civic Dialogue Integration}
\emph{Scope: Integrates civic feedback loops into governance operations.}
Civic dialogue channels capture community questions, impact statements, and oversight proposals. Feedback is triaged into categories (dignity, transparency, risk, innovation) and routed to relevant custodial teams. Responses are issued via AEIP Update messages, logged as Governance Disclosure Statement supplements, and revisited during quarterly accountability forums.

\subsection{RegOps Coordination Playbooks}
\emph{Scope: Aligns regulatory engagement tasks with stack artifacts.}
Playbooks outline pre-submission checklists, dossier assembly templates, and escalation contacts for each regulator. They map stack artifacts to regulatory clauses, specify persona signatories, and define response timelines for supervisory requests. Playbooks are versioned through Decision Rationale Records and shareable excerpts are posted for civic review to reinforce transparency.


\subsection{Custodial Performance Maturity Model}
\emph{Scope: Extends the Implementation Maturity Model with custodial performance tiers.}
The custodial performance maturity model defines tiered expectations for governance councils. Tier~1 custodians meet baseline training and ledger publication duties. Tier~2 custodians demonstrate proactive risk identification, scenario rehearsal leadership, and transparent remediation reporting. Tier~3 custodians orchestrate multi-jurisdiction collaborations, maintain civic liaison programs, and steward open benchmark exchanges. Advancement between tiers requires documented evidence, peer endorsements, and Integrity Ledger Entry approvals.

\subsection{Persona Collaboration Charters}
\emph{Scope: Governs interactions between multiple personas serving a shared mission.}
Persona collaboration charters delineate responsibilities, conflict resolution pathways, and shared affect constraints when multiple personas operate within a service. Charters include dependency maps, refusal escalation procedures, and AEIP handshake templates to prevent mandate overlap. Custodians review charters quarterly, adjusting instructions to maintain clarity and prevent accountability gaps.


\subsection{Audit Evidence Packaging Service}
\emph{Scope: Describes the offline tooling that assembles audit-ready bundles.}
The packaging service compiles Interpretive Trace Packages, Decision Rationale Records, Oversight Action Memoranda, and Integrity Ledger Entries into encrypted archives for auditor retrieval. Metadata manifests list artifact hashes, persona signatures, and temporal seals, ensuring reproducible verification.

\section{Custodial Oversight Structures and Simulation}
\emph{Scope: Describes custodianship, governance fire drills, human oversight obligations, ethical supply chain controls, and end-of-life governance.}
Custodianship rests with a designated Governance Council empowered to approve changes, review audits, and steward public accountability. Change management integrates AEIP change requests, Decision Rationale Record updates, and Oversight Action Memorandum tracking. Custodians SHALL maintain a change ledger, ensure training for human overseers, and coordinate with regulators to synchronize compliance submissions. Custodianship is designed to be portable across organizations while preserving the author's architectural intent.

Annual governance fire drills validate readiness for compound failures. Simulations incorporate synthetic data breaches, sudden policy shifts, and model drift scenarios. Outputs are consolidated into an Oversight Action Memorandum Resilience Report that documents response quality, time-to-remediation, and lessons learned. AEIP facilitates replay and external verification, ensuring simulations translate into tangible improvements. Distributed oversight architecture comprises Local Ledger Nodes, Regional Custodians, and the Global Trust Anchor, providing escalation pathways and cross-node quorum checks.

Human-in-the-governance-loop requirements mandate checkpoints during training data approval, instruction design, deployment readiness, and audit sign-off. Escalation paths route unresolved issues to the Governance Council, which can issue Oversight Action Memoranda or suspend operations. AEIP records human approvals to ensure traceability and to prevent automation from eroding oversight. The Cognitive Diversity Index guides council composition, encouraging inclusion of distinct epistemic families to sustain resilience.

Ethical AI supply chain governance requires vendors to publish Stack-aligned Governance Cards detailing Persona Architecture conformance, epistemic safeguards, and AEIP integration status. Procurement SHALL mandate Stack-Aligned Reports assessing Implementation Maturity Model levels. All supply chain attestations are logged via Integrity Ledger Entries to preserve accountability across the lifecycle.

Decommissioning and end-of-life governance demand deliberate closure. Organizations SHALL produce a Final Governance Disclosure Statement summarizing operational history, issue a Closure Integrity Ledger Entry indicating shutdown status, perform memory redaction consistent with privacy commitments, and publish a public ``AI epitaph'' explaining lessons learned. Ethical end-of-life practices honor affected communities and prevent orphaned systems from resurfacing without governance.
\section{Conclusion --- Architecture as Accountability}
\emph{Scope: Summarizes the accountability thesis and canonical commitments.}
Architecture is accountability. The AI OSI Stack v4 Master Edition demonstrates that governance can be engineered as infrastructure, binding dignity, epistemic integrity, and stewardship across every layer. By unifying conceptual references, AEIP transport mechanics, governance artifacts, and implementation rehearsals, the stack delivers a deployable blueprint that resists opacity and concentration. Institutions that adopt this specification SHALL publish their obligations, rehearse their responses, and federate their evidence in ways that honor human dignity. Future work extends through Persona-PKI, Open Governance Registry, and RegOps bridges, yet the normative commitments remain immutable: accountability is not an accessory but the core architecture of trustworthy intelligence.
\appendix
\section{Glossary}
\emph{Scope: Consolidates normative definitions and acronyms across integrated sources.}
\begin{description}[leftmargin=2.2cm, style=nextline]
    \item[AEIP] AI Epistemic Infrastructure Protocol --- protocol for exchanging reasoning artifacts with audit guarantees.
    \item[AGM] Adaptive Governance Metrics --- suite of continuous indicators (Transparency Ratio, Governance Coverage Score, Drift Index, Dignity Compliance Rate).
    \item[CDI] Cognitive Diversity Index --- ratio of distinct epistemic families engaged to total governance decisions.
    \item[DRR] Decision Rationale Record --- artifact preserving architectural justification, risk statements, and control selections.
    \item[GDS] Governance Disclosure Statement --- public artifact summarizing governance posture, metrics, and obligations.
    \item[ILE] Integrity Ledger Entry --- signed record anchoring governance claims in time, referencing AEIP transcripts.
    \item[IMM] Implementation Maturity Model --- six-level capability ladder describing governance readiness.
    \item[ITP] Interpretive Trace Package --- artifact capturing reasoning lineage, dataset provenance, and contextual constraints.
    \item[OAM] Oversight Action Memorandum --- document recording interventions, simulations, and remediations.
    \item[OIR] Open Implementation Registry --- ledger enumerating AI OSI Stack adopters.
    \item[Persona Architecture] Structured instruction and affect governance model ensuring AI personas act within declared mandates.
    \item[RegOps Bridge] Adapter translating AEIP artifacts into regulatory submission formats.
    \item[TRR] Temporal Review Record --- artifact documenting release checkpoints, drift findings, and follow-up actions.
    \item[Transparency Ratio] Published governance artifacts divided by required artifacts.
    \item[Dignity Compliance Rate] Verified adherence to affective constraints divided by total evaluated interactions.
\end{description}

\section{Layer-to-Standard Concordance}
\emph{Scope: Maps AI OSI layers to external standards and regulatory references.}
\begin{longtable}{p{2.6cm}p{5.2cm}p{6cm}}
    \caption{Layer Concordance with External Standards}\\
    \toprule
    \textbf{Layer} & \textbf{Primary External Standards} & \textbf{Concordance Notes} \\
    \midrule
    L0 --- Civic Mandate & OECD AI Principles, UNESCO Recommendation on AI Ethics & Civic directives SHALL align with participatory governance expectations and human rights clauses. \\
    L1 --- Physical Substrate & ISO/IEC~27001, NIST SP~800-53, EN~50600 & Facility attestations SHALL incorporate energy stewardship and tamper-evident custody logs. \\
    L2 --- Data Stewardship & ISO/IEC~27560, EU GDPR, NIST Privacy Framework & Provenance ledgers SHALL support consent traceability and revocation workflows. \\
    L3 --- Model Development & NIST AI RMF Map/Measure, ISO/IEC~24028 & Evaluation harnesses SHALL document fairness metrics, adversarial coverage, and regression baselines. \\
    L4 --- Instruction \& Control & Persona Architecture canon, IEC~61508 (safety cases) & Instruction packets SHALL encode refusal logic, affect boundaries, and persona mandates. \\
    L5 --- Reasoning Exchange & IETF-inspired handshake patterns, ISO/IEC~30170 security controls & AEIP headers SHALL be signed, hashed, and time-aligned with ledger nodes. \\
    L6 --- Deployment \& Integration & ITIL Change Management, NIST SP~800-184 & Temporal Review Records SHALL gate releases with blue/green stewardship protocols. \\
    L7 --- Governance Publication & EU AI Act Article~72, US OSTP Blueprint for an AI Bill of Rights & Governance disclosures SHALL maintain public cadence, multi-stakeholder review, and archival notarization. \\
    L8 --- Policy & Civic participation statutes, multi-stakeholder governance frameworks & Civic overlays SHALL reconcile inter-temporal mandates and federation obligations. \\
    \bottomrule
\end{longtable}

\section{Change Log}
\emph{Scope: Reproduces the canonical changelog.}
\begin{itemize}[leftmargin=*]
    \item \textbf{v1.0 --- Foundational Stack Overview (2019-09-01).} Introduced the seven-layer governance metaphor, integrity ledger concept, and initial stewardship obligations for AI services.
    \item \textbf{v2.0 --- Persona Architecture Expansion (2020-12-15).} Added Persona Architecture briefs, affective constraint models, and role-specific accountability patterns.
    \item \textbf{v3.0 --- Epistemology Alignment (2022-03-10).} Integrated Epistemology by Design, introduced Interpretive Trace Packages, and prepared AEIP compatibility guidance.
    \item \textbf{v4.0 --- Integrated Canonical Specification (2024-07-22).} Published AI OSI Stack v4 Test Integrated, harmonizing governance artifacts, temporal legitimacy, and the Implementation Maturity Model.
    \item \textbf{v4 Protocol Blueprint --- Blueprint Complete Reference Implementation (2025-11-30).} Finalized offline-first blueprint codebase, added governance maps, deferred system registry, validation artifacts, and Zenodo-ready release package for archival DOI issuance.
\end{itemize}

\section{References}
\emph{Scope: Lists primary documents integrated into this Master Edition.}
\begin{enumerate}[leftmargin=*]
    \item Daniel P.~Madden, ``AI OSI Stack v4 Test Integrated.''
    \item Daniel P.~Madden, ``AI OSI Protocol Spec.''
    \item Daniel P.~Madden, ``AEIP v1 Specification.''
    \item Daniel P.~Madden, ``Interface Definition Document Guide.''
    \item Daniel P.~Madden, ``Architecture Overview.''
    \item Daniel P.~Madden, ``Expansion Blueprint Register.''
    \item Daniel P.~Madden, ``Governance Map.''
    \item Daniel P.~Madden, ``Risk Taxonomy.''
    \item Daniel P.~Madden, ``Implementation Notes.''
    \item Daniel P.~Madden, ``The AI OSI Stack: A Governance Blueprint for Scalable and Trusted AI (README).''
    \item Daniel P.~Madden, ``AI OSI Stack --- Changelog.''
\end{enumerate}
\end{document}
