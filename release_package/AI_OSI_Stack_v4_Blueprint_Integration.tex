\documentclass[11pt]{article}
\usepackage[margin=1in]{geometry}
\usepackage{setspace}
\usepackage{lmodern}
\usepackage[T1]{fontenc}
\usepackage{hyperref}
\usepackage{enumitem}
\usepackage{tikz}
\usepackage{graphicx}
\usepackage{framed}
\usepackage{booktabs}
\usepackage{longtable}
\usepackage{float}
\usepackage{listings}
\usepackage{caption}
\usepackage{array}
\setstretch{1.2}
\hypersetup{
  colorlinks=true,
  linkcolor=blue,
  urlcolor=magenta,
  citecolor=blue
}
\newcommand{\license}[1]{\def\@license{#1}}
\makeatletter
\license{CC BY-NC-ND 4.0}
\title{The AI OSI Stack: Version 4 (Expanded — Blueprint Integration Release)}
\author{Daniel P. Madden}
\date{November 2025}
\newcommand{\doiplaceholder}{DOI: \texttt{10.0000/zenodo.placeholder}}
\lstdefinelanguage{json}{
  basicstyle=\ttfamily\small,
  breaklines=true,
  showstringspaces=false
}
\begin{document}
\begin{titlepage}
  \centering
  {\LARGE\bfseries The AI OSI Stack: Version 4\\(Expanded — Blueprint Integration Release)\\[1.5em]}
  {\large Canonical Blueprint Integration Reference\\[2em]}
  {\large Daniel P. Madden\\Author\\[1em]}
  {\large November 2025\\[2em]}
  {\large License: CC BY-NC-ND 4.0\\[1em]}
  {\large \doiplaceholder\\[4em]}
  \vfill
  \begin{minipage}{0.85\textwidth}
    \centering
    \textit{This publication integrates the AI OSI Stack v4 Test Integration draft, AEIP v1 transport specification, governance artifacts, and blueprint implementation notes into a single normative release prepared for archival submission.}
  \end{minipage}
  \vfill
\end{titlepage}
\pagenumbering{roman}
\begin{framed}
  \noindent\textbf{Normative Language Notice.} This specification uses normative language consistent with ISO/IEC 42010 and NIST conventions. “SHALL” denotes mandatory requirements, “SHOULD” denotes strong recommendations, and “MAY” denotes optional practices. Interpretations of this document must preserve authorial intent — fidelity to layered accountability, epistemic integrity, and human dignity as design constraints.
\end{framed}

\section*{Abstract}
\noindent\textit{Scope note: Summarizes the intent, coverage, and canonical artifacts of the integrated blueprint.}
This document consolidates Daniel P. Madden's AI OSI Stack v4 canon into a publication-grade reference that harmonizes the governance architecture, AI Epistemic Infrastructure Protocol (AEIP) transport, blueprint implementation, and risk controls required for accountable AI systems. It provides a layer-by-layer synthesis of design obligations, artifact schemas, and stewardship flows engineered for offline-first rehearsal and future standardization. Appendices capture the glossary, concordance mapping, authoritative change log, and references underpinning the release.

\tableofcontents
\listoffigures
\newpage
\pagenumbering{arabic}

\section{Introduction}
\noindent\textit{Scope note: Establishes background, lineage, and interpretive authority for the canonical release.}
Version 4 of the AI OSI Stack resolves earlier fragmentation by combining Persona Architecture, Epistemology by Design, and the AI Epistemic Infrastructure Protocol (AEIP) inside a single layered governance scaffold. The Test Integrated draft\footnote{Source: \textit{AI\_OSI\_Stack\_v4\_Test\_Integrated.md}.} provides the conceptual backbone, while the protocol, interface, and implementation notes extend the blueprint into executable rehearsal form. Jurisdictional neutrality remains foundational: conflicts in statutory or linguistic interpretation SHALL defer to the author's principles of transparency, accountability, and dignity by design. The repository intentionally operates offline-first, supplying deterministic schemas, governance artifacts, and integrity ledgers that organizations MAY adopt as rehearsal environments before any production instantiation.

\section{Architecture Overview}
\noindent\textit{Scope note: Describes the layered model, ethical guardrails, and offline blueprint constraints.}
The architecture encompasses seven mandatory governance layers with an optional civic mandate precursor (Layer 0). Each layer exposes canonical interfaces, primary artifacts, and ethical guardrails summarized in Table~\ref{tab:layer-summary}. Persona mandates, epistemic traceability, and ledger-grade accountability provide a triad of protection that enforces human dignity, reconstructable reasoning, and stewarded disclosure. Offline-only execution ensures reproducible audits and prevents premature coupling to live infrastructure. Organizations SHALL stage adoption by mapping existing controls to the stack, validating AEIP nodes in sandboxed conditions, and anchoring every modification in governance ledger entries.

\begin{table}[H]
  \centering
  \caption{Layered governance structure and primary artifacts.}
  \label{tab:layer-summary}
  \begin{tabular}{>{\raggedright\arraybackslash}p{1.8cm} >{\raggedright\arraybackslash}p{4.3cm} >{\raggedright\arraybackslash}p{4.8cm} >{\raggedright\arraybackslash}p{3.2cm}}
    \toprule
    \textbf{Layer} & \textbf{Core Objective} & \textbf{Canonical Interfaces} & \textbf{Primary Artifacts} \\
    \midrule
    L0 (Optional) & Civic mandate and social license & Legislative compacts, community oversight channels & Civic Directive, Governance Disclosure Statement (GDS) \\
    L1 & Secure compute foundations and supply integrity & Facility attestations, energy provenance feeds & GDS, Decision Rationale Record (DRR) \\
    L2 & Data stewardship and epistemic hygiene & Data lineage ledgers, consent registries & Interpretive Trace Package (ITP), DRR \\
    L3 & Model development rigor & Evaluation harness APIs, regression hooks & Oversight Action Memorandum (OAM), GDS \\
    L4 & Instruction and persona governance & Persona briefs, refusal logic manifests & ITP, DRR, GDS \\
    L5 & Reasoning exchange integrity & AEIP handshake headers, signed packets & Integrity Ledger Entry (ILE), OAM \\
    L6 & Deployment and integration assurance & Release manifests, incident playbooks & Temporal Review Record (TRR), OAM \\
    L7 & Governance publication and accountability & Disclosure portals, audit submission queues & GDS, ILE \\
    L8 (Overlay) & Policy harmonization and civic participation & Civic registry APIs, participatory audits & Civic Directive, ILE \\
    \bottomrule
  \end{tabular}
\end{table}

\section{Layered Architecture Detailing (L0--L8)}
\noindent\textit{Scope note: Enumerates the normative duties, risks, and evidence expectations for each layer.}
\subsection{Layer 0 — Civic Mandate}
\noindent\textit{Scope note: Captures the optional precursor requirements for legitimacy.}
The civic layer codifies jurisdictional legitimacy, community compacts, and public trust obligations. Civic directives SHALL articulate human rights clauses, oversight escalation pathways, and commitments for multi-stakeholder participation. Validity windows MUST be published through the governance portal, with updates notarized via AEIP handshakes to preserve temporal legitimacy.

\subsection{Layer 1 — Physical and Compute Substrate}
\noindent\textit{Scope note: Addresses hardware integrity, supply chain assurance, and resilience.}
Risks include hardware tampering, supply chain disruption, and resilience degradation. Sentinel indicators comprise access anomalies, unscheduled firmware changes, and energy draw deviations. Certified facility audits, redundant energy governance, and tamper-evident custody logs SHALL feed into GDS sections and DRR appendices. Temporal seals captured in ILE artifacts bind inspections to auditable timestamps.

\subsection{Layer 2 — Data Stewardship}
\noindent\textit{Scope note: Governs provenance, consent, and epistemic integrity of training corpora.}
Data poisoning, consent erosion, and epistemic contamination represent primary hazards. Consent traceability ledgers, dataset review boards, and epistemic hygiene playbooks SHALL remain active. ITP provenance bundles and DRR stewardship summaries MUST record dataset lineage alongside AEIP lineage signatures that permit auditors to reconstruct derivations.

\subsection{Layer 3 — Model Development}
\noindent\textit{Scope note: Ensures training rigor, evaluation breadth, and adversarial resilience.}
Bias amplification, evaluation blind spots, and adversarial regressions require composite evaluation harnesses, adversarial benchmarking cadences, and persona-aligned regression suites. Oversight Action Memoranda SHALL document evaluation actions and risk mitigations, while GDS validation annexes expose aggregated results to governance councils and the public.

\subsection{Layer 4 — Instruction and Control}
\noindent\textit{Scope note: Embeds persona mandates, refusal logic, and affective constraints.}
Persona Architecture enforces structured roles, refusal logic, and affect management so the AI remains within declared mandates. Prompt injection, persona drift, and affect misalignment trigger refusal pathways; AEIP instruction packets MUST include persona matrices and affect thresholds. Violations SHALL issue automatic OAM notices and route remediation through custodial councils.

\subsection{Layer 5 — Reasoning Exchange and Interface}
\noindent\textit{Scope note: Defines AEIP handshake execution, replay safeguards, and ledger guarantees.}
AEIP provides a five-step handshake (Intent, Justify, CounterSign, Commit, Update) with SHA3-512 hashing, Ed25519-compatible signatures, and monotonic temporal seals. Ledger desynchronization, packet replay, and counter-signature gaps are mitigated by deterministic replay checkpoints and quorum review across governance nodes. Successful handshakes MUST emit ILE bundles referencing canonical schema identifiers.

\subsection{Layer 6 — Deployment and Integration}
\noindent\textit{Scope note: Covers runtime assurance, change control, and incident rehearsal.}
Runtime drift, integration conflicts, and incident opacity are monitored via change gating, blue/green stewardship protocols, and incident rehearsal cadences. TRR release attestations SHALL capture approvals, and OAM incident records MUST summarize remediation timelines, lessons learned, and escalated risks for ledger anchoring.

\subsection{Layer 7 — Governance Publication}
\noindent\textit{Scope note: Manages transparency obligations, public accountability, and archival stewardship.}
Delayed disclosure, accountability gaps, and legitimacy erosion are addressed through scheduled public releases of GDS transparency packs, multi-stakeholder review councils, and archival notarization. ILE publication proofs MUST accompany each disclosure to provide cryptographic lineage.

\subsection{Layer 8 — Policy and Federation Overlay}
\noindent\textit{Scope note: Harmonizes federated governance, civic participation, and regulatory bridges.}
Optional policy overlays coordinate civic directives, regulatory submissions, and cross-jurisdiction collaboration. Proposed expansions such as the Persona-PKI, Open Governance Registry, RegOps Bridge, and Federation Testnet SHALL undergo quarterly governance review before activation. These initiatives ensure future readiness without compromising offline-first boundaries.

\section{AEIP v1 Transport Specification}
\noindent\textit{Scope note: Details handshake choreography, header semantics, integrity guarantees, and extensibility.}
AEIP v1 establishes a deterministic, persona-signed negotiation channel that links reasoning traces to ledger-grade artifacts. Mandatory headers include `aeipVersion`, `temporalSeal`, `personaSignature`, `governanceScope`, and `dignityCompliance`. Each handshake stage MUST incorporate provenance fields, persona identifiers, and monotonic temporal seals. Serialization uses canonical JSON, tolerates YAML for diagnostics, and requires lexicographic ordering before hashing.

\begin{figure}[H]
  \centering
  \begin{tikzpicture}[node distance=2.2cm,>=stealth,thick]
    \tikzstyle{step}=[rectangle,rounded corners,draw=blue!70,fill=blue!10,inner sep=6pt,text width=3.2cm,align=center]
    \node[step] (intent) {Intent\\(ITP attached)};
    \node[step,below of=intent] (justify) {Justify\\(DRR emitted)};
    \node[step,below of=justify] (counter) {CounterSign\\(Persona seal)};
    \node[step,below of=counter] (commit) {Commit\\(GDS minted)};
    \node[step,below of=commit] (update) {Update\\(OAM + ILE)};
    \draw[->] (intent) -- (justify);
    \draw[->] (justify) -- (counter);
    \draw[->] (counter) -- (commit);
    \draw[->] (commit) -- (update);
    \node[step,right=4.2cm of counter,text width=3.6cm] (integrity) {Integrity Controls\\SHA3-512 hash\\Ed25519 signatures\\Temporal seal drift checks};
    \draw[->,dashed] (intent) -- (integrity);
    \draw[->,dashed] (justify) -- (integrity);
    \draw[->,dashed] (counter) -- (integrity);
    \draw[->,dashed] (commit) -- (integrity);
    \draw[->,dashed] (update) -- (integrity);
  \end{tikzpicture}
  \caption{AEIP v1 handshake and integrity instrumentation.}
  \label{fig:aeip-handshake}
\end{figure}

Failure modes such as missing signatures or `dignityCompliance=false` SHALL abort the exchange and block ledger submission. Extensibility permits namespaced headers (`x-*`) and algorithm swaps, provided versions are negotiated during the Intent step. A canonical payload example is shown in Listing~\ref{lst:aeip-payload}. The integrity ledger SHALL capture the resulting transcript with deterministic hashes for replay and audit.

\begin{lstlisting}[language=json,caption={Canonical AEIP `Commit` payload excerpt.},label={lst:aeip-payload}]
{
  "aeipVersion": "1.0",
  "temporalSeal": "2025-11-30T18:45:22Z::c0ffee...",
  "personaSignature": {
    "algorithm": "Ed25519",
    "signature": "A1B2C3..."
  },
  "governanceScope": "blueprint-demo",
  "dignityCompliance": true,
  "payload": {
    "artifactType": "GDS",
    "artifactHash": "f1c9e6...",
    "obligations": ["transparency", "dignity"],
    "linkedArtifacts": ["ITP-2025-11-30", "DRR-2025-11-28"]
  }
}
\end{lstlisting}

\section{Governance Artifacts and Schemas}
\noindent\textit{Scope note: Defines the authoritative artifact classes and schema governance.}
Governance artifacts act as portable evidence across the stack:\footnote{Refer to `schemas/` directory for canonical JSON-LD definitions.}
\begin{itemize}[leftmargin=1.2cm]
  \item \textbf{Interpretive Trace Package (ITP):} Reconstructs evidentiary reasoning, preserving provenance and persona context.
  \item \textbf{Decision Rationale Record (DRR):} Captures architectural justification, risk trade-offs, and constraint acknowledgements.
  \item \textbf{Governance Disclosure Statement (GDS):} Public artifact summarizing compliance posture, Adaptive Governance Metrics (AGM), and change history.
  \item \textbf{Oversight Action Memorandum (OAM):} Documents interventions, remediation directives, and custodial sign-offs.
  \item \textbf{Integrity Ledger Entry (ILE):} Signed record anchoring governance claims in time, linking to supporting artifacts.
  \item \textbf{Temporal Review Record (TRR):} Validates release gates, drift indices, and change approvals.
\end{itemize}
Schemas reside alongside layer modules and enforce provenance (`layerId`, `temporalSeal`, `dignityCompliance`, `hash`) and canonical hashing rules. Validators SHALL reject payloads lacking monotonic temporal seals or persona signatures. Conformance tests located in `tests/test_layer_contracts.py` execute synthetic end-to-end flows to ensure IDD compliance, while `tests/test_aeip_handshake.py` verifies handshake replay integrity.

\section{Implementation Blueprint}
\noindent\textit{Scope note: Summarizes the offline reference implementation, toolchain, and conformance suite.}
The blueprint implementation demonstrates how layered governance becomes executable without live network services. Core directories include `/protocol/` (AEIP handshake logic and local governance ledger), `/schemas/` (machine-readable artifact definitions), `/src/layer1--8/` (modular validators per layer), `/tools/` (artifact generation utilities), and `/tests/` (conformance suites). Offline-first execution ensures deterministic rehearsals, with SHA3-512 hashing, persona identifiers, and UTC timestamps enforced across generated artifacts. Implementers SHALL treat this repository as a canonical rehearsal environment: design decisions are finalized, but production deployment awaits custodial approval.

\section{Governance and Future Expansion}
\noindent\textit{Scope note: Explores PKI extensions, federation pathways, and regulatory bridges.}
Future expansion blueprints extend governance capabilities while preserving the stack's normative commitments:
\begin{itemize}[leftmargin=1.2cm]
  \item \textbf{Persona-PKI:} Binds Persona Architecture roles to cryptographic identities, yielding revocable certificates and cross-layer trust anchors.
  \item \textbf{Open Governance Registry (OGR):} Federated disclosure catalogue for publishing GDS, ILE, and OAM artifacts under custodial oversight.
  \item \textbf{RegOps Bridge:} Adapter suite translating AEIP artifacts into regulatory submission formats without diluting normative clauses.
  \item \textbf{Federation Testnet:} Multi-jurisdiction sandbox validating ledger replay, escalation workflows, and distributed oversight resilience.
\end{itemize}
Governance councils SHALL review each initiative quarterly, advancing only when risk controls, human oversight protocols, and archival requirements align with the canonical specification. Persona-PKI development anticipates integration with AEIP signatures; federation roadmaps coordinate layered trust across sandboxed nodes before production roll-out.

\section{Governance Artifacts \& Oversight Structures}
\noindent\textit{Scope note: Maps risk controls, stewardship roles, and oversight cadence across layers.}
The governance risk map aligns primary risks, controls, and evidence expectations (Table~\ref{tab:risk-map}). Councils SHALL maintain the register as a living document, tailoring controls only where equivalent rigor, transparency, and dignity-first obligations remain intact. Distributed oversight architecture tiers (local nodes, regional custodians, global trust anchor) balance autonomy with assurance, enabling mutual recognition agreements via AEIP-secured exchanges.

\begin{table}[H]
  \centering
  \caption{Layer-to-governance risk alignment.}
  \label{tab:risk-map}
  \begin{tabular}{>{\raggedright\arraybackslash}p{2.4cm} >{\raggedright\arraybackslash}p{4.2cm} >{\raggedright\arraybackslash}p{4.8cm} >{\raggedright\arraybackslash}p{3.2cm}}
    \toprule
    \textbf{Layer} & \textbf{Primary Risks} & \textbf{Example Controls} & \textbf{Evidence Artifacts} \\
    \midrule
    L1 & Hardware tampering, supply opacity, resilience failure & Certified facility audits, redundant energy governance, tamper-evident custody & GDS, DRR \\
    L2 & Consent erosion, provenance drift, epistemic contamination & Consent traceability ledgers, dataset review boards, epistemic hygiene playbooks & ITP, DRR \\
    L3 & Bias amplification, evaluation blind spots, adversarial regressions & Composite evaluation harnesses, adversarial benchmarking, persona regression & OAM, GDS \\
    L4 & Prompt injection, persona drift, affect misalignment & Persona mandate enforcement, refusal logic tests, affect boundary audits & ITP, DRR, GDS \\
    L5 & Ledger desynchronization, packet replay, signature gaps & AEIP validation, deterministic replay checkpoints, cross-node quorum review & ILE, OAM \\
    L6 & Runtime drift, change conflicts, incident opacity & Governance gating, blue/green stewardship, incident rehearsal & TRR, OAM \\
    L7 & Delayed disclosure, accountability gaps, legitimacy erosion & Disclosure cadence, multi-stakeholder review, archival notarization & GDS, ILE \\
    \bottomrule
  \end{tabular}
\end{table}

Adaptive Governance Metrics (Transparency Ratio, Governance Coverage Score, Drift Index, Dignity Compliance Rate) SHALL be reported quarterly. Cognitive Diversity Index targets (\(\geq 0.4\) for high-stakes domains) ensure epistemic plurality; results MUST accompany IMM and AGM disclosures. Governance simulations, supply chain attestations, and human-in-the-governance-loop checkpoints provide additional assurance mechanisms with outputs anchored in AEIP ledgers.

\section{Implementation Blueprint Testing}
\noindent\textit{Scope note: Highlights conformance suites, ledger validation, and governance fire drills.}
Conformance testing spans unit validators, cross-layer contract tests, and AEIP replay verification. `tests/test_layer_contracts.py` exercises schema compliance for each Interface Definition Document (IDD), while `tests/test_aeip_handshake.py` validates handshake monotonicity and drift detection. Governance fire drills simulate compound failures—synthetic data breaches, policy shifts, model drift—to rehearse escalation protocols. Outputs compile into OAM Resilience Reports summarizing response quality, time-to-remediation, and improvement actions for custodial councils. Integrity ledgers SHALL capture rehearsal transcripts, enabling auditors to replay events deterministically.

\section{Federation and RegOps Roadmap}
\noindent\textit{Scope note: Details PKI enablement, federated ledgers, and regulatory adapters beyond the core stack.}
Public-key infrastructure for personas, federated governance registries, and RegOps bridges extend accountability across jurisdictions. Persona certificates SHALL encode mandate boundaries and refusal logic commitments, with revocation manifests published via AEIP. Federation testnets MUST validate ledger synchronization, escalation workflows, and cognitive diversity reporting prior to connecting production nodes. Regulatory adapters SHALL maintain semantic fidelity, exporting AEIP transcripts to supervisory portals without altering normative clauses.

\section{Conclusion}
\noindent\textit{Scope note: Reinforces the architecture-as-accountability thesis and custodial stewardship.}
The AI OSI Stack v4 blueprint positions architecture as accountability: each layer binds technical controls to governance evidence, ensuring dignity, epistemic integrity, and stewardship coexist with system functionality. By releasing a blueprint-complete, offline rehearsal environment with deterministic artifacts, Daniel P. Madden provides institutions with the scaffolding required to internalize governance before deployment. Custodial councils, ledger attestations, and persona mandates transform oversight from reactive audit to design-time infrastructure.

\appendix
\section{Glossary (Appendix A)}
\noindent\textit{Scope note: Defines canonical terms and abbreviations.}
\begin{description}[leftmargin=1.5cm,labelwidth=1cm]
  \item[AEIP] AI Epistemic Infrastructure Protocol providing signed reasoning transport with ledger anchoring.
  \item[AGM] Adaptive Governance Metrics encompassing Transparency Ratio, Governance Coverage Score, Drift Index, and Dignity Compliance Rate.
  \item[CDI] Cognitive Diversity Index measuring epistemic plurality across governance decisions.
  \item[DRR] Decision Rationale Record preserving architectural justification and risk considerations.
  \item[GDS] Governance Disclosure Statement summarizing compliance posture, metrics, and change history.
  \item[ILE] Integrity Ledger Entry capturing temporally sealed governance evidence.
  \item[IMM] Implementation Maturity Model guiding staged adoption of stack obligations.
  \item[ITP] Interpretive Trace Package reconstructing reasoning lineage and provenance.
  \item[OAM] Oversight Action Memorandum documenting interventions and remediation directives.
  \item[TRR] Temporal Review Record validating release readiness, drift indices, and custodial approvals.
\end{description}

\section{Layer-to-Standard Concordance (Appendix B)}
\noindent\textit{Scope note: Maps AI OSI layers to external standards and regulatory frameworks.}
\begin{longtable}{>{\raggedright\arraybackslash}p{3cm} >{\raggedright\arraybackslash}p{5cm} >{\raggedright\arraybackslash}p{6cm}}
  \caption{Concordance between AI OSI layers and external standards.}\\
  \toprule
  \textbf{Layer} & \textbf{Aligned Standards} & \textbf{Notes} \\
  \midrule
  L1 & ISO/IEC 27001, NIST CSF & Align physical security, resilience, and supply chain governance.
  \\
  L2 & NIST AI RMF (Data), GDPR Articles 13--22 & Emphasize provenance, consent, and contestability.
  \\
  L3 & ISO/IEC 42001 Clause 8, NIST AI RMF (Measure) & Combine model evaluation rigor with governance reporting.
  \\
  L4 & EU AI Act transparency requirements, Persona Architecture guidelines & Maintain persona declarations, refusal logic, and affective constraints.
  \\
  L5 & TRUST Framework interoperability principles, AEIP handshake specification & Ensure protocol compatibility and signed reasoning exchanges.
  \\
  L6 & ITIL Change Management, NIST SP 800-53 (IR controls) & Govern change gates, incident response, and deployment assurance.
  \\
  L7 & Open Government data publication norms, public accountability statutes & Publish disclosures, ledger attestations, and oversight records.
  \\
  L8 & Regional civic participation laws, data fiduciary regulations & Harmonize optional policy overlays with community mandates.
  \\
  \bottomrule
\end{longtable}

\section{Change Log (Appendix C)}
\noindent\textit{Scope note: Reproduces the authoritative version history for the blueprint release.}
\begin{itemize}[leftmargin=1.2cm]
  \item \textbf{v1.0 — Foundational Stack Overview (2019-09-01):} Introduced the seven-layer governance metaphor, integrity ledger concept, and initial stewardship obligations for AI services.
  \item \textbf{v2.0 — Persona Architecture Expansion (2020-12-15):} Added Persona Architecture briefs, affective constraint models, and role-specific accountability patterns.
  \item \textbf{v3.0 — Epistemology Alignment (2022-03-10):} Integrated Epistemology by Design, introduced Interpretive Trace Packages, and prepared AEIP compatibility guidance.
  \item \textbf{v4.0 — Integrated Canonical Specification (2024-07-22):} Published \emph{AI OSI Stack v4 Test Integrated}, harmonizing governance artifacts, temporal legitimacy, and the Implementation Maturity Model.
  \item \textbf{v4-protocol-blueprint — Blueprint Complete Reference Implementation (2025-11-30):} Finalized offline-first blueprint codebase, added governance maps, deferred system registry, validation artifacts, and Zenodo-ready release package for archival DOI issuance.
\end{itemize}

\section{References (Appendix D)}
\noindent\textit{Scope note: Lists cited works and governing sources.}
\begin{enumerate}[leftmargin=1.2cm]
  \item Daniel P. Madden, \emph{Persona Architecture: Designing Role-Specific AI Systems for Accountability and Trust}.
  \item Daniel P. Madden, \emph{Epistemology by Design: Embedding Reasoning Integrity in AI Systems}.
  \item Daniel P. Madden, \emph{The AI OSI Stack: A Governance Blueprint for Scalable and Trusted AI}.
  \item Daniel P. Madden, \emph{AEIP-00: The AI Epistemic Infrastructure Protocol}.
  \item National Institute of Standards and Technology, \emph{AI Risk Management Framework} (2023).
  \item ISO/IEC, \emph{42001:2023 Artificial Intelligence Management System}.
  \item European Union, \emph{Artificial Intelligence Act} (2024 provisional agreement).
  \item U.S. Department of Commerce, \emph{TRUST Framework} (2024).
\end{enumerate}

\end{document}
